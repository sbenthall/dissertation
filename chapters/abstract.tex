 \documentclass[../thesis.tex]{subfiles}
 \begin{document}

The creators of technical infrastructure are under social and legal pressure to comply with expectations that can be difficult to translate into the practice of engineering.
This dissertation aims to bridge this gap by progressively working at the tensions between social theory, law, computer science, and economics.
The three projects of the dissertation focus on privacy engineering, information security, and data economics, respectively.
The projects logically follow on each other and culminate in a new way of modeling problems and solutions in each of these domains.
This new modeling framework hinges on a specific conception of information flow in a causal context that builds on the theoretical contributions of Claude Shannon, Fred Dretske, Judea Pearl, Daphne Koller, and Helen Nissenbaum.

The first project included in this dissertation concerns the use
of Contexual Integrity in the discipline of computer science.
The field of Contextual Integrity accounts for society's expectations of,
disappointments in, and adaptations to information technology by
positing that privacy is contextually appropriate
information flow.
Social contexts are abstract, normatively oriented spheres of social activity
that evolve as a consequence of our ends, purposes, and values.
A survey of how computer scientists have interpreted and implemented
Contextual Integrity reveals both disciplinary blindspots that prevent
computer scientists from fully engaging with Contextual Integrity
as well as theoretical
gaps that must be filled to make Contextual Integrity more actionable for
technologists. This work was done in collaboration with Seda G{\"u}rses
and Helen Nissenbaum.

While social expectations and legal requirements assume that the semantics of information is fixed an narrowly identifiable, in reality data often carries a broader semantics that is discoverable using machine learning techniques.
The second project explores the efficacy of data protection
rules based on information origin, or Origin Privacy, as
a complement to data protection rules based on information
semantics.
A new conception of information flow that builds on the
work of Claude Shannon, Fred Dretske, Judea Pearl, and
Helen Nissenbaum is proposed.
In this conception, information flow is a property of a
system of causal relations, and consists of a causal flow
(a direct relation between probabilistic events) and nomic
associations of the effect, meaning the conditional dependencies
of the effect on other variables within the system.
This conception reveals why information flows can be revealing
in unanticipated ways as well as why information's origin matters
to its normative flow.
This is done using the well-understood statistical formalism
of causal Bayesian networks.

Modeling information flow through computational systems with
causal Bayesian networks provides a unifying framework for
understanding information security and privacy,
as is shown by expressing noninterference, semantic security,
origin privacy, differential privacy in it.
This analysis reveals a particular class of security and privacy
failure that results from engineering according to a security
criteria defined in terms of a causal intervention on the system
inputs.

The first project revealed that an adequate theory of privacy
must take into account cross-context information flows
and the implications this has for for design of technical
infrastructure.
The second project introduced a new model of information flow
as a property of a causal system.
The third project builds on these results and considers the
problem of modeling the impact of strategic economic decisions
involving information flow.
The earlier model of information is extended into a framework
for mechanism design by building on Koller and Milch's
Multi-Agent Influence Diagram formalism.
This framework can capture the strategic dynamics and consequences
of well-known information economic situations such as
principal-agent contracts and price differentiation, as
well as new economic models such as expert services and
cases of cross-context information flow.
These models have *****
\end{document}
