 \documentclass[../thesis.tex]{subfiles}
 \begin{document}

 The creators of technical infrastructure are under social and legal pressure to comply with expectations that can be difficult to translate into computational and business logics.
 This dissertation bridges this gap through three projects that focus on privacy engineering, information security, and data economics, respectively.
 These projects culminate in a new formal method for evaluating the strategic and tactical value of data: data games.
 This method relies on a core theoretical contribution building on the work of Shannon, Dretske, Pearl, Koller, and Nissenbaum: a definition of information flow as causal flow in a system
 causal relations and strategic choices.

 The first project studies privacy engineering's use of
 Contextual Integrity theory (CI), which defines
 privacy as appropriate information flow according to norms
 specific to social contexts or spheres.
 Computer scientists using CI
 have innovated as they have implemented the theory and
 blended it with other traditions, such as context-aware
 computing. This survey examines
 computer science literature using Contextual Integrity and
 discovers, among other results, that technical and social platforms
 that span social contexts challenge CI's current commitment
 to normative social spheres.
 Sociotechnical situations can and do defy social expectations
 with cross-context clashes, and privacy engineering
 needs its normative theories to acknowledge and address this
 fact.
 
 This concern inspires the second project, which addresses the
 problem of building computational systems that comply with data
 flow and security restrictions such as those required by law.
 Many privacy and data protection policies stipulate
 restrictions on the flow of information based on that
 information's original source.
 We formalize this concept of privacy as Origin Privacy.
 This formalization shows how information flow security
 can be represented using causal modeling.
 Causal modeling of information security leads to
 general theorems about the limits of privacy by design
 as well as a shared language for representing specific
 privacy concepts such as noninterference, differential
 privacy, and authorized disclosure.

 The third project uses the causal modeling of information
 flow to address gaps in current theory of data economics.
 Like CI, privacy economics has focused on individual
 economic contexts and so has been unable to comprehend
 an information economy that relies on the flow of
 information across contexts.
 Multi-Agent Influence Diagrams, a game theoretic extension
 of Bayesian Networks, are used to model the well known
 economic contexts of principal-agent contracts and
 price differentiation as well as new contexts
 such as personalized expert services and data reuse.
 This work reveals that information flows are not
 goods but rather strategic resources involving
 many market externalities.
 
 \end{document}
