 \documentclass[../thesis.tex]{subfiles}
 \begin{document}

Multi-Agent Influence Diagrams (MAIDs) are a game-theoretic
extension of Bayesian networks developed by Koller and Milch
\cite{koller2003multi}.
A MAID is defined by:
\begin{enumerate}
\item A set $\mc{A}$ of agents 
\item A set $\mc{X}$ of chance variables
\item A set $\mc{D}_a$ of decision variables for each agent $a \in \mc{A}$,
  with $\mc{D} = \bigcup_{a \in \mc{A}} \mc{D}_a$
\item A set $\mc{U}_a$ of utility variables for each agent $a \in \mc{A}$,
  with $\mc{U} = \bigcup_{a \in \mc{A}} \mc{U}_a$
\item A directed acyclic graph $\mc{G}$ that defines the parent function
  $Pa$ over $\mc{V} = \mc{X} \cup \mc{D} \cup{U}$
\item For each chance variable $X \in \mc{X}$, a CPD $Pr(X \vert Pa(X))$
\item For each utility variable $U \in \mc{U}$, a CPD $Pr(U \vert Pa(U))$
\end{enumerate}

The decision variables represent moments where agents can
make decisions about how to act given only the information
provided by the variable's parents.

\begin{dfn}[Decision rules]
  \label{dfn:decision-rule}
  A \emph{decision rule} $\delta$ is a function that maps each instantiation
  $\vec{pa}$ of $Pa(D)$ to a probability distribution over $dom(D)$.
\end{dfn}

\begin{dfn}[Strategy]
  \label{dfn:strategy}
  An assignment of decision rules to every decision $D \in \mc{D}_a$
  for a particular agent $a \in \mc{D}_a$ for a particular agent
  $a \in \mc{A}$ is called a \emph{strategy}.
\end{dfn}

\begin{dfn}[Strategy profile]
  An assignment $\sigma$ of decision rules to every decision
  $D \in \mc{D}$ is called a \emph{strategy profile}.
  A \emph{partial strategy profile} $\sigma_\mc{E}$ is
  an assignment of decision rules to a subset $\mc{E} \subset \mc{D}$.
  $\sigma_{-\mc{E}}$ refers to a restriction of $\sigma$ to variables
  not in $\mc{E}$.
\end{dfn}

Decision rules are of the same form as CPDs, and so a MAID
can be transformed into a Bayes network by replacing every
decision variable with a random variable with the CPD of the
decision rule of a strategy profile.

\begin{dfn}
  If $\mc{M}$ is a MAID and $\sigma$ is a strategy profile for
  $\mc{M}$, then the \emph{joint distribution for $\mc{M}$
    induced by $\sigma$}, denoted $P_{\mc{M}[\sigma]}$, is the
  joint distribution over $\mc{V}$ defined by the Bayes
  net where:
  \begin{itemize}
  \item the set of variables is $\mc{V}$;
  \item for $X, Y \in \mc{V}$, there is an edge $X \rightarrow Y$
    if and only if $X \in Pa(Y)$;
  \item for all $X \in \mc{X} \cup \mc{U}$, the CPD for $X$ is $Pr(X)$;
  \item for all $D \in \mc{D}$, the CPD for $D$ is $\sigma(D)$.
  \end{itemize}
\end{dfn}

\begin{dfn}
  Let $\mc{E}$ be a subset of $\mc{D}_a$ and let $\sigma$ be a strategy
  profile.
  We say that $\sigma*_\mc{E}$ is \emph{optimal for the strategy profile}
  $\sigma$ if, in the induced MAID $\mc{M}[\sigma_{-\mc{E}}]$,
  where the only remaining decisions are those in $\mc{E}$,
  the strategy $\sigma*_{\mc{E}}$ is optimal, i.e., for all
  strategies $\sigma'_{\mc{E}}$:
  $$EU_a((\sigma_{-\mc{E}},\sigma*_{\mc{E}})) \geq EU_a((\sigma_{\mc{E}}, \sigma'_{\mc{E}}))$$
\end{dfn}

A major contribution of \cite{koller2003multi} is their analysis
of how to efficiently discover Nash Equilibrium strategy profiles
for MAIDs.
Their method involves analyzing the qualitative graphical
structure of the MAID to discover the \emph{strategic reliance}
of decision variables.
When a decision variable $D$ strategically relies on $D'$,
then in principle the choice of the optimal decisionr rule for
$D$ depends on the choice of the decision rule for $D'$.

\begin{dfn}[Strategic reliance]
  \label{dfn:strategic-reliance}
  Let $D$ and $D'$ be decision nodes in a MAID $\mc{M}$.
  $D$ \emph{strategically relies on} $D'$ if there exist
  two strategy profiles $\sigma$ and $\sigma'$ and a
  decision rule $\delta$ for $D$ such that:
  \begin{itemize}
  \item $\delta$ is optimal for $\sigma$;
  \item $\sigma'$ differs from $\sigma$ only at $D'$;
  \end{itemize}
  but no decision rule $\delta*$ that agrees with $\delta$ on
  all parent instantiations $\vec{pa} \in dom(Pa(D))$
  where $P_{\mc{M}[\sigma]}(\vec{pa}) > 0$ is optimal for $\sigma'$.
\end{dfn}

\begin{dfn}[s-reachable]
  \label{dfn:s-reachable}
  A node $D'$ is \emph{s-reachable} from a node $D$ in a MAID
  $\mc{M}$ if there is some utility node $U \in \mc{U}_D$ such
  that if a new parent $\widehat{D'}$ were added to $D'$, there would
  be an active path in $\mc{M}$ from $\widehat{D'}$ to $U$ given
  $Pa(D) \cup \{D\}$, where a path is active in a MAID if it
  is active in the same graph, viewed as a BN.
\end{dfn}

\begin{thm}
  \label{thm:strategic-non-reliance}
  If $D$ and $D'$ are two decision nodes in a MAID $\mc{M}$
  and $D'$ is not s-reachable from $D$ in $\mc{M}$, then D
  does not strategically rely on $D'$.
\end{thm}

\subsection{Tactical independence}

In our analysis in this paper, we will not compute
equilibrium strategies for MAIDs.
Rather, we will focus on a newly defined property
of MAIDs, tactical independence.

\begin{dfn}[Tactical independence]
  \label{dfn:tactical-independence}
  For decision variables $D$ and $D'$ in MAID $\mc{M}$,
  $D$ and $D'$ are \emph{tactically independent} for
  conditioning set $\mc{C}$ iff
  for all strategy profiles $\sigma$ on $\mc{M}$,
  in $P_{\mc{M}[\sigma]}$, the joint distribution for
  $\mc{M}$ induced by $\sigma$,
  $$D \independent D' \vert C$$
\end{dfn}

Because tactical independence depends on the
independence of variables on an induced probability
distribution that is representable by a Bayesian
network, the d-separation tests for independence
apply readily.

\begin{thm}
  For decision variables $D$ and $D'$ in MAID $\mc{M}$,
  and for conditioning set $\mc{C}$, if
  $D$ and $D'$ are d-separated given $\mc{C}$ on
  $\mc{M}$ considered as a Bayesian network,
  then $D$ and $D'$ are tactically independent
  given $\mc{C}$.
\end{thm}

\begin{proof}
  Suppose $D$ and $D'$ are d-separated given $\mc{C}$
  on $\mc{M}$ considered as a Bayesian network.

  For any strategy profile $\sigma$,
  the joint distribution for $\mc{M}$
  induced by $\sigma$, $P_{\mc{M}[\sigma]}$
  has the same graphical structure as $\mc{M}$
  considered as a Bayesian network.

  Therefore, $D$ and $D'$ are d-separated given $\mc{C}$
  in the graph corresponding to $P_{\mc{M}[\sigma]}$
  for all $\sigma$.

  Because $D$ and $D'$ are d-separated given $\mc{C}$
  in the Bayesian network, $D \independent D' \vert C$.
\end{proof}

% Utility CPDs are deterministic
% Utility nodes have to be leaves (I bend this in my models!)

\subsection{Notation}
\label{sec:maid-notation}

We will use a slightly different graphical notation than that used by
\cite{koller2003multi}

In the models in this paper, we will denote random variables
with undecorated capital letters, e.g. $A, B, C$.
I will denote strategic nodes with a tilde over a capital
letter, e.g. $\tilde{A}, \tilde{B}, \tilde{C}$.
The random variable defined by the optimal strategy at a
decision node, when such a variable is well-defined,
will be denoted with a hat, e.g. $\hat{A}, \hat{B}, \hat{C}$.
Nodes that represent the payoff or utility to an
agent will be denoted with a breve, e.g.
$\breve{A}, \breve{B}, \breve{C}$.
Particular agents will be identified by a lower case
letter and the assignment of strategic and utility nodes
to them will be denoted by subscript.
E.g., $\tilde{A}_q$ and $\breve{U}_q$ denote an action
taken by agent $q$ and a payoff awarded to $q$,
respectively.

A dotted arrow in a diagram indicates an optional arrow.
They indicate that two separate models, one including the
arrow and one without, are being considered.
When considering an instantiation of the model with the dotted
edge present, we will say the model or edge is \emph{open}.
When the edge is absent, we'll say it's \emph{closed}.

\subsection{Value of Data}
\label{sec:value-of-data}

Using the MAID framework, we can develop a
the tools to analyze the value of data.

We proceed with several considerations.
First, we note that the meaning of data depends
on the context in which is flows.
This is a consequence of the fact that information
flow consists of both causal flow and the nomic
associations of the information, which are due
to the causal structure in which the flow takes
place.
Similarly, we cannot determine the value of data
except in the context of a system of causal relations
that give data its meaning.

Second, data is valuable not primarily as a good to
be consumed but as a resource for determining
correct actions.
The value of data is in its strategic and/or tactical
value.
This means that data is valuable only in so far as it
is available to an active agent.
The value of the data to that agent will be the value
of the strategic or tactical advantage that the data
provides.

Third, as a consequence of the second point,
the effect of data on an agent's action may or may
not effect outcomes for other agents.
The value of a particular flow of data to an agent
may have negative value \emph{to other agents}.

These three points can be seen by an analysis of the MAID
framework we have introduced in this section.
In this framework, the flow of data is represented by
an edge in the causal graph.
The value of a data flow can be computed as the difference
in utility to each agent between the open and closed cases
in the game.

\subsubsection{Strategic and tactical value}

As we have distinguished between strategic reliance and
tactical independence, we can distinguish between the
strategic and tactical value of information.

The strategic value of an information flow to an agent
is the difference in utility to that agent in the open
and closed conditions of the game, given each game
is at strategic equilibrium for all players.

\begin{dfn}[Strategic value of information]
  \label{dfn:strategic-value}
  Given two MAID diagrams $\mc{M}_{o}$ and $\mc{M}_{c}$
  that differ only by a single edge, $e$,
  and a strategic profile for each diagram, $\sigma_{o}$
  and $\sigma_{c}$, the \emph{strategic value of $e$ to $a$}
  is the difference in expected utility to $a$ under the
  two respective induced joint distributions:

  $$E(P_{\mc{M}_{o}[\sigma_{o}]}(U_a)) - E(P_{\mc{M}_{c}[\sigma_{c}]}(U_a))$$
\end{dfn}

*** This is a sketch definition.
It is has some notational problems, and it does
not match the descriptive text above it because there
is not yet a definition of an equilibrium strategy ***

In contrast with the strategic value of information,
the tactical value of information is the value of
the information to an agent given an otherwise fixed
strategy profile.
We allow the agent receiving the data to make a tactical
adjustment to their strategy at the decision variable
at the head of the new information flow.

\begin{dfn}[Best tactical response to information]
  Given two MAID diagrams $\mc{M}_{o}$ and $\mc{M}_{c}$
  differing only in optional edge $e$ with head in decision
  variable $D$,
  the \emph{best tactical response to $e$} given
  strategy profile $\sigma$, $\hat{\delta}_{\sigma,e}$
  is the decision rule $\delta$ for $D$ such
  that $\delta$ is optimal for $\sigma$
\end{dfn}

*** Problem: this definition assumes a unique best $\delta$
which may not exist.***

\begin{dfn}[Tactical value of information]
  \label{def:tactical-value}
  Given two MAID diagrams $\mc{M}_{o}$ and $\mc{M}_{c}$
  differing only in optional edge $e$ with head in decision
  variable $D$,
  the \emph{tactical value of $e$} to agent $a$ given
  strategy profile $\sigma$
  is the difference in expected utility of
  the open condition with the best tactical response to $e$
  and the close condition using the original strategy:

  $$EU_a((\sigma_{-D},\hat{\delta}_{\sigma,e}) - EU_a(\sigma)$$
\end{dfn}

 
\end{document}
