 \documentclass[../thesis.tex]{subfiles}
 \begin{document}

 The \textit{sina non qua} figure in the production of
 this dissertation is the inimitable Michael Carl Tschantz.
 I will be grateful for life to him for getting me hired by
 Helen Nissenbaum as a researcher on DARPA agreement number FA8750-16-2-0287.
 This act snatched me from the jaws of failure and delivered me
 to New York City, my home, where I could work towards completion with renewed vigor.

 Michael is a man who has, to use the Platonic metaphor,
 left the cave and seen the world in sunlight.
 He has been a great teacher to me in mathematical discipline,
 and his uncompromising feedback has allowed me to
 achieve a  level of rigor that I can be proud of.

 Helen is a firm taskmaster in addition to a visionary thinker.
 Her intellectual courage and ambition gave me the permission I needed to express my own.
 I thank her.
 
 Chapter \ref{chapter:CI-through-CS} of this dissertation began as the brainchild
 of Helen Nissenbaum and the brilliant Seda G{\"u}rses.
 It is with their permission that I include our collaborative work here.
 Seda has been another great teacher to me,
 and is proof that intellectually
 one \textit{can} have it all:
 instrumental effectiveness, moral value, social relevance,
 and aesthetic beauty.
 Through this work I've learned that what matters in
 scholarship is not ones individual powers but rather
 ones complementarity with collaborators of high caliber.

 Chapter \ref{chapter:origin-privacy} was originally written
 as a technical report in collaboration with Michael Tschantz
 with guidance from Anupam Datta.
 It took several years of graduate study to discover my
 fascination with the intersection of computer science
 research and the law.
 I thought it was a far-fetched dream.
 Then I met Anupam,
 who had been doing it successfully for over a decade.
 I aspire to one day have a perspective as correct as his.

 Chapter \ref{chapter:games-and-rules} contains what I feel
 is this dissertation's most original contribution:
 a framework for modeling the consequences of information flow
 in strategic games.
 The seed of this work was nurtured by John Chuang
 in my first year at the I School, and my collaboration
 with him on ``Strategic Bayesian Networks'' remains one of
 my happiest memories of Berkeley.
 Years later John became my advisor.
 I thank him for years of patience, honesty, sensitivity,
 kindness, and clarity of purpose.

 I am also deeply grateful to David Wolpert, who discovered
 my work with John on ``Strategic Bayesian Networks''
 and emailed me to inform me that the work had already
 been done by Daphne Koller ten years prior as
 ``Multi-Agent Influence Diagrams'' (MAIDs).
  MAIDs are now a cornerstone of the theoretical contribution
 of this dissertation.
 Dr. Wolpert's work, which unflinchingly tackles some
 of the foundational questions of physics and its ramifications
 in the social world, is source of inspiration to me.
 
 What words can do justice to Deirdre Mulligan?
 I thank her.

 I acknowledge David Wagner for accepting an
 inconvenient responsibility to help a stranger in need.

 So many colleagues at Berkeley helped me along my way.
 In particular, Daniel Aranki, Nick Doty, and Jen King
 for teaching me about privacy through their research
 before I ever knew I would join them in their field
 of study.

 Completing a dissertation at the remarkable institution
 of UC Berkeley is as much a bureaucratic challenge as it
 is an intellectual one.
 It has been all the more challenging for me as I
 have conducted most of this business from New York.
 This would have been impossible without the vigor,
 availability, and thoroughness of Catherine
 Cronquist Browning.
 
 Attendees at the 10th Annual Privacy Law Scholars Conference
 (2017) gave their helpful comments on
 Chapters \ref{chapter:CI-through-CS} and
 \ref{chapter:origin-privacy}.
 Special thanks goes to Blase Ur and Michael Hintze,
 our discussants.
 Relatedly, I am grateful to Chris Hoofnagle for his warm
 welcome into that community and early advice
 on how to contribute to privacy scholarship.

 I thank Ignacio Cofone, Dr. Cathy Dwyer, Yafit Lev-Aretz,
 Amanda Levendowski, John Nay, Dr. Julia Powles,
 Ira Rubinstein, Madelyn Sanfilippo, Andrew Selbst,
 Yan Shvartzshnaider,
 Katherine Strandberg, Ari Waldman, Elana Zeide,
 Bendert Zevenberger,  
 and other members of the Privacy Research Group at
 the NYU School of Law for their helpful comments
 on Chapter \ref{chapter:games-and-rules}.
 
 I gratefully acknowledge funding support 
 from the U.S. Defense Advanced Research Projects Agency (DARPA)
 under award FA8750-16-2-0287.
 The opinions in this paper are those of the author and do not
 necessarily reflect the opinions of any funding sponsor
 or the United States Government.
 I am also grateful to Tracy Figueroa and Dr. Finn Brunton
 for continuing to administer this grant from NYU after
 Helen transitioned to Cornell Tech.
 
 I also gratefully acknowledge John Scott and JC Herz
 at Ion Channel
 for taking me into their company as a data scientist and
 not flinching when completing my doctorate took much,
 much longer than expected.
 Their support as employers have meant much more than the paycheck,
 which I depended on for much of my dissertation work.
 Their understanding of business, technology, and government
 rivals that of any professors who could advise me
 and contributed significantly to the framing and direction
 of this work.

 There is an old Budokon saying, ``How you do anything
 is how you do everything.''
 My academic life improved immeasurably when I began
 to train physically in movement arts.
 I thank Professor Felippe, Sensai Petra, and Coach Mike
 from Elements Athletics for their guidance,
 support of my well-being, and their introduction
 into new worlds of embodied thought.

 I thank Dr. Yacine Kouyate for his comments on
 the mathematics of the Cosmos, and Floyd Tolliver
 and Allyson Erick for all the parties at
 Birdland Jazzista Social Club.
 
 Lastly, I thank Daria, my beloved wife,
 who has made it all worth it.
\end{document}
