 \documentclass[../thesis.tex]{subfiles}
 \begin{document}

 The \textit{sina non qua} figure in the production of this dissertation is the inimitable Dr. Michael Carl Tschantz.
 I will be grateful for life to him for getting me hired by Dr. Helen Nissenbaum as a researcher on DARPA agreement number FA8750-16-2-0287.
 This act snatched me from the jaws of failure and delivered me to New York City, my home, where I could work towards completion with renewed vigor.

 Michael is a man who has, to use the Platonic metaphor,
 left the cave and seen the world in sunlight.
 He has been a great teacher to me.
 His insistence on mathematical rigor and his clear
 articulation of the weaknesses of my arguments, again and again,
 me to reach for the level of rigor
 that I can be proud of.

 Helen is a firm taskmaster in addition to a visionary thinker.
 Her intellectual courage and ambition gave me the permission I needed to express my own.
 I thank her.
 
 Chapter \ref{chapter:CI-through-CS} of this dissertation began as the brainchild of Helen Nissenbaum and the brilliant Dr. Seda {\"u}rses.
 It is with their permission that I include our collaborative work here.
 Seda has been another great teacher to me,
 and is proof that in ones intellectual pursuits
 one \textit{can} have it all:
 instrumental effectiveness, moral value, social relevance,
 and aesthetic beauty.
 Through this work I've learned that what matters in
 scholarship is not ones individual powers but rather
 ones complementarity with collaborators of high caliber.

 Chapter \ref{chapter:origin-privacy} was originally written
 as a technical report in collaboration with Dr. Michael Tschantz
 with guidance from Dr. Anupam Datta.
 I had only dreamed of
 doing scientific research at intersection of law
 and computer science before meetig Anupam,
 who had been doing it successfully for over a decade.
 I aspire to one day have a perspective as correct as his.

 Chapter \ref{chapter:games-and-rules} contains what I feel
 is this dissertation's most original contribution:
 a framework for modeling the consequences of information flow
 in strategic games.
 The seed of this work was nurtured by Dr. John Chuang
 in my first year at the I School, and my collaboration
 with him on Strategic Bayesian Networks remains one of
 my happiest memories of Berkeley.
 Years later John became my advisor.
 I thank him for years of patience, honesty, sensitivity,
 kindness, and clarity  of purpose.

 I am also deeply grateful to Dr. David Wolpert, who discovered
 my work with John on Strategic Bayesian Networks
 and emailed me to tell me I should be looking at
 Multi-Agent Influence Diagrams (MAIDs).
 It was years later that I took his advice seriously;
 MAIDs are a cornerstone of the theoretical contribution
 of this dissertation.

 What can one say about Deirdre Mulligan?
 Words fail me.
 I thank her.

 I acknowledge Dr. David Wagner, for accepting an
 inconvenient responsibility to help a stranger.

 Attendees at the 10th Annual Privacy Law Scholars Conference
 (2017) gave their helpful comments on
 Chapters \ref{chapter:CI-through-CS} and
 \ref{chapter:origin-privacy}.
 Special thanks to Blase Ur and Michael Hintze,
 our discussants.
 I am grateful to Chris Hoofnagle for his warm
 welcome into that community and early advice
 on how to contribute to privacy scholarship.

 I thank Ignacio Cofone, Cathy Dwyer, Yafit Lev-Aretz,
 Amanda Levendowski, John Nay, Julia Powles,
 Ira Rubinstein, Madelyn R. Sanfilippo, Andrew Selbst,
 Yan Shvartzshnaider,
 Katherine Strandberg, Ari Waldman, Bendert Zevenberger,  
 and other members of the Privacy Research Group at
 the NYU School of Law for their helpful comments
 on Chapter \ref{chapter:games-and-rules}.

 I gratefully acknowledge funding support 
 from the U.S. Defense Advanced Research Projects Agency (DARPA)
 under award FA8750-16-2-0287.
 The opinions in this paper are those of the author and do not
 necessarily reflect the opinions of any funding sponsor
 or the United States Government.

 I also gratefully acknowledge John Scott and JC Herz
 for taking me into their company as a data scientist and
 not flinching when completing my doctorate took much,
 much longer than expected.
 Their support as employers have meant much more than the paycheck,
 which I depended on for much of my dissertation work.
 Their understanding of business, technology, and government
 rivals that of any professors who could advise me
 and contributed significantly to the framing and direction
 of this work.
 
 Lastly, I thank Daria, who has made it all worth it.
\end{document}
