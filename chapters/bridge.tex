 \documentclass[../thesis.tex]{subfiles}
 \begin{document}

 Chapter \ref{chapter:CI-through-CS} is an exposition of social theory and a technical literature survey.
 The authorial perspective starts from the standpoint of
 Contextual Integrity, a theory of social norms,
 and from it looks at computer science research.
 The following chapters of the dissertation have
 a different style.
 In contrast, each of Chapters \ref{chapter:origin-privacy} and
 \ref{chapter:games-and-rules} is intended to stand alone as a
 technical article, sharing Appendices
 \ref{appendix:information-theory-theorems}
 and \ref{appendix:maid} between them.
 Though they both draw on Contextual Integrity, the authorial perspective has different assumptions.

 This chapter serves as a disciplinary bridge between these perspectives.
 As a warning to the reader, this bridge is not meant to sustain heavy intellectual loads, so to speak.
 It is a rope bridge that may be traversed carefully by those interested in precarious adventure and panoramic view.
 Its purpose is to convince the reader that the transition
 into a technical discussion of information flow as defined
 in Section \ref{sec:causality} is not an abandonment of
 the theory and mechanisms of social norms, but rather is
 motivated by the understanding of them developed in
 Chapter \ref{chapter:CI-through-CS}.
 In the terms of Chapter \ref{chapter:introduction}, it
 is a bridge between the regulatory modality of social norms
 and the formal construction of situated information flow
 (see Figure 1.1)

\section{Expanding contextual integrity}

Chapter \ref{chapter:CI-through-CS} concludes in Section \ref{CI5} with both calls to action for computer scientists interested in open problems in privacy by design and pointers to theoretical gaps in contextual integrity.
 It calls for ``Contextual Integrity theorists'' to fill these gaps, inventing a category of researcher that had not existed before the article's publication.
 Our intention as authors was to frame Contextual Integrity's blossoming from a bounded theory into an expanding field of inquiry.
 
 We discovered that contextual integrity should be expanded:

 \begin{enumerate}
 \item (Section \ref{CI5.1.1}) ...to account for
   social and technologal platforms that span multiple
   social spheres, perhaps by introducing an ``operator'' context.
 \item (Section \ref{CI5.2.1}) ...to account for more of the meanings of `context',
   that range from abstract social spheres to concrete
   sociotechnical situations.
 \item (Section \ref{CI5.3.1}) ...for clarity on how
   social norms form to reflect ends, purposes, and values in
   society, and the relationship between these norms and the law.
 \item (Section \ref{CI5.4.1}) ...to address the challenging cases where multiple social contexts collide or clash.
 \end{enumerate}

 There is a reflexive, meta-theoretical irony to these gaps.
 Contextual integrity is an account of social norms that
 depends on the idea of a society divided into social spheres.
 These spheres are characterized by social roles,
 information attributes, and contextual purposes that robustly
 coevolve (see Section \ref{CI2.1}).
 Norms of information flow are defined \textit{within} a context
 in terms of its internal ontology.
 The is meant to reflect why and how privacy is socially meaningful:
 it is necessary for the maintenance of these meaningful social structures.

 The gaps in the theory of contextual integrity reflect changes in society's actual situation: technical and social platforms now literally fill the gaps between previously distinct social spaces.
 This makes the ways these platforms conform to or violate privacy
 difficult to conceptualize.
 If it is true that social norms are indexed to particular social
 spheres, the fact that technical infrastructure traverses social spheres does not make the infrastructure \textit{against} social norms so much as \textit{beyond} them, and consequently beyond society's
 capacity to regulate it through norms alone.
 Contextualized normative expectations are specifically vulnerable
 to infrastructure, and that is why we look
 to other modalities (technical design, the law, and the market)
 to protect society's interests.

 \section{Information in and out of context}

 The remainder of this dissertation grapples with the question of how technical and social platforms can be designed and regulated given that many of our socially comfortable expectations of information flow are ontologically mismatched to them.
 At the heart of this question is a deep theoretical question:
 what is information flow, and why is it valuable?
 What does information \textit{mean}?

 It is surprising that such a foundational question has
 not yet met with a scientific, transdisciplinary answer.
 The meaning of data is of great pragmatic concern
 to science and industry.
 Through data mining techniques and other innovations
 \cite{malin2001re} \cite{narayanan2006break},
 ``data's meaning has become a moving target''
 \cite{horvitz2015data} because the inferences data enable
 depend on its sources and what other data it is combined with.
 These techniques are driving business and technical innovation
 and legal regulation.
 Scholarship has perhaps not yet caught up.

 Disciplinary fracturing is part of the problem.
 A consensus definition of semantic information has eluded
 philosophers \citep{sep-information-semantic}.
 Linguistic analysis of the modern use of the word ``information''
 has concluded that it is a confused creole of distinct and
 incompatible meanings \citep{nunberg1996farewell}.
 Library and Information Sciences (LIS) has fruitfully analyzed the
 term and discovered that information can be both a process
 and a thing \cite{buckland1991information}. In LIS,
 \citet{brier2008cybersemiotics} provides a comprehensive account
 of ``cybersemiotics'' that traces the relationship between
 hierarchical layers of semiotics ranging from
 the basic information theoretic
 sense developed by \citet{shannon1948mathematical} to
 social and linguistic meaning based on the social theory of
 \citet{luhmann1995social}.
 \citet{brier2008cybersemiotics} argues that the latter, being
 a property of open systems, will always be to some extent
 indefinite.
 However true, this theory of social meaning
 is unsatisfying for those engaged in
 the practices of privacy engineering, public policy, or
 business.
 Surely something more actionable can be discovered about the
 contours and logic of social meaning.
 Meanwhile, in the natural sciences scholars in physics
 \citep{wolpert2008physical} and biology \citep{deacon2015steps}
 have expressed the aspiration to ground an understanding of
 inference and signification in nature in mathematical information
 and computational theory.
 
 The approach taken in this dissertation is to take inspiration
 from contextual integrity and philosophy of information
 in defining information flow, but
 to posit an analytically tractable, mathematically specific
 definition that is grounded in constructs that are well known
 in computer science, statistics, and social scientific methodology.
 I will refer to this definition as ``situated information flow''.
 Specifically, this account of information flow, which is introduced
 in Section \ref{sec:causality},
 builds on \citet{dretske1981knowledge}, who argued that an
 essential characteristic of information that
 it enables inference due to regularities in the environment
 that produces it, which Dretske calls ``nomic associations''.

 Contextual integrity is enriched by a Dretskian view of
 information flow because it helps explain how the practices
 that maintain the integrity of a social sphere and the
 meaning of information flows dynamically reinforce each
 other.
 If the meaning of information depends on regularities in
 the system through which it flows, then the fact that
 actors are exchanging information to fill well-understood
 roles for well-understood purpose \textit{is} the reason why
 the information they exchange has the meaning that it does.
 As a corollary, \textit{when} social practices change, for
 example due to a new kind of sociotechnical intermediary,
 the possible inferences from information change, resulting
 in new and sometimes unexpected meanings.
 The meaning of information flows and the sociotechnical
 practices around them are mutually constitutive.

 To make this more precise, it is necessary to define both
 the regularity of these practices and how they vary.
 As has been known since \citet{shannon1948mathematical},
 information only flows when a signal has many different
 potential values. It is, to paraphrase \citet{bateson1972steps}
 the difference that makes the difference.
 The mathematics of probability and statistics, which provide
 formal tools for understanding the relationships between variables
 whose values are uncertain, are intimately connected to the
 mathematics of information for precisely this reason.

 \citet{pearl2009causality} provides a robust and widely used
 formal account of structural flows of probabilistic influence
 through causal relationships.
 The position of this dissertation is that Pearlian causation
 and Bayesian networks can provide a useful and tractable
 formalism for understanding the meaning and value of information
 flows.
 The advantage of this formalism is that it can model the
 relationships between both technical components and social
 practices in an apples-to-apples way.
 This is illustrated first in Chapter \ref{chapter:origin-privacy}
 by applying the framework to computer security in embedded systems
 and then in Chapter \ref{chapter:games-and-rules} by modeling
 information economics.
 In order to model how social practices, and in particular strategic
 practices, change the meaning of
 information flows, I draw on \citet{koller2003multi} who expand
 Bayesian networks into a game theoretic formalism: 
 Multi-Agent Influence Diagrams (MAIDs).
 I extend MAIDs into \textbf{data games}, a formal method
 for mechanism design that elucidates the value of data.

 The aspiration of this work is to develop a scientific definition
 of information flow that is useful for understanding the
 interaction of the different modalities of cyberspace regulation
 (social norms, the market, the law, and technology).
 In order to achieve this, it's necessary to develop the definition
 in a way that is both precise and general.
 Mathematical formulations accomplish this precision and generality,
 and mathematical analysis of information flows and MAIDs are
 provided in Appendices \ref{appendix:information-theory-theorems}
 and \ref{appendix:maid}.
 These mathematics, which are objectively and proveably true,
 are intended to transcend any particular
 narrowly defined scholarly discipline.
 This aim of this work is a contribution to computational social
 science \citep{benthall2016philosophy} under conditions of
 disciplinary collapse \citep{benthall2015designing}.

 Shortcomings of this model and room for development
 are discussed in the Conclusion of this
 dissertation, Chapter \ref{chapter:conclusion}.
 
 \end{document}
