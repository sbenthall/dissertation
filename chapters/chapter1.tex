\documentclass[../thesis.tex]{subfiles}


\makeatletter
\newcommand\arraybslash{\let\\\@arraycr}
\makeatother
% List styles
\newcommand\liststyleWWNumi{%
\renewcommand\labelitemi{${\bullet}$}
\renewcommand\labelitemii{${\circ}$}
\renewcommand\labelitemiii{${\blacksquare}$}
\renewcommand\labelitemiv{${\bullet}$}
}
\newcommand\liststyleWWNumiii{%
\renewcommand\theenumi{\arabic{enumi}}
\renewcommand\theenumii{\alph{enumii}}
\renewcommand\theenumiii{\roman{enumiii}}
\renewcommand\theenumiv{\arabic{enumiv}}
\renewcommand\labelenumi{\theenumi.}
\renewcommand\labelenumii{\theenumii.}
\renewcommand\labelenumiii{\theenumiii.}
\renewcommand\labelenumiv{\theenumiv.}
}
\newcommand\liststyleWWNumii{%
\renewcommand\labelitemi{${\bullet}$}
\renewcommand\labelitemii{${\circ}$}
\renewcommand\labelitemiii{${\blacksquare}$}
\renewcommand\labelitemiv{${\bullet}$}
}
\newcommand\liststyleWWNumiv{%
\renewcommand\theenumi{\arabic{enumi}}
\renewcommand\theenumii{\alph{enumii}}
\renewcommand\theenumiii{\roman{enumiii}}
\renewcommand\theenumiv{\arabic{enumiv}}
\renewcommand\labelenumi{\theenumi.}
\renewcommand\labelenumii{\theenumii.}
\renewcommand\labelenumiii{\theenumiii.}
\renewcommand\labelenumiv{\theenumiv.}
}

% Page layout (geometry)
%\setlength\voffset{-1in}
%\setlength\hoffset{-1in}
%\setlength\topmargin{0in}
%\setlength\oddsidemargin{1in}
%\setlength\textheight{7.5795994in}
%\setlength\textwidth{6.5in}
%\setlength\footskip{0.9602in}
%\setlength\headheight{1in}
%\setlength\headsep{0.9602in}


% Footnote rule
\setlength{\skip\footins}{0.0469in}
\renewcommand\footnoterule{\vspace*{-0.0071in}\setlength\leftskip{0pt}\setlength\rightskip{0pt plus 1fil}\noindent\textcolor{black}{\rule{0.0\columnwidth}{0.0071in}}\vspace*{0.0398in}}


% Pages styles
%\makeatletter
%\newcommand\ps@Standard{
%  \renewcommand\@oddhead{}
%  \renewcommand\@evenhead{\@oddhead}
%  \renewcommand\@oddfoot{\thepage{}}
%  \renewcommand\@evenfoot{\@oddfoot}
%  \renewcommand\thepage{\arabic{page}}
%}

%\makeatother
%\pagestyle{Standard}
%\setlength\tabcolsep{1mm}
%\renewcommand\arraystretch{1.3}


% footnotes configuration
\makeatletter
\renewcommand\thefootnote{\arabic{footnote}}


\begin{document}

 \paragraph{Abstract:}
 The theory of Privacy as Contextual Integrity (CI) defines
 privacy as appropriate information flow according to norms
 specific to social contexts or spheres.
 CI has had uptake in different subfields of
 computer science research. Computer scientists using CI
 have innovated as they have implemented the theory and
 blended it with other traditions, such as context-aware
 computing. This survey examines
 computer science literature using contextual integrity and
 discovers:  (1) the way CI is used depends on the technical
 architecture of the system being designed, (2) `context' is
 interpreted variously in this literature, only sometimes
 consistently with CI, (3) computer scientists do not engage
 in the normative aspects of CI, instead drawing from their
 own disciplines to motivate their work, and (4) this work
 reveals many areas where CI can sharpen or expand to be
 more actionable to computer scientists. We identify many
 theoretical gaps in CI exposed by this research and invite
 computer scientists to do more work exploring the horizons
 of CI.\footnote{This chapter was originally published as:

 Sebastian Benthall, Seda G{\"u}rses and Helen Nissenbaum (2017), "Contextual Integrity through the Lens of Computer Science", Foundations and Trends\textsuperscript{\textregistered} in Privacy and Security: Vol. 2: No. 1, pp 1-69.

 I am grateful to my collaborators, Dr. Seda G{\"u}rses and Dr. Helen Nissenbaum, for permission to include our joint work here.}
 

\section{Introduction}
\label{CI1}

Privacy is both an elusive moral concept and an essential requirement
for the design of information systems. The theory of contextual
integrity (CI) is a philosophical framework that unifies multiple
concepts of privacy -- as confidentiality, control, and social
practice \citep{gurses2013two} -- and has potential as a
systematic approach to privacy by design \citep{nissenbaum2009privacy}. Indeed,
over the last decade, computer scientists in a variety of subfields
such as security, HCI, and artificial intelligence have approached the
challenge of technical privacy design by applying CI.

This is a structured survey and review of this body of work. This survey
has threefold aims: 1) to characterize the different ways various
efforts have interpreted and applied CI; 2) to identify gaps in both
contextual integrity and its technical projection that this body of
work reveals; 3) perhaps most significant, it aims to distill insights
from these applications in order to facilitate future applications of
contextual integrity in privacy research and design. We call this,
``making CI more actionable for computer science and
computer scientists.''

Over the last 20 years, privacy by design \citep{cavoukian2009whole,
federalprotecting, regulation2016regulation} and privacy engineering
\citep{gurses2016privacy} have become research topics that span
multiple sub-disciplines in computer science
\citep{rubinstein2010privacy, danezis2015privacy}.
Prior work has shown that computer scientists often stick
to a single definition of privacy, for example, confidentiality,
secrecy, or control over personal information. Although reducing
privacy to a narrow definition has generated interesting work, it has
limitations in addressing the complexities of privacy as an ethical
value. In the wild narrow definitions offer analytic clarity, yet they
may stray too far from a meaningful conception of privacy, that is, a
conception that people actually care about. 

The theory of contextual integrity (CI) was offered as a rigorous
philosophical account of privacy that reflected its natural meaning
while also explaining its moral force. Generally CI characterizes
privacy as appropriate information flow, and appropriate flow
characterized in terms of three parameters: actors (subject, sender,
recipient), information type, and transmission principles. This
definition immediately sets it apart from definitions in terms of
subject control or stoppage of flow. Besides allowing for a more
expressive framing of privacy threats and solutions than other
approaches, the additional factors allow for greater specificity --
hence less ambiguity -- in prescribing and prohibiting certain flows.
Because CI allows formal representation of flow constraints, it may
serve to bridge privacy needs experienced by humans, \textit{in situ},
with
privacy mechanisms in digital systems. Although CI's
account of privacy's ethical importance plays a lesser
role in the work we have surveyed it remains important as a normative
justification for Privacy by Design (PbD) initiatives grounded in
privacy as contextual integrity.

With this survey we aim for more than a description of leading
scientific applications of CI; in addition, we seek an exchange of
ideas. In assessing how these applications have engaged with CI we
ascertain, in one direction of exchange, how true to the letter they
have been and how the framework might have been better or more fully
reflected in the work. Equally, in the other direction, we assess how
these frontrunners may materially inform future developments of CI
itself. Such insights are crucial to enhancing the capacity of CI both
to challenge and inspire scientific work and technical design, thus
making CI more actionable for computer scientists. We conclude the
survey by providing prescriptive guidance going forward.

On the one hand, our findings reveal that, for the most part, computer
scientists engaged in technical design do not take up contextual
integrity in its full theoretical scope. They often give specificity
and depth to some of its concepts while bracketing others, rarely
addressing normative dimensions of CI explicitly. Another common
departure from CI is how researchers interpret context, which often
maps onto their respective disciplinary assumptions, strategies, and
literatures. In reviewing the literature, it was not our aim to declare
any of these approaches ``wrong'' or
``misguided''. Instead, our aim was
to record our findings, identify different perspectives, and note
discrepancies as opportunities to learn from, revise, expand, and
improve CI, to guide future research practice, and, in our terms, to
make CI more ``actionable.''

On the other hand, the forays into implementing systems using contextual
integrity led to significant innovations and improvisation that we
believe can inform contextual integrity theory.
For example, the papers
we have surveyed have elaborated on different types of contexts, most
prominently those that arise as a consequence of interactions (with
people and machines)
or those that come to bear as changes occur in the
environmental conditions surrounding users. In doing so, computer
scientists tease out different socio-technical situations that may
impact how informational norms play out in a social sphere.

Relatedly, the desire of computer scientists to design systems that
observe and adapt to changes throughout time is common. This is
reflected in the use of technical mechanisms that capture changes to
contexts, social norms and environments and that respond to the
evolving conditions. Finally, papers we surveyed often position users
as central actors, highlighting their role and agency in engaging and
transforming informational norms in a context and throughout time.
These concrete instances of socio-technical contexts, adaptivity and
user agency shed light on issues that, with some elaboration, could
enhance the analytical power of CI for privacy by design.

The authors have also taken up known challenges to CI, as in the case of
papers that propose solutions to applying the framework when
information flows from one social situation to the other or when
multiple contexts are co-located. The question of multiple contexts is
acute especially in technologies that act as infrastructures, e.g.,
platforms that host multiple applications or that host actors and
actions from multiple social contexts. The solution space proposed by
the authors include mechanisms to negotiate information flows across
contexts or agents introduced to reason about multiple contexts in a
single application.

Overall, when authors delegate responsibility of governing CI to
technical elements that act semi-autonomously in an adaptive
environment, they also raise novel questions for CI.
It is not uncommon
that researchers design agents that reason about contexts, auditing
mechanisms that ensure informational norms are not violated, or apps
that take active part in negotiating permissions. Can technical
mechanisms be seen as actors in CI? If so, are they acting in the same
context as they are ``serving'' or
are they in a different context? These are some of the hard questions
revealed by the works we studied, and that require input when
considering future applications of CI in technical systems.

Just as important to our analysis is what the authors have
not attended
to in applying the CI framework. We reference here CI's
account of privacy's ethical legitimacy, which
identifies stakeholder interests, societal values,
and contextual ends,
purposes, and values as the basis for such legitimacy. Although
omission of this aspect of CI might not be problematic for the
technical accounts of privacy given by computer scientists, it
nevertheless warrants explanation and justification; we discuss these
issues at the end of the paper. Finally, we found no reference to
information theoretic advances in privacy technologies, e.g.,
differential privacy or privacy preserving machine learning. Given the
growing role that machine learning and artificial intelligence is
playing in information systems, we believe there is a great potential
in exploring how CI may be applied in systems that have the ability to
infer and reason using data.

Finding that these papers both narrow and expand the CI framework, our
review concludes with issues that will be important to address if CI is
going to be useful to a wider spectrum of computer scientists. These
include attending to questions such as: how to be more technically
precise about the nature of contexts in order to relate the definition
given in CI with more concrete notions used by computer scientists; how
to advance normative concepts in CI -- i.e. ends, purposes, and
values -- by taking advantage of well developed methods in the
scientific study of privacy, including user studies, models, threat
models and threat agents; how to use CI in systems where data not only
flows, but also persists in a single place; and, how to apply
contextual integrity to systems that function as infrastructure across
multiple contexts. 

\section{Privacy and context in computing}
\label{CI2}

In this section we provide theoretical
background in privacy and context in computer science that situates our
findings. Section \ref{CI2.1} details contextual integrity as a philosophical
theory that aspires to be robust to changes wrought by technology,
rooted in legal, ethical, and social theory. Our study is primarily of
how this framework has been used by computer scientists. As an
inductive result we discovered that this work often drew from different
conceptions of context as relevant to privacy that came from different
subdisciplines in CS. In particular, we found computer scientists
working at the intersection of CI, introduced in computer science
literature with \citet{barth06sp}, and in the tradition of ubiquitous
computing that has given a central place to context and its
implications for privacy design \citep{dey2001conceptual, dourish2004we}.
We detail the latter in Section \ref{CI2.2}.
We note in Section \ref{CI2.2.2} how the
connection between context and privacy has become recognized by
policy-makers \citep{house2012consumer, wef2012rethinking}. We
speculate that this policy recognition was responsible for an uptick in
the interest of computer scientists in context and
privacy. The resulting creative synthesis
of multiple traditions offers an opportunity for realizing new
theoretical insights and opening new research problems.

\subsection{Contextual integrity}
\label{CI2.1}

The practice of privacy may be as old as social life itself but the
contemporary need for a concept of privacy rich enough to drive policy
and precise enough to shape architecture follows in the wake of
advances in technologies that have disrupted how we create, collect,
communicate, disseminate, interpret, process, and utilize data. (It is
worth noting, however, that rarely, if ever, is it raw technology that
stirs agitation; instead agitation is a response to technology embedded
within particular social practices and particular political
ecosystems.)
In the US the contemporary need to sharpen the concept
and strengthening protection is often dated back to 1895 with Warren
and Brandeis's historic call to define a legal right to
privacy in the wake of new photographic and printing technologies. In
1973, the landmark Report to the Secretary of Housing, Education, and
Welfare issued Principles of Fair Information Practices (FIPs)
following the rise of massive computerized databases. 

In the 1990s and 2000s, such systems extended to video,
audio, and online surveillance, RFID, and biometrics systems.
Subsequently, public attention has turned to hyperbole over
``big data'' -- database technologies, computational
power, and scientific advances in information and data
processing.
Dramatically amplifying the privacy impacts of these
technologies are transformations in the software engineering
industry -- with the
shift from shrink-wrap software to services--
spawning an agile and
ever more powerful information industry.
The resulting technologies
like social media, user generated content sites, and the
rise of data
brokers who bridged this new-fangled sector with traditional
industries, all contribute to a data landscape filled with
privacy perils. 

Approaches to privacy that depended on neat divisions of
spaces or data into public and private have been severely
challenged by these developments.
Long entrenched definitions of privacy rights as rights
to control information or rights to secrecy, that is,
to block access, were overly simplistic, either too easily
challenged by those with competing interests or over-claiming on the part of data subjects. An account that captured the
complex contingencies of legitimate privacy
claims was needed -- one that benefitted from conceptual
building blocks of existing theories but offered a greater
expressive agility,
to resist incursions while allowing the positive potential
of novel socio-technical systems to be realized.
Contextual integrity intends to
provide such an account. For one, it addresses gaps in prior entrenched
conceptions allowing it to identify privacy threats to which other
accounts were blind (e.g. ``privacy in
public''). It also offers a view on the nature and
sources of disruptive information flows in order to distinguish that
that constitute threats from those that do not. 

The theory of privacy as contextual integrity (CI) introduces three
key concepts into the privacy vocabulary:

\begin{enumerate}

\item \textbf{Contexts.} These refer to social contexts, not formally
constructed but discoverable as natural constituents of social life.
As theorized in sociology, social theory, and and social philosophy,
they have been assigned various labels, including, social domains,
social spheres, fields, or institutions. (Throughout this survey, we
will use the term \textit{sphere} to denote this sense of context.)
Societal recognition of distinct contexts, such as healthcare, family
and home life, politics, religion, commercial marketplace, and
education is clearly evidenced in distinctive structures of law and
regulation. 

For the framework of contextual integrity, contexts are formally
characterized in terms of key elements, which include, paradigmatic
activities, roles (or capacities), practices, and norms.
Distinguishing contexts from one another, are contextual goals, ends,
purposes, and values, around which activities and norms are oriented,
and to which respective contexts are committed.


\item \textbf{Contextual informational (privacy) norms.} Among
contextual norms, these govern information flows and, according to
contextual integrity, are likely to map onto people's
privacy expectations. Informational norms are well-formed only if
they refer to five parameters: sender, recipient, and information
subject, information types (topics, attributes), and transmission
principle. The parameters of actors and attributes range over
contextual ontologies, distinctive to respective social contexts, if
not unique. Thus, in healthcare context, senders, recipients, and
subjects range over agents acting in the capacities, such as,
\ doctor, nurse, patient, surgeon, psychotherapist, etc. and
topics may range over symptoms, diagnoses, and drug
prescriptions. Transmission principles condition the flow of
information from party to party, including those commonly associated
with privacy, such as, \textit{with permission of data subject},
\textit{with notice,} or \textit{in confidence, }in addition to those
less salient, such as, \textit{required by law}, \textit{with a
warrant,} and \textit{entitled by recipient}.

Privacy as contextual integrity is respected when entrenched
informational norms are followed. When these norms are violated (e.g.
by disruptive information flows due to newly functioning technical
systems) there is a prima facie case for asserting that privacy has
been violated. The framework of contextual integrity allows, however,
for the legitimacy of disruptive flows to be defended, as described
below. 


\item  \textbf{Contextual ends, purposes, and values.} These may be
considered the ``essence'' of a
context, without which respective contexts would not be comprehensible.
How would one properly describe a school, say, without indicating its
purpose? These -- let us call them -- teleological factors are also
important in defending the legitimacy of informational norms,
particularly useful when comparing novel information flows against past
expectations, or when no competing alternative is obvious, they are
useful in evaluating the ethical legitimacy of given flows taken alone.

\end{enumerate}

According to the CI framework, privacy norms can be assessed in terms of
how they affect the interests of relevant parties
(``stakeholders'') and how they
impinge on societal values, such as equality, justice, fairness and
political liberties. In addition to these considerations the norms
governing flow can be evaluated in terms of their impacts on the
attainment of contextual ends, purposes, and values -- either promoting
or confounding them. For example, informational norms enabling (and
enforcing) a secret ballot protects autonomous voting in elections and,
as such, promotes ends and values of democracy.

This structure ensures that though CI is conservative, in the sense that
it presumes in favor of entrenched norms, it nevertheless has built
into it a set way for systematically evaluating and updating norms.
This is done by examining balance of interests, general ethical and
political values, and contextual values and purposes. It follows that
norms adapt to their environment, crucial for an account of privacy to
remain relevant in the face of advancing technologies of information
and computational technologies. Societal and environmental shifts can
destabilize entrenched privacy norms in many ways, either revealing
that they are no longer optimal in achieving contextual ends and values
or have nothing to say about disturbing information practices. Although
such circumstances constitute challenges for ethicists and social
policy makers, they do not necessarily constitute challenges to CI
itself, which copes with novel or disruptive flows by presenting new
norms for consideration.

Adapting norms to novel or disruptive flows may involve adjusting any of
the parameters. For example, the increasing digital mediation of
transactions, communications, and interactions (including social media)
creates new data recipients, which forces the reconsideration of norms.
The same goes for increasing specialization and fracture of skills and
functions within traditional contexts, healthcare being a prime
example. The one-on-one physician-patient relationship paradigmatic of
the distant past has been replaced by an immensely complex care and
treatment ecosystem, involving specialists, insurance companies,
pathologists, public health officials, wireless pacemaker service
providers, and, with that, the emergence of new informational norms --
in the ideal, to serve contextual ends and values.

A note on terminology: We refer to aspects of CI that deal with
evaluating the \textit{legitimacy} of norms as its normative,
prescriptive, or ethical aspects. These aspects are contrasted with
what we might call its \textit{descriptive} or conceptual aspects,
referring to the the structure of informational norms.


\subsection{Context in computing}
\label{CI2.2}

CI bridges two worlds. In one, it is an account of
privacy as a term that has accrued meaning to describe alarm over
wide-ranging, technology induced practices of surveillance, data
accrual, distribution, and analysis. The alarm is due to disruptive
practices that violate privacy expectations and create or amplify
imbalances in power. CI posits contextual informational norms to model
privacy expectations and explains when such expectations are morally
legitimate and warrant societal protection. In the other, CI offers a
formal structure for expressing rules of information flow
(informational norms) and for building computational models of
people's privacy expectations in well-defined settings
(contexts.) The first is a world inhabited by humanists, social
scientists, lawyers, and regulators; the second is inhabited by
mathematicians, computer scientists, and engineers. Perhaps because it
seeks to map a meaningful conception of privacy onto a conception
that strives for formal rigor contextual integrity has been taken up by
computer scientists interested in privacy design and engineering. 

Although philosophical versions of contextual
integrity appeared in articles, dating back to 1998
\citep{nissenbaum1998protecting}
and, later, in the book, \textit{Privacy in Context} \cite{nissenbaum2009privacy},
it was not represented in
computer science literature until \citet{barth06sp}. This paper,
which we introduce as one of our survey exemplars in
Section \ref{CI3.3.1},
formalized the fragment of CI known as context-specific (or,
contextual) information norms. The authors, which include Nissenbaum,
developed a logical framework for expressing and reasoning about norms
and showed that this framework is adequate for expressing regulations
drawn from U.S. sectoral privacy laws, such as, HIPAA, GLBA, and COPPA.

In fact, contextual integrity is but one source of influence that has
drawn computer scientists to engage with the idea of context as it
relates to privacy. Two others are worth discussing because they have
roused the interest of computer scientists in context and shaped how
they conceive of context with respect to privacy. We therefore find in
computer science loose interpretations of contextual integrity that
consider these other forms of context. They are: the field ubiquitous
computing and the Obama White House Bill of Consumer Privacy Bill of
Rights \cite{house2012consumer} (also the World Economic Forum and FTC
Reports around a similar time \citep{wef2012rethinking, federalprotecting}.
Grasping these influences has been important in advancing our own
ability to analyze the articles we have chosen for this survey.

\subsubsection{Context in ubiquitous computing}
\label{CI2.2.1}

Contextual integrity is not the only research tradition linking context
and privacy in computer science. Many contemporary issues in
human-computer interaction around mobile devices and IoT were
anticipated in earlier waves of research into
``ubiquitous computing''. This
research program envisioned a world in which computation was not
restricted to specialized workstations but, instead, was embedded in
everyday objects and practices, enabling user interaction through
sensors and actuators. Within ubiquitous computing research interest
emerged in developing technologies that were responsive to social and
environmental context, that is to say,
`context-aware' computing.

In their ``anchor article'' on
context-aware computing, \citet{dey2001conceptual}
extensively analyze
definitions of `context' in the
literature of their field and settle on the following for their own
work:

\begin{quote}
\textbf{Context:}
any information that can be used to characterize the situation of
entities (i.e., whether a person, place, or object) that are considered
relevant to the interaction between a user and an application,
including the user and the application themselves. Context is typically
the location, identity, and state of people, groups, and computational
and physical objects. \citep{dey2001conceptual}
\end{quote}

This definition of context, specifically referring to the concrete
\textit{situation} of persons and objects, starkly contrasts with the
notion historically evolved, abstract, and normative social
\textit{spheres} of CI. In our survey of the computer science
literature invoking contextual integrity, we found that several papers
conceive of `context' in ways that have
more in common with context-aware computing than with context as
defined in CI. This has led to interesting synthetic work, in addition
to incipient disjunctures. 

That computer scientists have taken up the tradition of context-aware
computing in their work on privacy as contextual integrity is not
surprising. Early in this field, contributors
\citet{ackerman2001privacy}
anticipated that
context-aware
computing would lead to privacy by design, arguing that technical
systems and legal frameworks would be co-designed. But hints of the
connection between context-aware computing and contextual integrity,
which was not formulated as a framework until later, were present at
least as early as Dey et al.'s anchor article \cite{dey2001conceptual}:

\begin{quote}
As computational
capabilities seep into our environments, there is an ever-growing and
legitimate concern that technologists are not spending enough of their
intellectual cycles on the social implications of their work. There is
hope, however. Context-aware computing holds a promise of providing
mechanisms to support some social concerns. For example, using context
to tag captured information may at first seem like an intrusion to
individual privacy (``I do not want to meet in a room
that records what I say.''). However, that same
context can be used to protect the concerns of individuals by providing
computational support to establish default access control
lists to any
information that has been captured \cite{lau1999privacy},
limiting distribution to those who were in
attendance.
\end{quote}

Most relevant to this
article are the implicit connections these authors draw between the
design of technology responding to a particular situation
(certain people meeting in an office room)
and a general expectation of
privacy. The norm that literal, unfiltered information about what
happens in meetings is available only to those who attended could be
attributed to the abstract social sphere
of office meetings. This prefigures a result of our study, which finds
computer scientists taking up `context'
in ways that reflect both senses of the word, and in so doing
implicitly drawing connections between them.

This early work on context-aware computing reflected the state of the
art in sensors and the kind of sensing they made possible: explicitly
representing context as a kind of fixed container in which people could
act. This was famously critiqued in a paper by \citet{dourish2004we},
connecting how `context' is approached
in ubiquitous computing to broader questions in philosophy of science.
Dourish argued that representing context (e.g. location or time in
which an application is used), as a kind of container for activity
whose boundaries are delineable draws from the \textit{positivist}
tradition in social science that sees context as stable and separable
from the activity taking place within it. Dourish contrasts this
understanding of context with a different one deriving from the
\textit{phenomenological} tradition of social science.
According to it,
context is \textit{occasioned}, ``relevant to
particular settings, particular instances of action, and particular
parties to that action'', not an abstract category
that generalizes over settings, actions, and parties. This kind of
context arises from and is maintained by its activity, sometimes
dynamically adjusting along with the activities themselves. For
example, a private conversation between friendly
colleagues at work can
shift from a formal, professional discussion into an
informal, personal
discussion and back again. These shifts will occur as and through
changes in the conversational activity, such as changes in tones of
voice or comments such as, ``Well, we should really
get back to work; I have to go in twenty minutes.''

Dourish investigates how this conception of context ties into the
sociological mystery of how social order comes into being. There is a
tension between explanations of social order that attribute it to
rules, expectations, and conventions that have a broader reality beyond
particular occasions of interaction (what we might call a
`top-down' ordering), and explanations
that see all social order as arising from interaction itself as an
achievement of the social actors
(`bottom-up' ordering).

While it may be argued that top-down and bottom-up ordering
are always
co-occurring, often one or the other process is emphasized
in scholarly
work. Contextual integrity, in its original articulations
\citep{nissenbaum2004privacy, nissenbaum2009privacy},
tends to emphasize the top-down
pressure of
contextual ends, purposes, and values shaping norms that in turn guide
information flows. In contrast, while Dourish acknowledges the role of
top-down orderings, he highlights the bottom-up processes that make
each context occasioned and dynamic, in the spirit of his
interactional, phenomenological objection to static representations of
context. We find that both ways of thinking about context are prominent
in the literature that we review, even though we have limited this
review only to computer science literature that refers to contextual
integrity. 

\subsubsection{Context in privacy policy}
\label{CI2.2.2}
  
While we were looking specifically for computer science papers that
referenced contextual integrity, it was interesting to find many papers
that took ``privacy in context'' as
an idea (which also happens to be the title of
Nissenbaum's book about contextual integrity
\citep{nissenbaum2009privacy}), but that do not draw from the
framework of contextual integrity.
If the direct origin of ``privacy in
context'' was not contextual integrity, what was it?

Our contention is that interest was prompted by the general uptake of
context and contextual integrity in the formulation of several policy
documents from 2010 and later. For example, the White House Report,
``Consumer data privacy in a networked world: A
framework for protecting privacy and promoting innovation in the global
digital economy`` \citep{house2012consumer}, lists
``Respect for Context'' as one of
its seven principles: ``Consumers have a right to
expect that companies will collect, use, and disclose personal data in
ways that are consistent with the context in which consumers provide
the data.'' Around the same time, a report issued by
the U.S. Federal Trade Commision also invoked context when it
stipulated that data collection by companies should be restricted to
what was appropriate for the ``context of
interaction'' or else they should make
``appropriate disclosures''. In a
similar vein the World Economic Forum's 2012 report,
Rethinking Personal Data invokes the importance of context for policy
governing data in numerous places (e.g. pages 5, 10, 15, 16, 17, 19,
and more.) \citep{wef2012rethinking} 

We have found significant variation in how computer scientists have
interpreted the term ``context'',
often reflecting their disciplinary background and research agendas.
Some follow contextual integrity quite closely. Others cite
`contextual integrity' and
`privacy in context', possibly to
situate the privacy-context connection within a scholarly lineage
without following CI
substantively. Most of these
papers were written after the policy arena acknowledged CI theory,
while in parallel ubicomp researchers had already established a concept
of context. For some computer scientists, context as
\textit{situation}, reminiscent of ubiquitous computing research,
informs their work, as it does some of the work on privacy regulation
(albeit outside the focus of this paper.) Overall, though computer
scientists have, characteristically, explored the relationship between
various forms of context and privacy in a rigorous and pragmatic way,
they have not made the definition of context the subject of explicit
theoretical commitment.

Nevertheless this work in computer science at the boundaries of
contextual integrity makes important contributions to the theory
itself. Inspired broadly by contextual integrity, computer scientists
have explored aspects of the relationship between privacy and context
in detail. Our systematic study of these works has found in the
variations and commonalities within this literature insights that can
inform and inspire further developments in contextual integrity.

\section{The study}
\label{CI3}

The main objective of this study is to characterize the different ways
CI has been interpreted and applied in computer science, reveal its
technical projection, and thereby, capture gaps in CI itself. The long
term objective of this study is to identify ways that CI can be made
more actionable for computer scientists and systems developers. In
order to do so we systematically reviewed literature coming out of
different subfields of computer science explicitly stating the use of
contextual integrity in their problem or solution definition. We made
use of techniques proposed by \citet{kitchenham2007guidelines} to make our
study as comprehensive and transparent as possible. 

In their projects invoking CI, computer scientists have taken on the
hard task of translating an elaborate philosophical framework into
computer science research practice -- in which different theoretical
and methodological traditions apply.
This renders the translation of CI
into technical contexts a non-trivial task. For these reasons alone an
assessment of current uses of the theory in CS is valuable for
understanding how well the theory translates, what new questions arise
when applied in a technical context, and what obstacles become evident.
Through this survey we evaluate its uptake in computer science and
begin to sharpen the theory to make it more actionable for researchers
who want to use it in the future. 

\subsection{Research Questions}
\label{CI3.1}

Driven by the motivations listed above, we
decided to focus on four research questions as we take stock of the
use of CI in computer science research and assess it:

\subsubsection{RQ1. For what kind of problems and solutions do computer scientists use CI?}

As an initial question for our inquiry, we wanted to know if there were
any particularly notable categories of problems being addressed by
computer scientists using contextual integrity. Computer science is a
broad field; researchers may have found contextual integrity useful for
solving particular kinds of problems, focus on certain domains, or be
more likely to invoke CI in certain subfields of computer science.

\subsubsection{RQ2. How have the authors dealt with the conceptual aspects of CI?}

Contextual integrity is partly a conceptual theory that is predictive of
social concerns about privacy that originate and manifest themselves
especially with technological change. The theory posits \textit{social
contexts} as evolved abstract spheres of activity characterized by
\textit{ends, purposes, and values}. Social contexts have
\textit{information norms}, parameterized by \textit{actors (senders,
recipients, and subjects), information types, }and\textit{ transmission
principles}. Contextual integrity identifies privacy as appropriate
information flow; such flow would be characterized by contextual
informational norms.

We wanted to know to what extent the computer science researchers using
contextual integrity used this conceptualization of privacy. Do the
researchers define context in the way contextual integrity does, or in
other ways? And do they define privacy in terms of appropriate
information flow according to norms?

\subsubsection{RQ3. How have the authors dealt with the normative aspects of CI?}

Contextual integrity is a normative
framework of privacy.
It argues that privacy is an important value
because appropriate information flow promotes the data
subject's interests in balance with those of others as well
as societal and ethical values, and maintains
the ability of social contexts to fulfill their purposes.

We wanted to know if computer scientists
using contextual integrity take up this normative aspect of the theory.
If not, from where do they perceive the normative clout of privacy
coming? How do they evaluate whether privacy is addressed effectively
through their proposed mechanisms or solutions?

\subsubsection{RQ4. Do the researchers expand on CI?}

In developing technical systems computer scientists have to make a
number of substantive and specific design decisions. This is also the
point at which the rubber meets the road: how does a researcher
translate a philosophical theory into a formulation useful for
technical design? In executing this translation computer scientists are
likely to attend to concrete questions that CI may not provide explicit
guidance for. In these moments, researchers are likely to identify gaps
in CI and propose techniques to make up for these gaps. What are the
gaps that researchers identify, how do they expand on these, and how do
they stretch the theory explicitly or implicitly?

\subsection{Study Methodology}
\label{CI3.2}

In compiling and revising the relevant papers, we followed empirical
research methodologies recommended for use in software engineering
studies \citep{kitchenham2007guidelines}.
In order to answer our research
questions, we conducted the following four steps: 

\liststyleWWNumi
\begin{itemize}
\item Based on our research questions (Section \ref{CI3.1}), we iteratively
developed an initial template of analytic questions using a selection
of CI articles. 
\item We searched in online repositories for papers using CI as its
reference theory. To ensure we have a reasonable collection, we
searched digital libraries (Google Scholar, IEEE Xplore, ACM DL) for
papers that appeared in CS venues that had CI in their title or main
body. To cast a wider net, we included the key terms
``contextual integrity'' and
``context AND privacy''. For those
papers that explicitly invoked CI, we combed through later publications
that cited them to see whether the use of CI propagated. We carefully
evaluated the inclusion of papers that only reference CI without making
further use of the framework. In the process, we found a number of
papers that refer to context and contextual norms that do not refer to
Nissenbaum's work and removed these from the study.
Evaluating whether and how CI may have proven useful in these papers is
out of the scope of the current work. Some papers claimed they used CI
and integrated other conceptions of
``context'' in CS, we kept these
papers in our study. We initially categorized papers with respect to
the subfields of computer science from which they originated. The
represented fields of research included security engineering (including
privacy engineering and access control); artificial intelligence
(including papers on multi-agent systems, machine learning, semantic
web, social network analysis and community detection); systems
(distributed systems, pervasive and mobile computing); HCI (usable
security and privacy, ubiquitous computing); and software engineering
(requirements engineering and business process design). 
\item Once we had completed our search, we tested the completeness and
consistency of the template based on close reading of additional
articles. Once the template was stable (see Appendix \ref{appendix:template}), we (the
authors) independently read each paper and answered each question of
the analytic template for it. We did a comparative analysis of the
answers in order to distill those aspects of the papers that answered
our research questions. At this stage, we also concluded a quality
assessment of each paper with respect to its contributions to computer
science and removed those that failed our assessment. We documented all
of our analysis and discussions in an online repository.
\item We used the output of the templates to complete a thematic
analysis of each paper. We consolidated what we had discovered into
major categories of themes, one for each research question. Our work
indicated the most productive way to interpret these questions. For
RQ1, we found the most significant way we could characterize the
variety of problems addressed in the literature was by looking at the
kind of technological architecture researchers were designing. For RQ2,
we focused on how researchers characterized
``context'' in their work. We split
this concept of `context' down into many
finer-grained variables in order to show the variability between
papers. For RQ3, we looked specifically for sources of normativity used
by each paper and coded them accordingly. For RQ4, we analyzed the ways
in which the papers expanded on contextual integrity. Our analysis did
not reveal that the initial categorization of papers according to
subfields in CS revealed further insights for our study.
\end{itemize}

\bigskip

In the remainder of this Section, we provide detailed accounts of select
papers as illustrations of how we thematically analyzed each paper in
accordance with the steps described above.

\subsection{Three exemplars of analysis}
\label{CI3.3}

In order to provide the reader with a demonstration of how we got to the
different themes in our results, we pulled out three of the papers to
serve as exemplars. We selected these three papers as they deeply
engage CI; they stem from different subfields in CS with varying
methods and techniques; and, they allow us to demonstrate the rather
different ways in which the authors have elaborated on CI. The curious
reader is encouraged to read these full papers which are rich in ideas
and thoughtful in their use of CI. All other papers are are analyzed
according to respective categories and themes, extracted through the
template that had guided our reading of them.

\subsubsection{Privacy and Contextual Integrity: Framework and Applications
  (Barth, Datta, Mitchell, and Nissenbaum)}
\label{CI3.3.1}

The first published computer science paper to reference contextual
integrity was coauthored by Helen Nissenbaum and therefore can be said
to be an authoritative expression of the theory. It is not, strictly
speaking, a paper about the design of a technological artifact. Rather,
it is an articulation of a subset of the principles and parameters of
contextual integrity in a formal logic (something further discussed in
Section \ref{CI4.1} under RQ1). Formalization is a prerequisite to
computational implementation, and so this paper demonstrated the
potential of contextual integrity as a guide to the design of
computational systems. For the purposes of our study it is just as
notable what it did not formalize into logic, as this has left open
many challenges to computer scientists seeking to use contextual
integrity.

After grounding the work in an exposition on contextual integrity
theory, the first major contribution of the paper is a careful
translation of principles of contextual integrity into formal logic.
The particular flavor is Linear Temporal Logic (LTL), a type of logic
which is capable of expressing formulae of relationships of variables
arranged in time. This translation refines the ontology of contextual
integrity by making explicit that information flows have a temporal
order. This allows the authors to define specific transmission
principles that condition appropriate flow on previous flows (further
discussion under RQ4 in Section \ref{CI4.4}). The logical specification allows
a particular history or trace of information flows to be audited for
appropriateness according to formal rules.

One of the benefits of having to make the logic of contextual
integrity
explicit is that it brings to light aspects of the theory
that are easy
to take for granted but which have far-reaching implications.
The paper
explicitly models both the knowledge available to each actor at
different points in time as well as the ways that different attributes
are related to each other via inference. This paper therefore provides
an epistemic model that is only implicit in other accounts of CI.
Having provided a formal language for expressing policies in the style
of CI's context-specific information norms, the authors
go on to prove a number of theorems about the computational complexity
of auditing traces based on these policies, testing for the possibility
of complying with the policy, and comparing policies.

The authors do not tie their formalization back to the origin of norms
through the evolution of social sphere and its ends, purposes, and
values. Rather, after formalizing the aspects of contextual integrity
that they are able to, they validate their work by showing that it is
expressive of United States sectoral privacy laws: HIPAA, GLBA, and
COPPA (see \citet{datta2011understanding} for further work along these lines).
They also argue that the expressivity of their formalization compares
favorably with other proposed access control policy languages such as
XACML, ECAP, and P3P.

This paper is particularly notable as the first published computer
science paper concerning contextual integrity. Explicitly only a
formalization of \textit{part} of CI, \citet{barth06sp} provide a
way of expressing norms as policies that can be used in computational
tests for compliance. This sets a precedent for computer science papers
using contextual integrity to consider
`context' in a structured, abstracted,
and normative way (see RQ2 in Section \ref{CI4.2}). It sets aside parts of
contextual integrity that account for how norms form through adaptive
social processes. By focusing on regulatory compliance, it brackets the
social source of privacy norms (RQ3 in Section \ref{CI4.3}). If there is
something lost in this usage of contextual integrity in computer
science, it may be recovered through other uses and understandings of
social context that have influenced technical research.

\subsubsection{Android Permissions Remystified: A field Study on Contextual
  Integrity (Wijesekera et al.)}
\label{CI3.3.2}

The potential role that permissions in
mobile platforms can play in providing users with control and
transparency over how their information flows to apps and mobile
platforms has recently attracted much research. For a long time,
Android and iOS platforms asked users for permissions at install time.
Recently they have extended the framework to also make it possible to
prompt users for permissions during runtime. Prior research has shown
that few people read the Android install-time permissions and even
fewer comprehend the complexity that the permissions are loaded with
-- for example, the permission to access Internet may be bundled with
the permission to load ads. Prompting users too frequently causes
``habituation''. Limiting prompts,
however, raises questions about which ones to select in order to
effectively protect user privacy. \citet{wijesekera2015android} leverage contextual
integrity to approach the usable security and privacy problems that
arise when interfacing a permission request model to the users (a user
interface and technical platform problem as discussed in RQ1 in Section
\ref{CI4.1}).

In their study, the authors examine how applications are currently
accessing user data and assess whether or not it corresponds to
users' expectations. To do so, they instrumented
Android phones to log whenever an application accesses a
permission-protected resource and distribute these to 36 participants
who use the phones for a week. For each permission request, they keep a
log of the app name and permission, and further
``contextual factors'' which include
whether the requesting application was visible to the user (running
with or without user interaction, notifications, in the foreground or
background); screen status (whether the screen was on or off);
connectivity (the wifi connection state); location (the
user's last known coordinates); the UI elements that
were exposed to the user during the request; history of interactions
with applications; and, the path to the specific content that was
requested\footnote{For example, if Spotify requests a wi-fi scan while
the user is playing Solitaire, then visibility is set to false, the
history shows that prior to the Spotify prompt, the user had viewed
Solitaire, the screen status was on etc.}. After the week, study
subjects participated in an exit survey where they were shown a sample
of screenshots to inquire about their expectations relating to
requested permissions. The authors use the outcomes of the study to
start specifying a classifier to automatically determine whether or not
to prompt the user based on contextual factors.

During the one week study, the 36 phones logged 27 million application
requests to protected resources, translating to 100,000 requests per
user/day. The authors found that 75.10\% of the permissions were
requested by apps that are invisible to the user (most of these were
requested when the screen was turned off, which is 95\% of most phones
lifetime)\footnote{ The applications making the most permission
requests are Facebook, Google Location Reporting and Facebook
Messenger.}. Using the data they collected, they analyze which
permissions are requested and the different ways in which certain
information can be accessed (e.g., there are multiple ways to access
location). They argue that due to invisibility, frequency and exposure,
what the authors' have dubbed as users'
contextual integrity -- meaning what they expect from apps and their
permission behavior -- is violated.\footnote{
The user study provides greater insights
as to when users feel that their expectations are not met which is
worth reading but too detailed for the study at hand. 

The result of the exit survey shows that
users' decision to block a permission request was based
on a variety of contextual factors. When asked why they would want to
block certain permissions, 53\% of survey subjects stated that they
didn't think the permission was necessary for the
functionality of the application. However,
users do not categorically deny permission
requests based solely on the type of resources being accessed by an
app. They also take into consideration how much they trust the
application and whether they are actively using it. Moreover, the
status of the screen and whether the app is in the foreground has an
impact on whether users are more likely to appreciate the permission
type in their decision.

The authors use these insights to develop a
classifier that can infer when the user is likely to deny a permission
request and prompt for explicit run-time permissions. Their classifier
makes use of originating application, permission and visibility for
prompting users as well as personalization factors to meet
users' contextual expectations. They complete the study
of this classifier with a short evaluation of its accuracy.

In their reading of contextual integrity,
the authors abstract away the social contexts of apps (see RQ2). They are
not concerned with the information norms an app may be subject to due
its social context, e.g., is it appropriate for a health app to collect
user location? Rather, they equate privacy violations with occurrences
of the collection of personal information in ways that defy user
expectations in the context of an interaction based on their list of
contextual factors. Starting from this definition, they go on to study
those permissions and contextual factors that are most likely to defy
users' expectations and that may be good candidates and
situations for prompting users at run-time. }If we were we to describe
the study using Dourish's vocabulary, we would say that
the authors study which of these factors users consider to be
``contextual'' to their interaction
with their apps and mobile devices. This sets this study apart from
typical context-aware computing papers that have a more static view of
what counts as context. A more detailed discussion on how contexts are
handled in the different papers can be found in Section
\ref{CI4.2}.

While the authors lean on CI, they do not make explicit use of the
parameters part of the conceptual framework nor invoke its normative
aspects. Implicitly, we can interpret the model that underlies the
study to treat users as senders, apps as recipients, type of data as a
kind of contextual factor. Moreover, we can regard the permission
prompts as implementing transmission principles that make select
information flows conditional on user's approval.
However, by evaluating appropriateness of information flows with
respect to an app, rather than the social context that the app serves,
the study also falls short of understanding user expectations with
respect to information flows that may be initiated by the organization,
be it sharing user data with other companies or users (see RQ3). In
general, relying on users expectation as a normative source leaves out
other potential sources of information norms which may have been very
useful in further pruning those prompts that request permissions for
inappropriate information flows. As a result, the authors clearly
deviate from the normative ambitions of the framework and hold its
conceptual premises only in our interpretation.

Foregrounding apps does reveal interesting
results that go beyond what is typically in the scope of a CI analysis
(see RQ4). First, the authors find that users wanted to accept some
permissions because they were convenient and others they wanted to
reject because they requested access to privacy sensitive information
(e.g., SMS messages) regardless of the social context. Second, users
were more likely to expect and accept requests that occurred in the
foreground of an application than in the background, and they were more
likely to want to block a permission if it was from an app or process
in the background, too frequent or when the phone screen was locked. In
other words, users consider additional factors when it comes to
evaluating the appropriateness of an information flow. This result
stands to inform CI by pointing out the need to acknowledge technical
and operational contexts, which we discuss in Section 4.2.

\subsubsection{Implicit Contextual Integrity in Online Social Networks (Criado
  and Such)}
\label{CI3.3.3}

In this fascinating paper coming from the field of Multi-Agent Systems
in Artificial Intelligence applied in the context of Online Social
Networks (OSNs) the authors theorize and develop an agent that responds
to the problem of ``Implicit Contextual
Integrity''.\footnote{ The authors have written two
papers with the same title. Here we refer to the longer and more
detailed version published in the Information Sciences Journal 325
(2015) 48-69. } The main motivation for the authors is to introduce
mechanisms that address issues related to
``inappropriate exchanges and undesired
disseminations'' that happen due to lack of effective
privacy controls in OSNs (see RQ1). Pointing to numerous studies in
computer science, the authors argue that contextual integrity, a model
upon which effective computational mechanisms can be built, is the
right framework for developing effective controls for OSN users.
However, prior computational models have assumed the existence of
well-defined contexts with predefined roles and explicit norms. These
are not always available in OSNs, as context, roles and associated
informational norms are ``implicit, ever changing and
not a-priori known''.

In order to support users with these implicit norms, roles and contexts,
the authors propose an Information Assistant Agent (IA-agent) that can
infer the context that the user is in and the information norms
belonging to that context. In describing their solution, they first
present an information model for Implicit Contextual Integrity and then
characterize the IA-agent. The agent uses the information model and
further modules to learn implicit contexts, relationships and the
information sharing norms. It uses this information to warn users
before they make potentially inappropriate information exchanges
(within a context) or engage in undesirable dissemination of
information previously exchanged (across contexts). 

Criado and Such leverage a plethora of techniques available to them to
compose a formalization of appropriateness that can be used by the
IA-Agent. First, they assume that each information exchange can be
mapped to a range of finite topics, e.g., that a post about a tennis
match is about sports. The frequency with which certain topics gets
mentioned by members of a context is crucial -- messages pertaining
to topics that are rarely mentioned are considered inappropriate and
vice versa. Some exception is made, however, to infrequently
communicated topics: if reciprocity underlies a given communication
between members of the context, then the information flow is
reconsidered as appropriate. Furthermore, if information on a topic has
been previously disclosed in a given context, then a repeat disclosure
in that context is not seen as inappropriate, and hence is not regarded
as entailing new privacy risks. Appropriateness of a topic may increase
if members of a context start exchanging messages on the subject. It
may also decay, as information flows pertaining to the topic decrease
or disappear. 

A message may flow to people in multiple contexts, in which case it is
assumed to be flowing to the context with most recipients. For example:
if Mary is Alice's friend and workmate, and Alice sends
a message to Mary and three other people from her work context, then it
is assumed to be a message flowing in the work context. The agent also
takes into consideration whether the information in a message is known
in the different contexts shared by the recipients of that message.
Since the IA-agent needs to keep track of frequency, past mentions and
reciprocity, the corresponding design requires keeping track of all
past communications. 

In summary, the total appropriateness of a given information flow is
based on three different metrics: appropriateness of topic to
individuals, appropriateness of an information flow in a context, and
appropriateness of a message across contexts. Four modules allow the
IA-Agent to complete its tasks: 

\liststyleWWNumiii
\begin{enumerate}
\item Community finding algorithm: identifies new contexts made up of
densely connected members.
\item Passing time function: updates appropriateness of information
flows over time also based on the knowledge about different topics in a
context.
\item Message sending function: uses received messages to update the
different appropriateness and knowledge functions.
\item Message reception function: processes messages before they are
sent to either flag them to the user as inappropriate, avoid
undesirable dissemination of previously exchanged information, and
update appropriateness and knowledge functions.
\end{enumerate}
The authors conclude the paper with experiments based on simulations of
exchanges among multiple IA-agents. The results show that the agents
are able to infer information sharing norms even if a small proportion
of the users follow the norms; agents help reduce the exchange of
inappropriate information and the dissemination of sensitive
information; and, they minimize the burden on the users by avoiding
raising unnecessary alerts.

This paper mostly remains faithful both to the definition of CI as well
as its parameters (see RQ2). The model includes sender, receiver,
messages, topics and context. The authors make no explicit comments
about the transmission principle, however, one could argue that the
agent implements transmission principles: information may flow as long
as it passess the contextual norms of a context, or norms of
dissemination across contexts. Otherwise, the user is presented with an
alert which gives her an opportunity to double-check on the
appropriateness of an information flow.

The authors assume that contexts emerge in interaction, an approach very
much aligned with \citet{dourish2004we}. Contexts are not predefined, but as
communities of users establish connections and communications they are
detected by the ``community finding
module''. Hence, users' communication
patterns, networking patterns as well as the IA-agent become sources of
normativity (see RQ3). This implies a division of labor between the user
and the agent: the agent plays an active role in maintaining
informational norms and the user is still able to practice discretion
when it comes to determining what is considered an appropriate
information flow in a context.

In addition, the authors introduce a number of distinctions and
parameters that go beyond those of CI (see RQ4). The distinction between
inappropriate exchange and undesirable dissemination allows the authors
to express norms with respect to information flows within and
\textit{across multiple contexts}.  

The different functions for frequency, reciprocity and prior knowledge
give the authors the tools to explore adaptivity of informational norms
throughout time and in multiple contexts. This allows the authors to
capture norm development and also make explicit the role that users
play in maintaining norms. In many ways, the
``implicit CI model'' the authors
introduce is complementary to CI, in that it provides means to extend
social norms in a context with changes to those norms through
interactions over time. Adaptivity, multiple-contexts, temporality and
user engagement in contextual norms are further discussed in Section
\ref{CI4.4}.

The model underlying the IA-Agent also exhibits some differences in
interpretation of aspects of CI. The agent relies on frequency of
exchanges on topics as a means to infer norms. Norms are not the same
as the most frequent information flows, nor would such a definition do
justice to topics that are pertinent but infrequently exchanged. 

Finally, the proposed IA-agent helps maintain contextual integrity but
is outside of the scope of CI analysis. The appropriateness of
information flows to the OSN provider, the provider of the IA-Agents as
well as other third parties is not discussed. It is as if CI only
applies to social relations but the service providers are outside of
the scope of CI. This leaves out questions like whether an IA-Agent
should compile and keep all past communications of all members of a
social network, and if so, who can have access to the
Agent's memory? This aligns with industrial practices
where OSN companies claim that they are only facilitating information
flows deemed appropriate by their users. It is possible to argue that
what norms should apply to an IA-Agent is too much to ask of a single
CS paper. However, this type of scoping is not exceptional among the
papers we found and worthy of a lengthier discussion which we come back
to in Section \ref{CI5}.

\section{Results}
\label{CI4} 

Through our study of 20 computer science papers invoking contextual
integrity we discovered a variety of themes and innovations in privacy
engineering that also reflect on improvements to the privacy framework.
After parsing each paper into our review template (see Section 3), we
coded our results and surfaced a number of recurring themes. We then
consolidated these themes into answers to our research questions. We
discuss those answers in this section.

The papers included in the survey were:
Barth et al., 2006 \cite{barth06sp};
Barth et al., 2007 \cite{barth07csf};
Criado and Such, 2015 \cite{criado2015implicit};
Datta et al., 2011 \cite{datta2011understanding};
Jia et al., 2017 \cite{jia2017contexiot};
Kayes and Iamnitchi, 2013a \cite{kayes2013aegis};
Kayes and Iamnitchi, 2013b \cite{kayes2013out};
Krupa and Vercouter, 2012 \cite{krupa2012handling};
Netter et al., 2011 \cite{netter2011assisted};
Omoronyia et al., 2012 \cite{omoronyia2012caprice};
Omoronyia et al., 2013 \cite{omoronyia2013engineering};
Salehie et al., 2012 \cite{salehie2012adaptive};
Samavi and Consens, 2012 \cite{samavi2012l2tap+};
Sayaf et al., 2014 \cite{sayaf2014mathrm};
Shih and Zhang, 2010 \cite{shih2010towards};
Shih et al., 2015 \cite{shih2015privacy};
Shvartzshnaider et al., 2016 \cite{shvartzshnaider2016learning};
Tierney and Subramanian, 2014 \cite{tierney2014realizing};
Wijesekera et al., 2015 \cite{wijesekera2015android};
and Zhang et al., 2013 \cite{zhang2013no}.

The three categories we derived to
answer RQ1, RQ3, and RQ4, and the themes within each category as they
apply to each paper, are in Table 2.2 at the end of Section \ref{CI4.4}. A
separate table (Table 2.1) is provided for our results for RQ2 in Section
\ref{CI4.2}. In Sections \ref{CI4.1}-\ref{CI4.4}, we detail each
set of results and its
relation to our research questions. Blank fields in the tables stand
for cases where none of the themes in our taxonomy were applicable to
respective papers in question.

\subsection{(RQ1) Architecture}
\label{CI4.1}

Our first research question was
``What kind of problems and solutions do computer
scientists use CI for?'' CI is a philosophical theory
of privacy with social and legal implications that are designed to
apply across a wide range of technologies. Computer scientists do not
have the luxury of this insensitivity to technical detail. Their work
reveals how specific classes of technical architecture have different
socially meaningful implications for privacy.

There was variation in the kind of system
described in each paper. Far from being neutral with respect to the way
CI was used by the papers, focus on different technical architectures
resulted in different manifestations of the privacy theory. Some
themes within this category were \textbf{user interfaces and experience
}(\ref{CI4.1.1}), \ \textbf{infrastructure }(\ref{CI4.1.2}),
and \textbf{decentralized} architectures (\ref{CI4.1.3}).

\subsubsection{User Interfaces and Experiences}
\label{CI4.1.1}

Four papers surveyed (\citet{shih2010towards};
\citet{shih2015privacy}; \citet{zhang2013no};
\citet{wijesekera2015android}) studied the experience of users
with applications or with designing user facing interfaces or
applications. Since contextual integrity theory operates at the level
of social norms and says little about user interfaces, and user
experience in different situations, these papers raise the question of
how user-facing apps and their interfaces are related to broader
normative questions about what is appropriate information flow in
differentiated socio-technical situations. One paper
(\citet{zhang2013no}) explicitly drew its motivation from
the FTC's view of the importance of
``context of interaction'' rather
than a broader social or normative view of privacy. Nevertheless, these
cited contextual integrity as part of its motivation and study set up.
This prompts contextual integrity theorists to address the theoretical
connection between `context of
interaction' and social spheres.

In general, these papers were not concerned with modeling social norms
of a large population of users. Rather, they were more concerned with
individual user's activity and their interaction with a
device in different situations. Situations could be environmental
conditions (e.g., where the user is located, night or day); social
situations (e.g., work, home, among friends); or, technical situations
(e.g., whether an app is in use when it asks for permissions),
comparable to conceptions of context a la \citet{dey2001conceptual}.

Two themes were common to these papers: they implicitly highlighted that
in addition to what may be appropriate flow of information in different
\textit{social spheres}, users may have further criteria for what
information flows they expect or prefer in different
\textit{situations} (See Section \ref{CI4.2} for further discussion on this
topic). Second, these papers aspired to generalize their results in
order to provide recommendations concerning infrastructural design that
can be implemented to respect contextual integrity. For example,
\citet{wijesekera2015android} propose
introducing techniques to improve Android permission models to better
cater to user preferences and expectations in different interactional
situations.


\subsubsection{Infrastructure}
\label{CI4.1.2}
  
Many of the papers in our sample were about formal models for or
techniques specific to systems that serve as \textit{infrastructure}.
By infrastructure, we mean technology that is designed to cater to a
large set of users and diversity of applications. We distinguish
between social and technical platforms since they raise different kinds
of challenges to applying CI in practice.

\paragraph{Social platform:} A social platform is a technology that
mediates social interaction as an affordance or service. In the papers
we surveyed, it was used synonymously with online social networks (OSN)
and social ecosystems (\citet{kayes2013aegis}), examples varying
from Facebook, Snapchat, SMS, to Amazon and Google Play store reviews,
and email.

From the perspective of contextual integrity and privacy, what is most
pressing about social platforms is how they can potentially mediate
activity pertaining to multiple social spheres. Friends, family,
classmates, work associates, and so on may all interact using the same
social platform.

This poses a challenge to contextual integrity because while the
framework is well tailored to evaluating the ethics and impact of a
particular technology by identifying the (singular) context it is in,
social platforms are designed to mediate more than one social context
and perhaps to create entirely new social spheres through their use.

In order to accommodate the uses of the technology in multiple spheres
simultaneously, computer scientists are challenged with modeling not
just the norms within a single social sphere, but contexts in general
and how they interact. Contexts may be very fluid in social platforms.
Papers we reviewed looked at scenarios where contexts may collapse;
multiple contexts may produce conflicting norms \citep{tierney2014realizing}; contexts and social norms may change over time
\citep{netter2011assisted} \citep{sayaf2014mathrm};
and, as in the case of
\citet{criado2015implicit}, how contexts may emerge in interaction.
Contextual integrity scholars have not yet provided much guidance on
how to deal with the fluidity of social contexts and its impact on how
to interpret informational norms, leaving computer scientists to come
up with creating solutions themselves.

Note that the definition of a social platform is agnostic about the
particular implementation or location of the technology that undergirds
a social service. The technology may be distributed, federated or
centralized; include apps on a smartphone; web pages in a browser;
servers hosted in a ``cloud'', and
telecommunications infrastructure supporting information flow via
Internet protocols. This technical complexity is addressed in what we
call \textit{technical platforms}.


\paragraph{Technical platform:}
A technical platform is a technology that
mediates the interactions between other heterogenous technologies
connecting multiple users. Examples include Android smartphones
\cite{wijesekera2015android},
Platforms for Smart Things \cite{jia2017contexiot},
the Web (and web browsers)
and Smart Grids \cite{salehie2012adaptive}.

A difficulty in defining ``technical
platform'' is that the technology in question is often
designed as a ``stack'' with multiple
layers, each layer being a 'platform'
on which the next one operates.
Hence there are many technologies, such
as Facebook, that are both in a sense
``applications'' that stand on technical
platforms and are also technical platforms in their
own right as they
mediate other applications through a developer API.
This is in part due
to a design principle that has influence on the
Internet \cite{clark2002tussle} that recommends having as few controls as possible
introduced on
each layer to allow for a wider range of possibilities at
the higher
layers.

From the perspective of contextual integrity, the challenge with
analyzing technical platforms is that they necessarily involve the
participation of many social actors who may access and process data
(and especially personal information) flowing through them. In
contemporary applications the actors involved with operating the
technical platform are subject to a number of technical, legal and
social norms, some of which are substantial to the social contexts
their users see themselves as operating in. We tentatively propose the
concept of ``operator context'' that defines
the roles and norms of the operator of communications infrastructure
that acts between users. 

\paragraph{Formal models:} By formal models we refer to papers that
conceptualize frameworks that can be part of an infrastructure that
serves many different social contexts or technologies, but the
implementation details of which are either irrelevant or considered
only at an abstract level. Such papers come with verification of the
consistency and completeness of the formal model as well as a prototype
to show the feasibility of actually implementing the system. These
papers provide useful insight into how CI can be operationalized,
raising issues at the logical level that are difficult to surface in
more empirical work.

Examples include papers on access control models that preserve
contextual integrity in an enterprise, like \citet{barth06sp} and
\citet{barth07csf}; frameworks that describe and evaluate ontologies
to audit privacy relevant processes in a linked data environment
\cite{samavi2012l2tap+}; or adaptive systems that monitor when new
threats arise, reconfiguring information flows to continue matching
user privacy requirements \cite{omoronyia2012caprice},
or that identify
when information norms themselves change \cite{shvartzshnaider2016learning}. 

\subsubsection{Decentralization}
\label{CI4.1.3}

The rare paper in our sample dealt with a specifically challenging
technical feature: decentralized architectures. We highlight this
theme, however rare, because of the way it positions technology
relative to social spheres and interactions. While user interfaces and
experiences are connected to individual users (and their expectations),
social platforms are central and common to a large number of diverse
social contexts. Decentralized architectures have an interactivity and
topological complexity that mirrors that of society itself, and trust
and reputation mechanisms come to play a greater role in the absence of
a centralized entity that can arbiter information norms. We look
forward to more papers on this theme and CI.

\subsection{(RQ2) What did they talk about when they talked about context?}
\label{CI4.2}

Our second research question was ``How have the authors
dealt with the conceptual aspects of CI?'' Contextual
integrity theory has a specific understanding of context as social
sphere, parameterized by roles, norms, purposes, and values. The norms
are parameterized by their actors (senders, receivers, and subjects in
contextually defined roles), information topics, and transmission
principles. We wanted to know whether and how computer science papers
used this conceptualization of privacy. We found that while several
papers drew closely from the concepts in CI, others represented context
very differently. As we have discussed, many computer scientists
interpreted `context' in a way that
draws from the research field of ubiquitous computing (See Section
\ref{CI2.2.1}). Because of these discrepancies, we have chosen to focus on the
nuanced differences in how context is represented rather than on which
of the parameters are used.

We have coded the way each paper has defined and used context across
five binary dimensions, which we have named: substantiality, domain,
stability, valence, and epistemology. Within each dimension there are
two opposed poles.

\bigskip

\liststyleWWNumii
\begin{itemize}
\item \textbf{Substantiality}. Some papers discuss contexts as an
\textbf{abstract} type or ideal of a situation. Others discussed
contexts as \textbf{concrete} happenings and situations.
\textit{Example: hospitals in general are an abstract context. Mount
Sinai Beth Israel hospital in Manhattan is a concrete context.}
\item \textbf{Domain}. Some papers discuss \textbf{social} contexts,
defined by configurations of people to each other. Others discuss
\textbf{technical} contexts, defined by objective properties of
mechanical devices and the environment they were in. Some papers
understood contexts as combining \textbf{both} social and technical
factors. \textit{Example: a classroom with a teacher and students is a
social context. A language education mobile app that prompts the user
with questions and sends results back to a server for analysis is a
technical context.}
\item \textbf{Stability}. We draw on \citet{dourish2004we} for this
distinction. Some papers treat context as a \textbf{representational}
problem, as if they were stable, delineable, and distinct from the
activity that contained them. Others treat them more as an
\textbf{interactional} problem, as arising from interactions between
people and things, defined by specific circumstances. \textit{Example:
The Oval Office in the White House is a stable context. A flash mob is
an interactional context}.
\item \textbf{Valence}. Some papers see the \textbf{normative} aspects
of privacy as being inherent in context. Others treat contexts merely
\textbf{descriptively}, without normative force. \textit{Example: A
conference Code of Conduct is an account of norms inherent in a
context. A list of attendees, keynote speakers, and program committee
members is a description of the context.}
\item \textbf{Epistemology. }Some papers adopt a \textbf{model-building}
approach to defining contexts. They posit a schema or model of context
and derived conclusions from it. Other papers take a more
\textbf{empirical} approach, deriving context definitions from data. A
parameterized definition of a context, e.g., context is location, time,
and activity, is an example of a model based approach, whereas applying
traffic and topic analysis to communications in order to surface
contexts is an example of an empirical approach that can be used to
characterize different contexts.
\end{itemize}

\bigskip

We note that as far as CI is concerned, it is essential that contexts be
understood as \textbf{normative}, as one important trait of contexts is
that they have ends, purposes, and values. They are \textbf{social}
contexts, pertaining to relationships between people in defined roles,
but they are oriented around functions, purposes, aims, goals,
activities, values, etc. As these social norms evolve in society in
general and then are applied to particular cases of information flow,
contextual integrity conceptualizes contexts \textbf{abstractly}.
``Context'' interpreted to mean
\textit{sphere}, as discussed above, has these three properties (i.e.
they are normative, social, and abstract). To the extent the papers
draw on different meanings of context, they diverge from CI. For
example, when the literature interprets context as \textit{situations},
as discussed in Section \ref{CI2.2.1}, it conceptualizes contexts as
\textbf{concrete} and at least partly \textbf{technical}. Our study has
surfaced that computer scientists, in trying to make CI actionable,
have encountered the problem of applying abstract social norms to
concrete socio-technical situations.

\clearpage

\begin{landscape}
  \begin{table}
    \label{tab:RQ2}
 \caption{Results from RQ2.}
\begin{center}
 \begin{tabular}{| c | c | c | c | c | c |} 
 \hline
  \textbf{Paper} &
  \textbf{Substantiality} &
  \textbf{Domain} &
  \textbf{Stability} &
  \textbf{Valence} &
  \textbf{Epistemology} \\
 \hline\hline
\citet{barth06sp} &
ABSTRACT &
SOCIAL &
REPRESENTATIONAL &
NORMATIVE &
MODEL \\
\hline
\citet{barth07csf} &
ABSTRACT &
BOTH &
REPRESENTATIONAL &
NORMATIVE &
MODEL \\
\hline
\citet{criado2015implicit} &
CONCRETE &
SOCIAL &
INTERACTIONAL &
NORMATIVE &
EMPIRICAL \tabularnewline
\hline
\citet{datta2011understanding} &
ABSTRACT &
SOCIAL &
REPRESENTATIONAL &
NORMATIVE &
MODEL \tabularnewline
\hline
\citet{jia2017contexiot} &
CONCRETE &
TECHNICAL &
REPRESENTATIONAL &
DESCRIPTIVE &
EMPIRICAL \tabularnewline
\hline
\citet{kayes2013aegis} &
CONCRETE &
SOCIAL &
REPRESENTATIONAL &
DESCRIPTIVE &
EMPIRICAL \tabularnewline
\hline
\citet{kayes2013out} &
CONCRETE &
SOCIAL &
REPRESENTATIONAL &
NORMATIVE &
MODEL \tabularnewline
\hline
\citet{krupa2012handling} &
ABSTRACT &
SOCIAL &
REPRESENTATIONAL &
DESCRIPTIVE &
MODEL \tabularnewline
\hline
\citet{netter2011assisted} &
CONCRETE &
SOCIAL &
REPRESENTATIONAL &
NORMATIVE &
EMPIRICAL \tabularnewline
\hline
\citet{omoronyia2012caprice} &
ABSTRACT &
BOTH &
REPRESENTATIONAL &
DESCRIPTIVE &
MODEL \tabularnewline
\hline
\citet{omoronyia2013engineering} &
ABSTRACT &
BOTH &
REPRESENTATIONAL &
DESCRIPTIVE &
MODEL \tabularnewline
\hline
\citet{salehie2012adaptive} &
CONCRETE &
BOTH &
INTERACTIONAL &
DESCRIPTIVE &
EMPIRICAL \tabularnewline
\hline
\citet{samavi2012l2tap+} &
ABSTRACT &
BOTH &
REPRESENTATIONAL &
NORMATIVE &
MODEL \tabularnewline
\hline
\citet{sayaf2014mathrm} &
CONCRETE &
SOCIAL &
INTERACTIONAL &
DESCRIPTIVE &
EMPIRICAL \tabularnewline
\hline
\citet{shih2010towards} &
BOTH &
BOTH &
REPRESENTATIONAL &
DESCRIPTIVE &
EMPIRICAL \tabularnewline
\hline
\citet{shih2015privacy} &
CONCRETE &
TECHNICAL &
INTERACTIONAL &
DESCRIPTIVE &
EMPIRICAL \tabularnewline
\hline
\citet{shvartzshnaider2016learning} &
ABSTRACT &
SOCIAL &
REPRESENTATIONAl &
NORMATIVE &
BOTH \tabularnewline
\hline
\citet{tierney2014realizing} &
BOTH  &
SOCIAL &
REPRESENTATIONAL &
NORMATIVE &
MODEL \tabularnewline
\hline
\citet{wijesekera2015android} &
CONCRETE &
TECHNICAL &
INTERACTIONAL &
DESCRIPTIVE &
EMPIRICAL \tabularnewline
\hline
\citet{zhang2013no} &
CONCRETE &
TECHNICAL &
INTERACTIONAL &
DESCRIPTIVE &
EMPIRICAL \\
\hline
 \end{tabular}
\end{center}
\end{table}
\end{landscape}

The first paper of our study in publication order is \citet{barth06sp}, which we have detailed in Section \ref{CI3.3.1} of this paper. Helen
Nissenbaum is a coauthor and the paper includes a summary of contextual
integrity theory. The technical contribution of the paper focuses on a
``fragment'' of contextual
integrity. It is this technical contribution that we have assessed
according to the criteria above. The \citet{barth06sp} technical
presentation of context is as one that is \textbf{abstract, social,
representational, normative, }and\textbf{ modeled}. Their work models
the specific normative logic of contextual integrity.. It shows how
norms and laws can be represented as abstract policies amenable to
automated enforcement.

This paper is one end of a spectrum. Other papers in our sample drew
their understandings of context from other traditions, including
ubiquitous computing (discussed in Section \ref{CI2.2.1} of this paper).

Following \citet{dourish2004we}, some papers eschewed explicit abstract
representational modeling of context for what resembles interactional
views of context derived from empirical data about user behavior or
human-computer interaction. Several papers considered the narrow
context of a user and their device, as opposed to social relations more
generally. Most papers did not see norms as inherent to the contexts
they studied, but rather saw contexts descriptively. (Some of these
papers sourced their normativity from other factors, see Section \ref{CI4.3}).
Our paper exemplars (\ref{CI3.3.1}-\ref{CI3.3.3}) provide deeper explanations of the
dimensions used to classify contexts here.

What we have discovered in answer to RQ2 is the distribution of papers
across these dimensions. This tells us how well contextual integrity as
a conceptual theory of privacy has made it into computer science. CI
conceptualizes contexts as \textbf{normative} and \textbf{social}.
Papers that have modeled context as either purely technical or purely
descriptive have missed some of the core intent of CI.

To the extent that it sees the formation and maintenance of a social
context as an adaptive social process, we argue that contextual
integrity is consistent with the \textbf{interactional} view of
contexts from \citet{dourish2004we},
though in its concrete application it has
a tendency to work from a \textbf{representation} of context. We
believe this leads to deep sociological questions about how social
norms and purposes, which can seem \textbf{abstract} and theoretical,
can form from \textbf{concrete} human interactions. 

We note with special interest \citet{criado2015implicit},
detailed in Section \ref{CI4.1.3},
which stands out as a paper that addresses a particularly difficult
challenge. It is the only paper in our sample that manages to be both
concrete, interactional, and empirical as well as socially normative.
We see this as an important innovation in the use of contextual
integrity in computer science.

\subsection{(RQ3) Source of Normativity}
\label{CI4.3}

Our third research question was, ``How have the authors
dealt with the normative aspects of CI?'' In
contextual integrity, the normative (in the sense of prescriptive or
ethical) force of information norms comes from the purposes, ends, and
values associated with each social sphere. This complex metaethical
theory rarely finds its full expression in the computer science
literature. Instead, the papers in our sample take a variety of
narrower positions, implicitly or explicitly, on the source of
normative values that motivate the importance of privacy.

The subsections here explain the themes we found in this category.
\textbf{Compliance and policy} refers to when normativity was taken
from legal authority or some other unquestioned source of policy.
\textbf{Threats} refers to the computer security research practice of
positing a threat model to motivate research. \textbf{User preferences
and expectations} locates the source of normativity in the subjective
perspective of individual users. \textbf{Engagement} refers to designs
that allow users to dynamically engage with each other to determine
norms.

\subsubsection{Compliance and Policy}
\label{CI4.3.1}

Some of the papers in our sample took their motivation from the
practical problem of compliance with legal regulation, such as HIPAA.
These papers effectively outsource their normative questions to the
legal system. They at times argue as if compliance is relevant because
it is internalized as a business interest \cite{barth07csf}.
One line of this compliance-based research is contiguous with
other work on
formalizing privacy regulations in ways that are less closely tied to
contextual integrity \cite{DeYoungGJKD10}.
\citet{datta2011understanding} synthesize the contributions of this
research trajectory.

Other papers are less specific about source of the specific form of
their restrictions, but nevertheless have an explicit mechanism for
stating \textit{policy}. Some computer research in this field
culminates in the articulation of a policy language, which is valid for
its expressivity, not for the specific character of the content of any
particular expression it allows.

In both the cases of compliance and policy, normativity is exogenous to
the technical design. 

\subsubsection{Threats}
\label{CI4.3.2}

Some of the papers motivated their research goals in terms of privacy
\textit{threats}. These presumably adopted this stance as a
continuation of practices from security research, which typically
posits a threat model of potential attacks and adversarial capabilities
before detailing how a technical improvement can mitigate these
threats.

Taking this position alleviates the researcher from having an
overarching theory of privacy; they can instead work from specific
cases that are plausible or undisputed cases of privacy violation.

\subsubsection{User Preferences and Expectations}
\label{CI4.3.3}

\bigskip

Some papers motivated their research either explicitly or implicitly in
terms of whether a technical design was able to meet user preferences
or expectations of privacy. Preferences and expectations are not the
same thing, but they are related in that they depend primarily on the
individual subjectivity of the user. A user's
expectation is the outcome they desire or is in their acknowledged
interest, and a number of papers explore the expectations users have in
different social or interactional context. User preferences on the
other hand were often used to study what kind of controls users may
prefer to have or exercise when using systems. User perceptions also
played a role in papers where researchers explored what information
flows users noticed or how they perceived them
\cite{wijesekera2015android} \cite{zhang2013no}.

Measuring user expectations and preferences as a way of assessing the
appropriateness of information flow is consistent with contextual
integrity. This can be done explicitly through survey methods, as is
done by \citet{shvartzshnaider2016learning}. In
CI, appropriateness is a function of social norms, and these norms do
codify social expectations and values. Certainly in some cases user
expectations will track social expectations. But though they are
related, we caution researchers against conflating social norms with
user expectations and preferences. This is because individual users are
more prone to becoming unreflectively habituated to a new technology
than society as a whole. Also, individual user preferences may at times
be opposed to the interests of society. We have identified elaborating
on the relationship between individual preferences and social norms as
a way to improve CI.


\subsubsection{Engagement}
\label{CI4.3.4}

Some papers explicitly articulated mechanisms through which users could
engage with a system to define what's normative for the
system. Rather than accept a policy or threat model exogenously or see
an individual's opinions and satisfaction as the ends
of design, these papers allowed for the establishment of norms to be a
dynamic social process accomplished through use of the technology
itself. For a more in depth discussion of how this can work, see the
more detailed discussion of \citet{criado2015implicit} in
Section \ref{CI3.3.3}.
Another example is \citet{tierney2014realizing} who describe a
marketplace or library of abstract context definitions, complete with
roles and access controls corresponding to transmission principles,
that are developed by a community of context designers. Users can then
instantiate the context template that best fits their social needs.

\subsection{(RQ4) Expanding Contextual Integrity}
\label{CI4.4}

Our fourth research question was ``Do the researchers
expand on contextual integrity?'' The rigors of
computer science led many paper authors to innovate and improvise as
they used contextual integrity in their designs. We grouped these
innovations into the category \textbf{Expanding Contextual Integrity}.
We found many papers were engaged in developing mechanisms for
technological \textbf{adaptation} to changing social conditions
(\ref{CI4.4.1}). Some addressed the challenges associated with technologies
that operated within \textbf{multiple contexts} at once (\ref{CI4.4.2}). Some
developed ideas concerning the \textbf{temporality and duration} of
information and how this affects privacy (\ref{CI4.4.3}). Others were
particularly concerned with \textbf{user decision making} (\ref{CI4.4.4}) with
respect to privacy and information controls. While all these
innovations are compatible with contextual integrity as outlined in
\citet{nissenbaum2009privacy}, we found the detail with which the paper authors
engaged these topics showed ways to expand contextual integrity.

We note that many of these themes echo discoveries made with respect to
our other three research questions. For example, those papers that
addressed the design of social infrastructure (see Section \ref{CI4.1}) had to
address the problem of how to handle multiple contexts in the same
technology, and as they did so they had to make decisions about how to
represent context that did not necessarily accord with
CI's concept of context as social sphere (see section
\ref{CI4.2}). Of the four research questions, this one reflects the technical
accomplishments discussed in Sections \ref{CI4.1}-\ref{CI4.3}
back on CI in order to
identify the limits of the framework itself. Table 2.2 shows how themes
from different research questions were distributed across the papers in
the survey.

\subsubsection{Adaptation}
\label{CI4.4.1}

The most common way in which computer science papers expanded on
contextual integrity was to address questions of social adaptation.

As noted in Section \ref{CI2.1} above, CI theorizes that norms are the result of
a process of social adaptation. Social spheres have ends, purposes, and
values robustly as a function of their evolution. Norms within these
spheres are legitimate to the extent that they serve their contextual
purposes, but environment changes (such as the prevalence of new
digital technologies) are the stimulus for further adaptation of norms.
To the extent that CI has a conservative impulse, it is to warn against
the agitation caused by disruptive technologies that change the
environment too quickly for social evolution to adapt.


This grand theory of privacy is not actionable for computer scientists.
In the papers we found that dealt with \textit{adaptation}, the
researchers were interested in designing technology that is responsive
to social change at a much smaller scale in both space and time.
\citet{criado2015implicit} discuss the adaptation of an
informal sports discussion
group emerging out of a collegial working forum. If large-scale
evolution of social spheres and privacy norms depends on variation on
the level of social interaction, it is challenging to design technology
that keeps up with this variation. If large scale agitation about
threats to privacy happens when technology disrupts a shared social
expectation, then small scale agitation can occur when technology fails
to address emerging norms. For computer scientists to deal with these
challenges, they have to be more specific about these processes of
adaptation than CI currently is.

Many of the papers we reviewed concerned themselves with the problem of
maintaining contextual privacy under conditions of social change. Few
adopted the theory proposed by \citet{nissenbaum09book}; instead these papers
proposed their own mechanisms to account for and capture changes in
context and norms. Most often these did not take into account the
stability of contextual ends, purposes, and values. Rather, they
generally took on the problem of having technology react appropriately
to exogenous social change. \citet{criado2015implicit}
design agents that
guess rules for appropriate information flow from regularities in user
behavior. \citet{shvartzshnaider2016learning} experiment with a method for
empirically surveying for opinions on social norms and translating
results into a logical specification. Such mechanisms could be used to
build a system that is robust to changes in social opinion.

Papers that addressed social adaptation were likely to also use
concrete, interactional and empirical concepts of context (see Section
\ref{CI4.2}). Some designed methods to have users engage in the process of
determining norms (see \ref{CI4.4}). In general, technical systems that are
adaptive to changes in social behavior can be prone to the failure of
maladaptation. To be actionable for these designs, CI would benefit
from more specificity regarding the process of social evolution that
legitimates the norms of social spheres.

\subsubsection{Multiple Contexts, and Context Clash}
\label{CI4.4.2}

Another common way in which computer science papers expanded on
contextual integrity is that many discussed technologies that
recognized the existence of multiple contexts at once. This was common
for those papers that addressed the design of social infrastructure
(see Section \ref{CI4.1}), for example.
Contextual integrity as a privacy
framework posits many different social spheres with different norms of
information flow. But as it is currently resourced, CI provides little
conceptual clarity as to how different contexts relate to each other,
and no normative clarity as to how this multiplicity of contexts
affects the appropriateness of information flow.

As a result, many of the paper in our study improvised solutions to the
problems associated with representing multiple social contexts. In
some, system users were registered or detected as being in one or
another context, with shifting access control policies in a
context-appropriate way, something the agent in the
\citet{criado2015implicit} paper is tasked with reasoning about.
Some papers accommodated
the relationship between contexts through a mechanism of context
adaptation (see above). Others addressed the specific problem of what
happens when information \textit{flows between contexts}. For example,
\citet{sayaf2014mathrm} raised the privacy concern that a photograph might
move from a context where it was interpreted as a swimsuit
advertisement into one where it was sexually objectified. 

All the papers that dealt explicitly with the problem of using CI when
multiple contexts affected a situation used a concrete and empirical
concept of context (see Section \ref{CI4.2}).
This points to an insight about CI that we see as a research finding:
a more actionable CI would address how situations (concrete context)
can be empirically analyzed to determine which sphere or spheres
(abstract, normative, social contexts) apply. For example, could a
system that monitors communication within a university in general
classify a particular message as belonging to a classroom, employment,
or social sphere? It may be possible to formulate this as a machine
learning problem.

\subsubsection{Temporality and Duration (Read/Write)}
\label{CI4.4.3}

Several of the papers in our sample extended contextual integrity by
explicitly addressing restrictions or allowances on information flow
based on the timing of flows. For example, a flow might be allowed
after the sender has received permission, but not before, or until
certain actions are completed in the future. These extensions are not a
challenge to contextual integrity as a theory; they are fully within
the scope of what is possible as a \textit{transmission principle}.
However, the specific elaborations of the relationship between timing
and information flow policies were notable.

A related theme which does more conceptual work within contextual
integrity is that of data's \textit{duration}. In
technical terms, this was expressed in our sample as restrictions of
reading, writing, and deleting data, as found in \citet{kayes2013aegis}.
These operations stretch the idea of information
``flow'' so much that they perhaps
require an entirely different notion, that of information
``stock''.

Another line of research discusses the relationship between temporality
and the possibility of privacy policy enforcement.
\citet{datta2011understanding}
note that some aspects of privacy policies cannot be completely
enforced at the time when information flows because the policy mandates
what happens \textit{after} the flow. For example, some policies impose
restrictions on how information is used.

\subsubsection{User decision making}
\label{CI4.4.4}

Contextual integrity as a theory of privacy abstracts away from
individuals in order to provide a normative framework that is
independent of specific actors and their interests. It is this
stability that gives it much of its normative power. Nevertheless, many
computer science papers that used contextual integrity were concerned
with user's individual decision making.


While voluntarity is one factor that can affect the transmission
principles of information norms, contextual integrity has little to say
about the role of the individual in shaping norms and social contexts
more generally. These computer science papers put emphasis back on the
individual and her decisions in context.

\clearpage
\begin{landscape}
  \begin{table}
    \label{tab:RQX}
    \caption[Results from RQ1, RQ3, and RQ4.]{
      Results from RQ1, RQ3, and RQ4.
      RQ1 Codes:
      FM = Formal Model;
      I:S = Infrastructure:Social;
      I:T = Infrastructure:Technical;
      UI = User Interface.
      DC = Decentralized.
      RQ3 Codes:
      COMP = Compliance.
      ENGAGE = Engagement.
      UP = User Preferences.
      UE = User Expectations.
      RQ4 Codes:
      TEMP = Temporality.
      ADAPT = Adaptation.
      MC = Multiple Contexts.
      UDM = User Decision Making.
      
    }
    \begin{center}
      \begin{tabular}{| c | c | c | c |} 
        \hline
      \textbf{Paper} &
\textbf{RQ1. Architecture} &
\textbf{RQ3. Source of Normativity} &
\textbf{RQ4. Expanding CI}\\\hline\hline
\citet{barth06sp} &
FM &
COMP &
TEMP\\\hline
\citet{barth07csf}&
FM &
COMP &
TEMP\\\hline
\citet{criado2015implicit} &
I:S &
ENGAGE &
ADAPT, MC\\\hline
\citet{datta2011understanding} &
FM &
COMP &
TEMP\\\hline
\citet{jia2017contexiot} &
I:T &
THREATS &
~
\\\hline
\citet{kayes2013aegis} &
I:S &
ENGAGE, COMP, UP &
MC, ADAPT, TEMP
\\\hline
\citet{kayes2013out} &
I:S &
~
 &
~
\\\hline
\citet{krupa2012handling} &
DC &
ENGAGE, UP &
ADAPT\\\hline
\citet{netter2011assisted} &
I:S &
THREATS, UP &
~
\\\hline
\citet{omoronyia2012caprice} &
I:S &
THREATS, UP &
ADAPT\\\hline
\citet{omoronyia2013engineering}&
FM &
THREATS, UP &
ADAPT\\\hline
\citet{salehie2012adaptive} &
I:T &
THREATS &
ADAPT\\\hline
\citet{samavi2012l2tap+}&
I:T &
COMP, UP &
TEMP\\\hline
\citet{sayaf2014mathrm} &
I:S &
UP &
MC, TEMP\\\hline
\citet{shih2010towards} &
UI &
UP &
MC\\\hline
\citet{shih2015privacy} &
UI &
UP &
UDM\\\hline
\citet{shvartzshnaider2016learning} &
FM &
UE &
ADAPT\\\hline
\citet{tierney2014realizing} &
I:S &
ENGAGE &
~
\\\hline
\citet{wijesekera2015android} &
UI &
UE, UP &
APPS AS ACTORS\\\hline
\citet{zhang2013no} &
UI &
THREATS &
MC\\\hline
\end{tabular}
\end{center}
\end{table}
\end{landscape}

\section{Findings and discussion}
\label{CI5}

We have summarized the achievements of
computer scientists in developing contextual integrity. The research we
have reviewed has variously used parts of contextual integrity and
innovated on the relationship between privacy and context. Through our
analysis, we have identified new research questions and opportunities
at the intersection of CI and computer science.

In the time since contextual integrity
first emerged, it has attracted useful insights from legal and ethical
theorists as well as social scientists. Some of the toughest challenges
have come from those seeking to apply CI to problems in their home
fields, whether law and public policy or computer science, design, and
engineering -- the focus of this paper. Like most efforts to apply
theory and other abstractions to concrete or real-world challenges,
these, too, require that a distance be traveled to leverage the
theoretical constructs of contextual integrity, to concrete privacy
challenges of computer science, design, and engineering. In traveling
this distance, the efforts we have surveyed reveal unanswered
questions, conceptual gaps, and realities that do not align fully with
the CI model. These findings call attention to several specific ways to
expand, explain, and adjust CI in order to make it more responsive to
the needs of computer science and engineering researchers seeking to
inform their work with a meaningful account of privacy. 

In this final section, we present
theoretical gaps in CI
that our literature survey has exposed,
systematically organizing them into four subsections, each associated
with our four research questions: RQ1 - Architecture; RQ2 - Character
of Contexts; RQ3 - Privacy as a Moral Imperative; and RQ4 - Expanding
CI. In each subsection, we describe the nature of the theoretical gaps,
i.e between theory and practical application, followed by a discussion
of lessons learned that could translate into guidance for those
embarking on new technical privacy research and design projects. The
task is challenging because although the parameterized informational
norms of contextual integrity offer greater specificity than other
normative privacy theories, there remains significant room for
interpretation. This room for interpretation, on the one hand, is what
distinguishes contextual integrity from accounts of privacy that are
not adaptive in the face of historical, cultural, social, and even
personal variations, but it can be frustrating for those looking for
precise, literal rules that are both correct and directly
implementable. 

For each research question we also have
sections that we have titled, ``call to
action,'' in which we discuss the lessons learned
from past applications that can positively inform further forays into
using CI in privacy research. We encourage the creative spirit we
observed in our survey and recommend lessons learned and open questions
to inspire future researchers in the field of context integrity.

\subsection{Architecture: Towards a Modular Contextual Integrity}
\label{CI5.1}

Corresponding to our
\textbf{RQ1,}
we have discovered that the way contextual integrity is used in
technical design depends on the architectural properties of the
technology being designed. This presents an opportunity for faceting CI
into more specialized programs that are targeted at specific classes of
technical problems. At the same time, our study revealed that the
technical designs of computer science researchers often bracketed the
social roles of those operating technical and social platforms, despite
these being central to public discussion and scholarship on privacy and
technology.

\subsubsection{Theoretical gaps}
\label{CI5.1.1}

We see the demand for ``modular contextual
integrity'', faceting CI and giving guidelines for
design and research at specific levels of the technical stack, be it in
designing user interfaces or experiences,technical or social platforms,
or devising formal conceptualizations. Providing these guidelines may
require that we derive frameworks of heuristics and principles from
CI's conceptual and normative facets. We expect to do
this in tandem with further elaboration of the fundamental concept of
``context'' (see Section \ref{CI5.2}) and
the concept of Transmission Principle, both distinctive of CI and often
not well understood, despite their importance for CI's
power as a normative, or ethical theory. 

An example of a promising strategy to address this problem is to
identify and describe social spheres specific to the design, provision,
operation, and use of technology. This is especially relevant in those
papers where the designers explicitly delegated responsibilities for
enabling contextual integrity to technical elements. In the case of
\citet{criado2015implicit}, the agent co-regulates norms.
In \citet{wijesekera2015android} apps actively take
part in asking for information flows
and the authors consider a classifier that would reason as to when
information flows may breach contextual integrity.
In \citet{samavi2012l2tap+}, the
authors produce an auditing mechanism that checks logs for potential
breaches to contextual integrity. These mechanisms are very different
with respect to the degree of autonomy they provide to technical agents
and those who are operating them. However they all invoke the question:
to what extent these mechanisms are subject to the norms of the context
they are co-regulating, acting in or auditing? Should these mechanisms
be subject to other contextual norms (pertaining to intelligent agents
and their administrators)? In the practical world, this is comparable
to the question of whether operators of systems and the technical
infrastructures they deploy can simply posit themselves as (providers
of) communication channels that are not bound by the social context of
their users. The papers we surveyed consistently treat them as a
product of but not as subject to the application of the CI. We have
raised the possibility of an
``operational'' context, with an
`operator' role empowered with certain
privileges and responsibilities with respect to information flowing on
the platform. Further guidance on this matter will be pertinent to
enabling a holistic application of CI to technical designs.

On a related matter, further guidance is also necessary with respect to
systems that provide infrastructure to multiple contexts: Such
systems are expected to reflect on the normative aspects of CI while
promoting a logic that can provide the flexibility for multiple social
or technical contexts with potentially diverging informational norms to
co-exist. What role the normative and conceptual aspects of CI can and
should play in infrastructure underlying multiple contexts is an open
research question.

\subsubsection{Calls to action}
\label{CI5.1.2}

We call computer science researchers to be as explicit as possible about
how the technologies they design are situated in the broader complex of
platforms, operators, users, and moderators. If there is an implicit
hierarchy (such as users whose activities are logged, agents that track
all conversations, and auditors who use these records, or an operating
system with many dependent applications), computer scientists can be
explicit about this and address the privacy and information flow issues
resulting from this differentiation of roles. If there are critical
roles in the operation of the system (such as an auditor or operator),
can privacy tools inspired by contextual integrity be built for them?

Most papers we selected did not focus on social spheres but on
situations, proposing techniques that surface or implement
informational norms that arise in a representational or interactional
context. This focus often meant that in their models the authors did
not consider normative rules applying to a specific context,
abstracting the social sphere away. Some of this is justified: the
intention is to develop designs that are flexible enough to function in
different social spheres. It is also possible that the authors are more
comfortable making normative judgements about what constitutes a
relevant situation, e.g., some combination of location and activity,
these also being things that the researchers can measure using sensor
data. However, the numerous research papers showing user concerns due
to context clash in online social networks, as well as ever increasing
public calls for curation of user generated content suggests that lack
of attention to informational norms in social spheres may have negative
consequences and should not be an afterthought. One way to guarantee
this in abstract formal models as well as in infrastructure, is to at
least provide a placeholder in associated conceptual frameworks that
can be used to express and enforce normative rules when these systems
are implemented. Better would be to also consider how well a proposed
technique can sustain divergent informational norms pertaining to
different social spheres that an infrastructure comes to play a role
in. Developing and evaluating systems that enable a fluid interaction
between informational norms in a social sphere and user preferences
presents itself here as an interesting research question.

\subsection{Diverse concepts of context}
\label{CI5.2}

Our investigation into how computer scientists conceptualize contexts
when they employ CI revealed diverse and divergent theoretical
assumptions. Some researchers were well aligned with
CI's concept of contexts as abstract, normative, social
spheres; others drew on other traditions such as ubiquitous
computing's concept of context as situations including
users and technology. Still others supported users co-creating contexts
through their engagement with each other and the technology. Some drew
inspiration from multiple sources in order to provide a new technical
solution to privacy.

This variety of work demonstrates that privacy and context are closely
linked. It also demonstrates that \textit{context }is a polysemous
(many-meaning) term. The different senses of context have different
implications for privacy by design. Our survey suggests that no one
sense of context supports either a complete normative theory or
technical design, and that there is a rich design space at the
interplay between diverse specific meanings.

\subsubsection{Theoretical gaps}
\label{CI5.2.1}

Our investigation revealed inductively that computer scientists use
diverse meanings of context that vary across many dimensions.
``Context'' can refer to something
abstract or concrete, social or technical, representational or
interactional, normative or descriptive, and a priori modeled or
empirically discovered. Only a subset of this space of meanings are
addressed by CI in its current form, specifically, in the way it
conceptualizes contexts as abstract, normative, social spheres
continuously evolving within differentiated society.

It is not surprising that technical designs are concerned with the
concrete circumstances of both users and technical applications. In
order for CI to be actionable in this sense, what is needed is a
theoretical account of how social spheres relate to sociotechnical
situations. This account may well address other tensions between the
many senses of ``context''. For
example, an advanced CI would be able to guide how to infer from the
observed, descriptive details of a situation a model of the norms
appropriate to guide behavior within it. This is a philosophical
problem, but one that is made urgent by the demands of existing
research on privacy by design.

Another theoretical challenge to CI is raised by
the \citet{dourish2004we} critique of ubiquitous computing.
CI's model of contexts as social spheres parameterized
by roles, information types, and transmission principles does suggest
what Dourish describes as a positivist model of social contexts:
contexts as containers of social action with specific
expectations and
prescriptions associated with them. To the theorist, we raise the
question: what is the relationship between the situated,
interactional
account of context in Dourish and the social spheres in CI?
The theory
of ``Implicit Contextual Integrity''
invented by \citet{criado2015implicit} has suggested that the
spirit of CI
can be extended to social situations that evolve on a much
smaller and
more specific scale than is currently suggested by CI.
Philosophical
theorists can work to make this claim more precise.

\subsubsection{Calls to action}
\label{CI5.2.2}

Computer scientists need not wait for theoretical prescriptions to
continue to do good work at the intersection of CI and privacy by
design. There is much to be done designing systems that address the
reality that supporting users' privacy requires skillful creation and
moderation of context. We anticipate that the best work will be
explicitly aware of the challenges of matching concrete situations with
the abstract spheres from which CI posits users get their normative
expectations. Applications of CI are especially likely to be relevant
in the smart environment applications, where sensors and actuators will
interact with many users at once, making it hard to rely on individual
preferences and expectations. 

Indeed, any one of the dimensions of variation in the meaning of context
(abstract or concrete, social or technical, representational or
interactional, normative or descriptive, and a priori modeled or
empirically discovered) presents a technical problem to computer
scientists wishing to implement privacy by design based on CI. One
concrete issue that persisted throughout many papers is the scoping of
context. Papers, for example, that focused only on the social context,
be it either due to their focus on user interfaces and experiences or
social platforms, neither considered what we call the operational
context, nor did they pay attention to how social informational norms
may be impacted by flows of personal information to third parties. It
would be very valuable to have studies that not only consider norms of
information flow among users or towards an app provider, but also flows
to other third parties, like other services, companies or governments.
If a study intends to focus only on a subset of the information flows,
than the limitations of the results due to this decision should be made
explicit. We call upon computer scientists to work on pragmatic
solutions to the problems these conceptual discrepancies pose to
designers and users.

\subsection{Privacy as a moral imperative: between bits and norms}
\label{CI5.3}

One major finding from our investigation of
\textbf{RQ3} is that of the papers in
our review used the normative aspect of contextual integrity as a basis
for their technical contributions. In contextual integrity, the
normativity of privacy comes from the ends, purposes, and values of
social contexts (spheres) as they have evolved over time. These ends,
purposes, and values legitimize the norms that determine the
appropriateness of information flow, even as technology changes what
those norms should be. Computer scientists sometimes acknowledge this
aspect of contextual integrity, but they do not ground their technical
contributions in it. Rather they draw on other sources of normativity,
such as threats, user expectations, or legal compliance, to motivate
their work.

For a number of reasons, these moves are understandable. Computer
science has not traditionally equipped itself to deal with the hard
problems surrounding the origins of ethics and morals. Threat modeling
is narrowly pragmatic and has proven to be suitable for engineering
purposes. User expectations are measurable and so attractive to those
concerned with empirical validity. Considerations of legal compliance
are part of the real business logic of functioning organizations. By
focusing on these sources of normativity, computer scientists make
their research more actionable. But these methods also carry the risk
of falling short of socially maintained norms of privacy. Threat
modeling may miss the mark; user expectations can be habituated by
technology that works at odds with social principle; laws may be
unjust. The burden is on contextual integrity theorists to show how its
social and philosophical theory of social norms relates to these more
concrete factors. In turn, we call computer scientists to stretch
towards the social and philosophical sources of normativity. Our survey
has shown that such ambition can lead to new technical innovation.

\subsubsection{Theoretical gaps}
\label{CI5.3.1}

Contextual integrity theorists need to address how their metaethical
theory, whereby norms arise from the evolution of social spheres, ties
in with the concrete sources of normativity used by computer
scientists. We have identified three areas that need elaboration.

Contextual integrity must provide a way of translating from the
information norms of social spheres into a characterization of
enumerated and discrete privacy \textit{threats}. This is connected to
the task of deriving mid-level theories of CI for modules of the
technical stack (see Section \ref{CI5.1.1}).

Contextual integrity must also articulate the special place for user
expectations, preferences, and control within the general framing of
appropriate flow. This would require greater fleshing out of situations
where user control is legitimate, given its importance in the sphere of
technical device usage. It would also address questions of how to
resolve conflicts between user preferences and social norms, and
between users with different preferences and expectations.

Finally, CI theorists must clarify the relationship between social
spheres and the law. While there is in the United States an attractive
synergy between the structure of sectoral privacy laws (like HIPAA,
GLBA, and the like) with the view of society differentiated into social
spheres, the relationship between social spheres and the law is less
clear in jurisdictions of omnibus data protection laws such as the
EU's GDPR \cite{regulation2016regulation}.
The CI theorist must
address what circumstances social norms provide important guidelines to
appropriate information flow that are not covered by law, and what
advantages they provide to technology designers who heed them.

\subsubsection{Calls to action}
\label{CI5.3.2}

Computer scientists need not wait for passive instruction on the
normative goals of their work. Rather, the problem of measuring
\textit{social norms}, in contrast to user's
expectations, is one that requires technical sophistication.
\citet{shvartzshnaider2016learning} is one example of a paper that takes this
task on explicitly in service of contextual integrity.

Computer scientists are particularly well situated to study
users' perception and expectation of informational
norms in different social spheres (and not independent of them).
Developing and evaluating techniques to do so remains an open question.
Coming back to Dourish, exploring how far users and different actors
can be brought into engage in the evolution of informational norms is
another avenue of exploration that has not been exhausted by
researchers in our survey. Computer scientists may also consider
designing systems that support communities of users' to
determine their own norms.

Many of the studies did not consider conflicts among actors: these could
be conflicts in informational norms across contexts, in different
situations, even for individual users due to how their expectations
evolve in relation to norms throughout time. Discrepancies between
ideal vs. actual norms may also lead to conflicts. Looking at these
conflicts not as something to be designed away but as productive points
of departure for engagement presents itself as another interesting
research question.

\subsection{Expanding and sharpening contextual integrity}
\label{CI5.4}

This leads to our findings from \textbf{RQ4}, where we look for aspects
of CI that computer science researchers expanded on through their work.
Computer scientists sometimes worked through mechanisms of technical
adaptation to social change as they tried to respect privacy norms that
were grounded in descriptions of concrete social and technical
interaction. 

\subsubsection{Theoretical gaps}
\label{CI5.4.1}

CI theorists must develop the framework's account of
normative change and adaptation. The work we surveyed suggests a need
for technical systems that automatically recognize contexts and that
are sensitive to the norms of their users, even though social contexts
and norms change. What principles can CI offer to adaptive system
designers to ensure that these coevolving sociotechnical systems
maintain their legitimacy with respect to the purposes of some more
abstract social sphere? Do such systems challenge the theory that
social contexts are robust in their ability to maintain their purpose?
On what grounds would such a system be considered maladaptive? Is there
any danger that technology will derail the social processes that
reproduce contexts, or can society always be trusted to correct its own
use of technology over time? What if powerful actors leverage existing
systems with appropriate flows for ends, purposes and values that lack
legitimacy? These thorny theoretical questions are both profound and of
practical import for system design.

CI must also address the critical
``sore'' point in the present-day
when many systems and devices span multiple contexts. Our inquiry here
into the many relevant senses of
`context' sheds light on this
phenomenon. Contexts can clash when the norms of multiple social
spheres applicable to the same situation conflict with each other.
Information can also flow inappropriately between different situations.
A more actionable version of CI will address these complex privacy
challenges specifically.

CI also needs to better account for the relationship between privacy and
time. Some papers in our survey tried to adapt CI to systems in which
data did not only flow, but also was stored, processed, and
deleted.
Current versions of CI do not
recognize that sometimes merely holding data (sometimes for great
durations) can pose privacy threats. We are considering developing a
concept of exposure to characterize this threat. Relatedly, there is a
nuance discovered by \citet{datta2011understanding} that is not observed within
CI: that it may not be apparent whether a case of information flow is
inappropriate at the time that it flows because prescriptions refer to
actions in the future. A more mature version of CI would account for
the conditions under which parties can be aware of their privacy
violations, and how ambiguities can be resolved.

Related to the questions resulting from the ambiguity or incompleteness
of privacy norms are questions concerning the relationship between user
choice and privacy. CI can in principle accommodate a wide range of
preferences and a pluralistic society despite presupposing robust
social agreement on information norms and the nature of social spheres.
Technology is often designed to maximize adoption to diverse users and
consequently can give (or restrict) users' control over
how their data flows. A refinement of CI would address how user control
and user diversity relate to social norms.

\subsubsection{Calls to action}
\label{CI5.4.2}

Computer scientists have already made significant contributions to CI by
providing valuable exemplars of research on adaptation, multiple
contexts, temporality and duration, and user decision making. This work
is invaluable for the evolution of CI.

We see further potential at the intersection of information theoretic
approaches to privacy and contextual integrity. Many of the papers made
use of techniques coming from machine learning, access control, formal
methods, and user surveys, however, while inferences from information
flows were a concern in some papers \cite{omoronyia2012caprice}
\cite{omoronyia2013engineering} \cite{datta2011understanding},
we were missing works that looked at
evaluating or enforcing desired norms using information theoretic
models. It is one thing to have a policy that expresses a norm
that limits
the flow of information about race, gender, class, religion and other
sensitive attributes; it is another to guarantee that this information
cannot be inferred otherwise. One could also imagine novel protections
like differential privacy being used to develop elaborate transmissions
principles. The numerous papers we surveyed demonstrate that computer
scientists have actively applied and contributed to the evolution of
contextual integrity using novel techniques. We hope these results
serve to provide inspiration and guidance to all those researchers who
are committed to leveraging or further developing CI in theory and
practice.
 
\end{document}
