 \documentclass[../thesis.tex]{subfiles}

 \newcommand{\W}{\mathcal{W}} % world
%\newcommand{\S}{\mathcal{S}} % Sensors/Inputs


\def\approxindep{\mathrel{%
    \mathchoice{\APPROXINDEP}{\APPROXINDEP}{\scriptsize\APPROXINDEP}{\tiny\APPROXINDEP}%
}}
\def\APPROXINDEP{{%
    \setbox0\hbox{$\independent$}%
    \rlap{\hbox to \wd0{\hss$\sim$\hss}}\box0
}}



 \begin{document}

% Sebastian Benthall, Cornell Tech \\
%  Anupam Datta, Carnegie Melon University \\
%  Michael Carl Tschantz, ICSI
%}
 
 
\paragraph{Abstract}
Many privacy and data protection policies stipulate
restrictions on the flow of information based on that
information's original source.
We formalize this concept of privacy as Origin Privacy.
This formalization shows how information flow security
can be represented using causal modeling.
Causal modeling of information security leads to
general theorems about the limits of privacy by design
as well as a shared language for representing specific
privacy concepts such as noninterference, differential
privacy, and authorized disclosure.
These considerations raise questions for future work
about whether policies should be design with respect
to the feasibility of automating their enforcement.

%\keywords{information flow security, Bayesian networks, noninterference, differential privacy, causality} 

\section{Introduction}
\label{sec:orgheadline2}

%\mct{The introduction is missing some high-level vision.  Why are you focusing on origins?  What limitation exists in the prior work to require this work?}
% sb: will hold off on revising since in future versions we may drop
%     'origin' angle all together

Many policies (e.g. HIPAA, GLBA, FERPA, and
Executive Order 13526 in the United States and the GDPR
in the European Union) place restrictions on the collection,
flow, and processing of personal information. When engineers
build technical systems that collect and use personal data,
they are under business and social pressure to translate
prescriptive privacy policies, fitted to their case by
lawyers, ethicists, and other policy-makers, into engineering
requirements \cite{barth07csf,fisler2010embracing,swire14iapp,sen14sp}.
The goal of engineering a privacy policy is to enable automated 
enforcement of some or all of the policy.
This reduces the cost of protecting privacy.
To automate enforcement of a privacy policy, that policy must 
first be translated into a machine-readable language with a 
precise syntax and semantics.

Prior research has explored the feasibility of translating
classes of privacy clauses into formal logic for enforcement.
Some attempt to formalize a wide range of policies, such
as those expressible within the framework of Contextual
Integrity \cite{barth07csf,shvartzshnaider2017vaccine}.
Others focus on particular kinds of clauses, such as those
restricting information based on its purpose
\cite{tschantz13esorics} or its use \cite{datta2017use}.
This article is concerned with clauses in privacy policies
that restricting information based on its \emph{origin}.
We consider the origin of data to be the processes that 
created or transmitted that data to the system 
governed by the privacy policy, that is, the data's provenance.

Section~\ref{sec:policy} will show how existing
policies motivate this work
by defining restricted classes of information in terms
of the processes that originated them.
This policy analysis reveals that origin clauses are most
often used to identify a broad class or type of information
that is then subject to restrictions or exemptions.
In addition, information topic
(what the information refers to or is about) is also
frequently used alongside information origin.
We show that the distinction between information origin
and information topic is subtle but important for the
purposes of determining the conditions under which
a formalized policy can be enforced.

In Section~\ref{sec:ontology}, we derive, from the policies
analyzed, a general ontology of systems, processes, and messages.
This ontology is intended to bridge between policies as
intuitively articulated in natural language and the mathematical
formalisms we will use in the rest of the paper.
Using this ontology, we propose Origin Privacy as a
framework for understanding
privacy requirements and the knowledge necessary to enforce
them by design.

Section~\ref{sec:causality} shows how this informal
specification can be mapped to a well established
formal representation
of causal models~\cite{pearl1988probabilistic}.
In addition to presenting the formal theory,
we show that causal modeling makes clear distinctions
between two elements of information flow that are
sometimes conflated: causal flow and nomic association,
where ``nomic'' means law-like, or regular.
These two aspects of information flow correspond
to information origin and information topic.

In Section~\ref{sec:security}, we combine the
ontology with causal modeling to develop
the Embedded Causal System (ECS) model.
This model represents a computational system
embedded in a larger environment.
This is motivated by the need to consider technical
systems in their environments (possibly interacting
with third parties) when assessing their privacy,
fairness, and security properties~\cite{veale2017fairer}.
We show demonstrate conditions under which an ECS model
is secure according to the well-established formal security model
of noninterference \cite{gm82security}.
We also formalize semantic security for an ECS model
and reproduce the result that it is, in general, impossible
for a system designer to guarantee semantic security
on a statistical database given auxiliary knowledge.
It is well known that information
can be revealing of other sensitive information given
auxiliary knowledge. Our theorem reflects the conditions
under which auxiliary knowledge is possible.
The contibutions from the model are due to explicit
causal modeling of the generative process of data
input into the system as well as the operations
of the system itself.

In Section~\ref{sec:robustness}, we build on these results
to develop security models for cases where information
is restricted based on its origin.
We find these models analogous to noninterference
and semantic security, and demonstrate sufficient
conditions under which an ECS has these properties.

Section~\ref{sec:usecase} shows a case study of using Origin privacy.
We show that in a case of biometric sensing with
Internet of Things devices, origin privacy specification
can be used to enforce GDPR compliance.
Section~\ref{sec:differential} shows how differential privacy
is a special case of origin privacy.
Section \ref{sec:incentives} demonstrates how a game theoretic
layer can be added to the ECS model to show how system
security properties relate to the impact on system users.
Section~\ref{sec:future} addresses directions for future work.


%\mct{Is there no prior work section?
%I think the noninterference composition results should be covered.}
% sb: Good point. Let's revisit this when we are preparing for publication.
%


\textbf{Contributions}. The contributions of this paper
include:
\begin{itemize}
\item An analysis of privacy policies, specifically with
  respect to how they determine protected classes of information
  through information topic and information origin.
\item An informal ontology and articulation of Origin Privacy
  appropriate for use by policy designers.
\item Disambiguation of the concept of ``information flow''
  into causal flow and nomic association components through
  causal modeling.
\item The Embedded Causal System (ECS) model, a model of a
  causal system embedded in its environment suitable for
  proving properties of information flow security under these
  conditions.
\item Proofs of conditions for noninterference and semantic security
  in causal and embedded causal systems.
\item Formal security models for origin noninterference and origin
  semantic security, with proofs of sufficient conditions.
\item Demonstration of the use of Origin Privacy in a
  biometric Internet of Things use case.
\item Relaxed security models based on mutual information and
  proofs of their formal relationship to differential privacy.
\item A demonstration of the use of ECS models in game theoretic
  modeling using Multi-Agent Influence Diagrams.
\end{itemize}

This paper includes two technical appendices.
Appendix \ref{appendix:information-theory-theorems}
proves several supplemental theorems in information theory
that are used in a proof of the relationship between
causal modeling and differential privacy.
Appendix \ref{appendix:maid} outlines the major
findings of \cite{koller2003multi} on Multi-Agent
Influence Diagrams, and introduces a new concept:
tactical independence.

\section{Policy Motivations for Origin Privacy}
\label{sec:policy}

To motivate a consideration of Origin Privacy as a flavor
of privacy specification, we look to existing policies that
include rules that restrict information flow based on its origin.
We build on prior work in logical specification of privacy
laws as inspiration for this 
approach~\cite{barth06sp,DeYoungGJKD10}.

\subsection{Policy Example: HIPAA Psychotherapy Notes}
\label{sec:orgheadline3}

Some laws define a class of information in terms
of the process that creates it.
A straightforward example from law are psychotherapy notes
as defined under HIPAA\footnote{45 CFR \S 164.501.}:

\begin{quote}
Psychotherapy notes means notes recorded (in any medium) by a health care provider 
who is a mental health professional documenting or analyzing the contents of 
conversation during a private counseling session or a group, joint, or family 
counseling session and that are separated from the rest of the individual's 
medical record. Psychotherapy notes excludes medication prescription and 
monitoring, counseling session start and stop times, the modalities and 
frequencies of treatment furnished, results of clinical tests, and any 
summary of the following items: Diagnosis, functional status, 
the treatment plan, symptoms, prognosis, and progress to date.
\end{quote}

In this definition, there is a reference to a process involving 
``documenting or analyzing [\ldots] a [\ldots] counseling session''.
Any information with provenance beginning with its creation by a health 
care provider who is a mental health professional documenting a conversation 
during a counseling session separately from an individual's medical record, 
barring some exceptions, are psychotherapy notes.

\subsection{Policy Example: GLBA}
\label{sec:orgheadline4}

Some laws include references to information origin in the 
definition of what information is protected. 
We find this in particular in the Privacy Rule of 
the Gramm--Leach--Bliley Act, which applies to 
``nonpublic personal information'' (NPI)\footnote{GLBA, 15 U.S. Code \S 6809}. 
This class of information is defined as personally identifiable
financial information that is
\begin{enumerate}
\item provided by a consumer to a financial institution;
\item resulting from any transaction with the consumer
  or any service performed for the consumer; or
\item otherwise obtained by the financial institution.
\end{enumerate}

Reading progressively through each of these
criteria, the concept of \emph{origin} 
helps us understand the differences between
them and how that affect their enforceability.
% \mct{Before getting to the criteria, there's general condition that
% the information must be ``personally identifiable financial
%  information''.  Maybe that should be discussed first.}
Criterion~1 explicitly refers to the channel
of transmission and does not refer 
to any specific meaning or category of
the information transmitted 
(though examples are provided in the law, these are
constrained only by what is normally 
transmitted in the process of
procuring a financial service).
It sets clear guidelines as to the
process of creation and transmission.
Criterion~2 refers to broad classes of
ways in which the information is transmitted 
to the governed entity.
Criterion~3 is written as if to cover all other 
cases where a financial institution could 
collect individualized information, 
regardless of the process of transmission. 
It is agnostic to process of transmission. 
It raises the question of whether information can be
personal financial information without having the specific
provenance of a transaction or service with a
financial institution.
For example, information about a person's income may be
personal financial information no matter how it is discovered.

%\mct{It appears that Criterion~3 is a superset of 1 and 2.  So, why
%  are 1 and 2 even stated?  In fact, Criterion~3 doesn't even seem to
%  be a limitation.  Why not just say it's all ``personally
%  identifiable financial information''?}
% sb: I don't know. Do you think we need to address this?
%

\subsection{Policy example: PCI DSS}
\label{sec:pci-dss}

Though not a law, we consider the Payment Card Industry 
Data Security Standard (PCI DSS) as an exemplary 
security regulation~\cite{bradleypayment}.
Established as a proprietary information security standard 
by the Payment Card Industry Security Standards Council, 
it applies to organizations that use
major branded credit cards such as Visa, Mastercard, 
and American Express.
Though referenced in the laws of some U.S. states, it is 
mandated mainly by Visa and Mastercard themselves through 
their dealings with 
merchants\footnote{Minnesota Session Laws - 
CHAPTER 108--H.F.No. 1758. Nevada Revised Statutes, Chap. 
603A \S 215. Wash. Rev. Code \S 19.255.020 (2011). 
See~\cite{pcisecuritystandardscouncilFAQ}}.

We note two aspects of the PCI DSS that make it an effective 
regulatory standard.
First, the PCI DSS governs only types of data the enforcing 
industry is responsible for generating: cardholder data and 
sensitive authentication data~\cite{pcisecuritystandardscouncil2016DSS}.
This data is generally not meaningful outside of the operations 
that the payment card industry enables, because the potential 
uses of the data are a function of the system that created 
the data in the first place.
This allows the payment card industry to 
straightforwardly enforce contractual obligations on those 
that use the data.
This is in contrast with legal regimes that regulate the 
flow of more generally meaningful information between persons 
and organizations.

A second feature of PCI DSS is that they are explicit about 
the application of the standard to networks of technology and 
business processes,
which they call the cardholder data environment 
(CDE) \cite{pcisecuritystandardscouncil2016DSS}:
\begin{quote}
  The PCI DSS security requirements apply to all system components 
included in or connected to the cardholder data environment.
  The cardholder data environment (CDE) is comprised of people, 
processes and technologies that store, process, or transmit 
cardholder data or sensitive authentication data.
\end{quote}
which PCI DSS further defines as 
\cite{pcisecuritystandardscouncil2016DSS}:
\begin{quote}
  The first step of a PCI DSS assessment is to accurately 
determine the scope of the review.
  At least annually and prior to the annual assessment, 
the assessed entity should confirm the accuracy of their 
PCI DSS scope by identifying all locations and flows of 
cardholder data, and identify all systems that are connected 
to or, if compromised, could impact the CDE (for example, 
authentication servers) to ensure they are included in the
  PCI DSS scope.
  All types of systems and locations should be considered as 
part of the scoping process, including backup/recovery sites 
and fail-over systems.
\end{quote}

This is a clear example of how a privacy policy can be specific
about the technical system to which it applies.

\subsection{Other policies}

The examples above have been chosen for their representativeness
of the concepts developed in this paper.
Other information protection policies do not define protected
information solely in terms of its origin but rather depend
in whole or in part on
a definition of information topic. For example,
the Family Educational Rights and Privacy Act (FERPA)
defines ``education records'' as
\begin{quote}
  those records, files, documents, and other materials which—
  
  (i) contain information directly related to a student; and

  (ii) are maintained by an educational agency or institution or by a person acting for such agency or institution.
\end{quote}

The Children's Online Privacy Protection Rule (COPPA)
includes a broad definition of ``personal informal''
as any individually identifiable information collected
on-line, but includes special restrictions on personal
information collected from a child, which could be
read as a restriction based on origin.

The United States Executive Order 13526 of 2009,
gives certain government officials authority to classify
documents they determine pose a risk to national
security.
In some cases, the information may be classified
as soon as it is produced.
In all cases, the information's classification
prevents unauthorized access.
Derivative information carries the classification
of the original information.
Information may be declassified when the
conditions for classification no longer hold.
Information restrictions on information
therefore depend partly on its procedural history,
but also on predictions made about the effects of
disclosure.

Our analysis of Origin Privacy explores the
limits of privacy by design and what policies
can be automatically enforced on a system
bound by laws of statistical causation.
We will demonstrate the ambiguity of
the term ``information'' and how this
renders the application of many policies
that restrict information based on information
topic indeterminate.


\section{Origin Privacy}
\label{sec:ontology}

We will now provide an informal definition of Origin Privacy.
First, we will introduce an ontology appropriate to the
design of technical systems and inspired by the policy
examples above.
Three concepts are introduced in this section.
\emph{Systems} are assemblages of people and technologies
through which information flows and is transformed.
\emph{Processes} are regular events, implemented as
a person or technology's behavior, which act on
information.
Processes pass information to other processes as data.
The \emph{origin} of data is the history of processes
that led to its creation.
Origin privacy includes any privacy restrictions made
on information flow conditioned on that information's
origin.

\subsection{Systems}

Regulations mark out spaces or domains in which information 
may flow more freely than others, or where restrictions 
specifically apply.
For example, HIPAA applies to ``covered entities'', including 
health care providers\footnote{``Covered entity means: 
(1) A health plan. (2) A health care clearinghouse. 
(3) A health care provider who transmits any health 
information in electronic form in connection with a 
transaction covered by this subchapter. 45 CFR \S 160.103}.
These spaces may be defined by legal jurisdiction, or they 
may be defined through networks of contractual relations.
More often than not, the regulated space is not bounded 
geographically but rather by relationships between 
personnel, institutions, and technical systems,
all of which are participating in information flows 
and processing.

The Payment Card Industry Data Security Standard (PCI DSS) 
is explicit about this in its definition of the covered system 
in terms of the
``people, processes, and technologies that store, process, or
transmit'' data (Section \ref{sec:pci-dss} provides further details) \cite{pcisecuritystandardscouncil2016DSS}.
We generalize this concept and
refer to assemblages of people, processes, and technologies 
such as these \emph{information processing systems}, or 
just \emph{systems}.

There will inevitably be people, processes, and technologies
that interact with a governed system without being included
within it. We refer to these external factors as the
\emph{environment} of the system.
Systems have inputs and outputs, which are events
that pass data from and to the environment.
We will find it useful to model the world of a system
within its environment as a larger, superordinate system.
This causally embedded system framework is formalized
in Section \ref{sec:ecs}.

The use of the term ``system'' here is abstract and intended
to provide a precise analytic frame.
How it gets applied to an empirical case can vary.
For example, in healthcare, we might consider a single hospital
with its procedures as a system, or a network of covered entities
collectively as a system.
A complete discussion of systems and how they may be nested or
interconnected is beyond the scope of this paper.

\subsection{Processes}

For the purpose of these definitions, we will assume that
technical systems and their environments consist of many
different components implementing information processes.
People in their professional roles are, for the purposes of 
this framework, included as another way of implementing processes.

The outputs of a process may be the input to another.
These messages between processes are \emph{data}.
When a process $A$ sends data to another process $B$ under one or
more conditions, we say that there is a channel from $A$ and $B$,
or, equivalently, that (the state of) $A$ directly causes
(the state of) $B$.

The structure of processes and their dependence on each other
implies a causal graph \cite{pearl1988probabilistic} with directed
edges representing the channels between processes.
We will discuss the implications of causal modeling of systems
more thoroughly in Section \ref{sec:causality}.

%
% NOTE: Contiguity is mentioned here, but not formally defined
% in the mathematical section of this paper.
%

Systems are composed of contiguous processes,
meaning that for any two processes in the system there will
be an (undirected) path between the two through channels that
includes only other processes in the system.

While causal modeling has been used in a security context
(e.g.,~\cite{feng2014security}), formal security research
more often relies on static analysis of programs (e.g.,~\cite{mclean90sp}).
We choose a causal models in this article because
these models can represent both programs and their
subprograms (see Section \ref{sec:programs}),
as well as non-technical environmental processes
that generate data.\footnote{While it may be the case that
  programs are as expressive as causal models, this
discussion is beyond the scope of this paper.}
An example of such a process is a medical examination
conducted via an interpersonal interaction between
patient and doctor.

\subsection{Origin and Provenance}

The data resulting from a process depends causally
on a history of prior processes.
Sometimes data that is an input to a system
is generated by a process that is not immediately part
of the system.
Data can flow through 
a series of relays before it reaches
the system on which a privacy policy is being enforced.
The entire history of the information as it flows from its 
creation to the system input is the information's 
\emph{provenance}.
The governed system may or may not have access to
assured metadata (such as cryptographic certificates)
about the provenance of its data.

\iffalse
## Commenting out for PoPETS
See Figure \ref{fig:provenance}.

\begin{figure*}
\noindent
Data's Provenance:
\begin{center}
\footnotesize
\begin{tikzcd}[cramped, sep=small, math mode=false]
   \begin{tabular}{ll}data creator\\ (ultimate source)\end{tabular} \arrow[r]
 & \begin{tabular}{ll}intermediate\\ system $A$\end{tabular} \arrow[r]
 & \begin{tabular}{ll}intermediate\\ system $B$\end{tabular} \arrow[r]
 & \begin{tabular}{ll}transmitting system\\ (immediate source)\end{tabular} \arrow[r]
 & \begin{tabular}{ll}governed\\ system \end{tabular}\arrow[r, dotted]
 & \begin{tabular}{ll}potential\\ receipt\end{tabular}\\
\end{tikzcd}
\end{center}
\noindent
Data's Known Provenance:
\begin{center}
\footnotesize
\begin{tikzcd}[cramped, sep=small, math mode=false]
   \phantom{\begin{tabular}{ll}data creator\\ (ultimate source)\end{tabular}} \arrow[r]
 & \begin{tabular}{ll}intermediate\\ system $A$\end{tabular} \arrow[r]
 & \phantom{\begin{tabular}{ll}intermediate\\ system $B$\end{tabular}} \arrow[r]
 & \begin{tabular}{ll}transmitting system\\ (immediate source)\end{tabular} \arrow[r]
 & \begin{tabular}{ll}governed\\ system \end{tabular}\arrow[r, dotted]
 & \begin{tabular}{ll}potential\\ receipt\end{tabular}\\
\end{tikzcd}
\end{center}
\caption{Data's Origin.  The data is created by some system and makes it way to the governed system, subject to the privacy policy, by way of intermediate entities and finally the transmitting system.  The governed system must consult the privacy policy to determine whether it may further transmit the data to some potential receipt.  In some cases, the governed system might not have a complete view of the data's provenance, knowing only a subset of the other entities involved in the data's transmission to it.}
\label{fig:provenance}
\end{figure*}
\fi

We consider the origin of data to be the processes
that have caused it, either directly or indirectly.
For the purposes of enforcement of a policy, these processes
may be either in the governed system or outside of it,
in an originating system or the system's environment.

\subsection{Origin privacy}

Given the above ontology, we can now provide a definition
of origin privacy.
Origin privacy includes any and only those information
flow restrictions implemented in a system that are conditioned
on the provenance of system inputs.
 
\section{Information Flow and Causal Models}
\label{sec:causality}

We have motivated Origin Privacy as a concept of privacy
that is useful when considering how to design
information processing systems to be compliant with
laws and other rules regarding the flow of personal 
information.
The ontology in Section~\ref{sec:ontology} is motivated
by policies that specifically mention systems and
restrict information flow based on its origin.

In this section we will introduce philosophical and formal
concepts with which we will make our definitions and claims
about origin privacy precise.
Philosophically, privacy depends on appropriate
information flow~\cite{nissenbaum09book}, where information is defined
as that which allows somebody to learn about something else
based on its regular associations with it.
We find a formalization of this idea in Bayesian networks,
a common formalism in statistics which represents the
relationships between random variables with a directed
acyclic graph.
Bayesian networks have two attractive properties which
we will explain.
First, it is easy to derive some independence relations
between variables from the graph structure of a Bayesian
network.
Second, this formalism supports an intervention operation
that gives it robust causal semantics.
All of these conceptual tools will be used in proofs
later in this paper.
We will close this section by showing how this formalism
rigorously clarifies an ambiguity in the term `information flow',
which refers to both causal flow and nomic associations between
variables.

\subsection{Philosophy: Contextual Integrity and Information Flow}

Origin Privacy is intended to be broadly consistent
with the Contextual Integrity \cite{nissenbaum09book}
philosophy of privacy in so far as it defines privacy
as \emph{appropriate information flow}.
Specifically, Contextual Integrity maintains that
privacy expectations can be characterized by norms
about how information about persons flows in particular
social contexts, or spheres.

In this article, we restrict our analysis of Origin Privacy
to cases where expectations of privacy have been articulated
as \emph{policies} in natural language and endowed with a 
social system of enforcement.
We consider laws and contracts as kinds of policies.
Policies may or may not express social norms as per
Contextual Integrity; addressing the conditions under
which policies reflect social norms is beyond the scope
of this article.
However, we maintain that some policies are specifically
\emph{privacy} policies because they, like privacy norms,
prescribe personal information flows. 

As a way of bridging from Contextual Integrity through
privacy policies to the specification of privacy-preserving
mechanisms, we address \emph{information flows} in general.
Despite its wide use, the phrase ``information flow''
is rarely given a precise definition.
However, philosophical and formal work on information flows have
provided general insights that can bridge between social, legal,
and technical theories of privacy.
We will build on these insights to make
arguments about origin privacy.

There is a long history of literature on information flow
in computer security and privacy 
research \cite{mclean90sp,gray91sp,barthe04csf,tschantz15csf,smith15lics}.
%\cite{csftschantz13blackboxtr} \cite{tschantz14arxiv}
These take their inspiration from Shannon's classic formulation 
of information theory~\cite{shannon1948mathematical}.
Dretske's philosophical
formulation of information flow~\cite{dretske1983epistemology}
also draws on Shannon's information theory.
In this work, we are explicitly bridging between philosophy and engineering
principles, finding common ground between.

According to Dretske's theory, a message
carries information about some phenomenon if a suitably equipped
observer could learn about the phenomenon from the message.
In other words, a message carries information about anything that
can be learned from it.
For an observer to learn from it, the message must have
a \emph{nomic} connection with its subject, where here "nomic" means
``law-like'' or ``regular''~\cite{dretske1981knowledge}.
%\mct{How does this differ from the information as mere association view?}
% sb: Do you have a citation for this view that I could use
%     in order to make the comparison?
Messages, in this understanding, get their meaning from the
processes that generate them, because these processes connect
the content of messages reliably to other events.
There is a logical connection between the definition of information
flow and the structure of the regular dependence between events.
The formal theory of the causal dependence between events has been
worked out in the literature on causal graphical 
models \cite{pearl1988probabilistic}.

\subsection{Causal Probabilistic Graphical Models}
\label{sec:orgheadline18}

There is the a well known formalism for representing
the causal relationship between uncertain events: the
\emph{Bayesian network}, or probabilistic 
graphical model, framework \cite{pearl1988probabilistic}. 
As we will see, this form of modeling can be used to represent
processes within a system, as well as in its an environment.
Before showing the relationship between Bayesian networks
and Origin Privacy, we will present them and a few of their
formal properties, drawing heavily on \cite{koller2003multi}
for our choice of formal notation and wording.

\subsubsection{Bayesian networks}
\label{sec:orgheadline16}

A Bayesian network represents the joint probability distribution
of a set of random variables with a graph. Consider variables
\(X_1, ..., X_n\) where each \(X_i\) takes on values in some set
\(dom(X_i)\). We use \(\mathcal{X}\) to refer to the set \(X_1, ..., X_n\)
and \(dom(\mathcal{X})\) to refer to their joint domain.

A Bayesian network (BN) represents the distribution using a graph
whose nodes represent the random variables and whose edges represent
direct influence of one variable on another.

\begin{dfn}[Bayesian network]
A Bayesian network over variables $\mathcal{X} = X_1, ..., X_n$ is 
a pair (G,Pr). G is a directed acyclic graph with $n$ nodes, 
each labeled for one of the variables in $\mathcal{X}$. We use
$Pa(X)$ to denote the parents of $X$ in the graph. Pr is a mapping
of each node $X$ to a conditional probability distribution (CPD),
$Pr(X \vert Pa(X))$.
\end{dfn}

\begin{exm}
  Alice will be on time for work $W$ if she sets her
  alarm $A$ early enough and traffic $T$ allows.
  Bad traffic can be caused by construction $C$
  or an accident on the road $D$.
\end{exm}
\begin{center}
\begin{tikzcd}
  A \arrow[rrd] & & \\
  C \arrow[r] & T \arrow[r] & W \\
  D \arrow[ur] & & 
\end{tikzcd}
\end{center}

Given the discussion of information flows above,
we can see why this
is relevant to privacy. The condition dependence
functions between
random variables are the \emph{nomic relations}
between events and messages.
If two variables are conditionally dependent on
each other, and this
conditional dependence is known to the observer of one
of the variables,
then the observer can infer something (have knowledge of)
the other
variables. Hence, by our definitions, the variables carry
information about each other.
If privacy is appropriate information flow, then
the privacy of a system will depend on the
causal relationships between its components
and the environment.

A directed edge between one variable and another
indicates a possible
conditional dependence between them.
Strictly speaking, it does not
necessitate that there is a conditional dependence between them,
it only necessitates that there is a conditional
probability distribution
function defined between them.
But it does guarantee that at least one such conditional
probability distribution does exist, and under reasonable
conditions \emph{most} possible functions
(in a measure-theoretic sense) will exhibit
the conditional independence~\cite{meek1995strong}.
As functions ensuring
independence are quite rare in the space of all possible
conditional probability functions, this quirk in the
notation has not prevented
this formalism from being useful in identifying
independence in practice.

%%% Seems to be repeated below
% One property of causal models is that the
% conditional independence of its variables can
% to some extent be read off of the model's
% graph structure.

\subsubsection{D-separatedness}

A useful property of probabilistic graphical models
is that some aspects of the joint probability
distribution of all variables represented in the graph
can be read easily from the graph's structure.
Of particular interest in the analysis of the joint probability
distribution is when and under what conditions
two random variables are independent.

\begin{dfn}[Path]
A \emph{path} between two nodes \(X_1\) and \(X_2\)  in a graph 
to be a sequence of nodes starting with \(X_1\) and ending with \(X_2\)
such that successive nodes are connected by an edge (traversing
in either direction).
\end{dfn}

\begin{dfn}[Head-to-tail, tail-to-tail, head-to-head]
For any three nodes (\(A, B,C\)) in succession on a
path, they may be \emph{head-to-tail} 
(\(A \rightarrow B \rightarrow C\) or \(A \leftarrow B \leftarrow C\)), 
\emph{tail-to-tail} (\(A \leftarrow B \rightarrow C\)), or \emph{head-to-head}
(\(A \rightarrow B \leftarrow C\)).
\end{dfn}

We will find it useful to refer to a special kind of paths, \emph{direct paths}.

\begin{dfn}[Direct path]
  A \emph{direct path} from \(X_1\) to \(X_2\)
  is a path starting with \(X_1\) and ending with \(X_2\)
  such that all triples are head-to-tail.
\end{dfn}

\begin{dfn}[Ancestors and descendants]
  If there is a direct path from \(X_1\) to \(X_2\),
  then $X_1$ is an ancestor of $X_2$ and $X_2$ is a descendant
  of $X_1$.

  Let $descendants(X)$ be the set of descendants of $X$.
\end{dfn}


There are two ways in which a variable
\(A\) can be conditionally dependent on another variable \(B\) without
one of them being a descendant of the other.
The variables may share an unobserved common cause
or they may share an observed common effect.

\begin{exm}
  One building in a neighborhood loses power, $B_1$.
  One can guess that other buildings $B_i$ around nearby lost power,
  because power in each building is dependent on the electric grid
  $G$. All the buildings may be affected by the common cause of a grid
  failure.
\end{exm}
\begin{center}
\begin{tikzcd}
   & B_1 \\
  G \arrow[ur] \arrow[r] \arrow[dr]& B_2 \\
   & B_3& 
\end{tikzcd}
\end{center}

\begin{exm}
  Suppose we observe that Alice is late for work $W$, as per our
  earlier example. This could be due to many reasons,
  including traffic $T$ and missing her alarm $A$.
  Traffic may be due to construction $C$ or an accident $D$.
  The probability of any particular cause is conditionally
  dependent on the others, because if any one cause is
  ruled out, the others are more likely.
\end{exm}

The existence of a path between two nodes is
necessary for their probabilistic dependence on each other.
It is not sufficient, particularly when considering
their dependence \emph{conditional on other variables}.
For this reason, paths in a Bayesian network can be
blocked or unblocked based on a set of variables that is
otherwise known or observed, the \emph{conditioning set}.


\begin{dfn}[Blocked path]
A path is considered to be \emph{blocked} if either:
\begin{itemize}
\item it includes a node that is in the conditioning set \(C\) 
where the arrows point to it do not meet head-to-head, or
\item it includes a node where arrows do meet head to head, but
neither this node nor any of its descendants is in the
conditioning set
\end{itemize}
\end{dfn}

\begin{dfn}[D-separation]
If every path from \(X_1\) to \(X_2\) given conditioning set \(C\)
is blocked, then \(X_1\) and \(X_2\) are d-separated.
\end{dfn}

\begin{thm}\label{thm:d-seperated}
  If \(X_1\) and \(X_2\) are d-separated conditioned on set \(C\),
  then $X_1 \independent X_2 \vert C$.
\end{thm}

\begin{proof}
  Proof is discussed in \cite{koller2003multi}.
\end{proof}
\hfill
\hfill

The converse (that independence implies d-separatedness) is
not true in general because specific conditional distribution
functions can imply independence.
Similarly, it is not generally
true that the absence of d-separatedness implies conditional
dependence.
However, is has been shown that conditional distribution functions
implying conditional independence are rare
in a measure-theoretic sense
\cite{geiger1990logical,meek1995strong,koller2003multi}.
%When there is uncertainty about which of a wide range of functions
%characterizes the probability distribution over many variables,
%absence of d-separation is strong evidence of the probability of
%independence.
%\mct{This is hard to follow.}

\subsubsection{Intervention}

We have used the terms \emph{Bayesian network} and
\emph{causal model} interchangeably.
This is because Bayesian networks support a causal
interpretation through one additional construct,
\emph{intervention} \cite{pearl1993bayesian}.
An intervention on a Bayesian network sets
the values of one or more of its values.
Unlike an observation of a variable,
an intervention effectively creates a new graphical
model that cuts off the
influence of a set variable on its parents and vice versa.
Descendants of the set variable are affected by
the intervention according to the probability
distribution of the original model.

\begin{dfn}[Intervention]
  An \emph{atomic intervention} setting variable
  $X_i$ to $x'_i$ on a Bayesian network $\W$ creates a new
  network $\W'$ with post-intervention probability
  distribution $Pr_{x'_i}$

  $$Pr_{x'_i}(X_1,X_2,...,X_n) = \begin{cases}
               \frac{Pr(X_1,X_2,...,X_n)}{Pr(X_i = x'_i \vert Pa(X_i))} & \text{if} X_i = x'_i \\
               0, & \text{otherwise}\\
  \end{cases}$$
\end{dfn}

Theories of causation based on manipulation and intervention
have been influential in philosophy~\cite{woodward2005making}
and have been shown to be effective theories in
psychology of causation~\cite{sloman2005causal}
including the role of causation in moral 
reasoning~\cite{sloman2009causal}, suggesting interventionist
causation as a potential bridge between computer science
and ethical domains such as privacy and fairness.
Cowgill and Tucker~\cite{cowgillalgorithmic} 
discuss the evaluation
of algorithmic impact using counterfactuals, which
draws on a different but compatible theory of
causality~\cite{rubin1974estimating}.

\subsection{Ambiguity of information flow}
\label{sec:ambiguity}

We have drawn a connection
between information flow in the philosophical sense
relevant to Contextual Integrity and Bayesian networks.
A Bayes network is a way of representing the nomic dependencies
between phenomena.
They are ``nomic'' because they describe probability 
distributions that generalize over particular instances of a 
system's functioning. These nomic relations
are factored out as an explicit structure of causal relationships.

This reveals an ambiguity in the very concept of \emph{information flow},
illustrated in the following example.

\begin{exm}
Alice, a teacher tells every student
privately their test score's rank $R$ (first in class,
second in class, etc.) after every test, with class
participation used as a tie-breaker.
Alice sends a message $B$ to Bob with the information that
he has the second highest rank in the class.
Alice also sends a message $E$ to Eve that she has
the highest rank in the class.
From her message and knowledge of the test environment,
Eve learns from her message that Bob was told that he
was, at best, second in class.
Did information about Bob flow to Eve?
\end{exm}

A formal representation of this example makes the
root of the ambiguity clear.
Consider a three node Bayesian network where $R$ is the test
results, $B$ is the message sent to Bob, and $E$ is
the message sent to Eve:
\begin{center}
\begin{tikzcd}[row sep=tiny]
   & B \\
  R \arrow[ur] \arrow[dr]&  \\
   & E
\end{tikzcd}
\end{center}
There is \textbf{causal flow} along the edges from
$R$ to $B$ and from $R$ to $E$.
But an observer of a single variable aware of the
system's laws (nomic connections,
graphical structure) can learn \textbf{nomic associations}
of a message that inform about variables that
are not in the message's causal history.
Despite $E$ neither causing nor being caused by $B$,
$E$ reveals information about $B$.

The phrase ``information flow'' is ambiguous because the
word ``information'' is ambiguous \cite{nunberg1996farewell}:
it can refer to both a message and the contents of a message.
We do not favor either sense.
Rather, we propose that to resolve this ambiguity,
one has to recognize how the systematic creation and
transfer of messages--represented in general by a graph
of causal flows--gives each message its meaningful
contents.
In our formalism, an information flow is a causal flow
that, by virtue of its place in a larger causal structure,
has nomic associations.

Our analysis of privacy policies shows how
they variously restrict the flow
of information based on its contents as well as its
causal history or origin.
This is consistent with our analysis of ``information flow''
as refering to one causal flow within a larger system
of causes that give it contents.
This scientific formulation of information flows is not
yet native to the the language of the law.
That the law refers variously to aspects of information flow
based on contents and causal history reflects how
both are essential to the meaning of the term.

Following on this account of information flow,
in the following sections we will precisely model a system
in its environment
in order to disambiguate the different
aspects of information flow and understand the conditions
under which a system can be free of information leaks.
We will measure the strength of
nomic associations using a well-understood measure,
\emph{mutual information}.
Mutual information 
captures how much about one random variable can be
learned from observing another.

\begin{dfn}[Mutual information]
  The mutual information of two discrete random variables
  $X$ and $Y$ is

  $$I(X,Y) = \sum_{x \in X} \sum_{y \in Y} p(x,y) log \frac{p(x,y)}{p(x)p(y)}$$
\end{dfn}

In particular, $I(X,Y) = 0$ if and only if $X \independent Y$.

%space
Mutual information is a technical term with a specific mathematical
meaning.
It is no etymological accident that it forms part of the analytic
definition of ``information flow'' that we have developed in this
section.
In the Appendix, we derive several theorems relating to
the mutual information of variables organized in a Bayesian
Network.
We will use these theorems in proofs about system security
in the following sections.

\section{Bayesian networks and information flow security}
\label{sec:security}

In this section, we formalize the ontology from
Section~\ref{sec:ontology} that we derived from
privacy policies.
Our formalization uses the causal graphical modeling
tools outlined in Section~\ref{sec:causality}.
We show that several known results in information flow
security have dual results within our formal causal modeling
of systems in their environments.
We demonstrate this for concepts of 
noninterference~\cite{gm82security} and the impossibility
of guaranteeing secrecy given the possibility of
arbitrary auxiliary knowledge~\cite{dwork06icalp,dwork08jpc}.

Where our analysis goes beyond these known results
in information flow security are that our models
explicitly take into account the relationship between
a technical system and its environment.
In each case, our theorems prove conditions of the
security properties that could not be discovered
by static program analysis in isolation.
For example, it is well known that information
can be revealing of other sensitive information given
auxiliary knowledge. Our theorem reflects the conditions
under which auxiliary knowledge is possible.

Because the CPD defined by \(Pr\) between each random
variable and its
parents can be an arbitrary function, including deterministic
logical operations, it is possible to encode a system of
computational components, including sensors, data processors,
and actuators, as a BN.
Earlier we defined Origin Privacy in terms of systems, processes,
and messages.
These concepts map easily into this formalism: systems are
Bayesian networks; processes are random variables or events;
the inputs and outputs of processes are determined by links
connecting them to other processes; messages are
the instantiation 
of particular random variables,
which are available as inputs to
later variables.

%\mct{I think key points of this section is the following:\\
%CS people have been using noninterference, which models systems as a computation.  This model is limited in what it can say since it doesn't have a model of the environment in which the system operates.  We will instead models systems as ECSes, expressed with causal graphs that can extend out to model the environment.  ECSes have a property that implies noninterference.  To do this, we have cut out part of the ECS to model the system found in noninterference, leaving the rest as the environment.  We present a property that these cut-out graphical sub-models corresponding to systems can have that implies the computation modeling the same system having noninterference.
%\\ 
%I'm I right?  If so, I'm missing the part where you show that your model is the same as the old model.
%}
%
%sb: I'm not sure what you're getting at, but it sounds important.
%We should discuss before publication.
%

\subsection{Embedded Causal System (ECS) Model}
\label{sec:ecs}

In the subsequent sections we will use the following standard
notation and model, which we will refer to as the
Embedded Causal System (ECS) Model.

\begin{dfn}[World]
  A \emph{world} $\W$ is a set of random variables
  with a conditional
  probability distribution that may be modeled as a
  Bayesian network $(G_\W,Pr_\W)$.
\end{dfn}
\begin{center}
\begin{tikzcd}
   \W \\
\end{tikzcd}
\end{center}

%
%\begin{dfn}[Contiguous]
%  Given a Bayesian network $(G,Pr)$ over $\mathcal{X}$,
%  a subset $\Y \subset \mathcal{X}$ is \emph{contiguous}
%  if for all $Y_1, Y_2 \in \Y$, there is a path between
%  $Y_1, Y_2$ such that all nodes on the path are members
%  of $\Y$.
%\end{dfn}
%

\begin{dfn}[System]
  A subset of the world $\Y \subset \W$ is the \emph{system}.
\end{dfn}

%% A system must be contiguous.

\begin{dfn}[Environment]
  The \emph{environment} of $\Y$ is the set of 
  nodes in the world that are not in the system,
  $\mathcal{E} = \W - \Y$ 
\end{dfn}
\begin{center}
\begin{tikzcd}
  \Y \arrow[d, bend left=50]\\
  \mathcal{E} \arrow[u, bend left=50]
\end{tikzcd}
\end{center}

(In this and a few other diagrams, we will include cycles because
these are diagrams of blockmodels over other networks.
In a blockmodel of $G$, a partition $\{P_1, P_2, \ldots\}$ of the set
of original nodes $\mathcal{X}$ is treated as a new set of nodes,
with an edge between $P_1$ and $P_2$ iff
there exists $X_1$ and $X_2$ such that $X_1$ is in $P_1$, $X_2$ is in $P_2$, and $(X_1,X_2)$ is in $G_{\msf{edges}}$)

It is common in security research to consider \emph{systems}
as units of analysis; these systems contain \emph{programs}; system
input is data and system output is the result of the programs
operating on the data~\cite{mclean90sp}.
These programs are represented using a formal approximation of a
programming language
in order to prove security properties of systems.
Systems in our formalism also have inputs and outputs,
which are defined by their position relative
to the environment.

\begin{dfn}[Sensors and inputs]
  A \emph{sensor} is an edge $(A,B) \in G_\W$
  such that $A \in \mc{E}$ and $B \in \Y$,
  An \emph{input} is the head node of a sensor, $B$.
  Denote the set of all inputs with $\mathcal{S}$.
\end{dfn}

%
% MCT: Shouldnt' inputs be represented as I?
% SB: Maybe. We can revisit this before publication.
%

\begin{dfn}[Actuators and outputs]
  An \emph{actuator} is an edge $(A,B) \in G_\W$ such
  that $A \in \Y$ and $B \in \mc{E}$.
  An \emph{output} is the tail node of an actuator, $A$.
  Denote the set of all outputs with $\A$.
\end{dfn}
\begin{center}
\begin{tikzcd}[column sep=tiny]
  \mathcal{S} \arrow[r] &
  \Y \setminus (\mathcal{S} \cup \A) \arrow[r] &
  \A \arrow[dl, bend left=50]\\
  & \mathcal{E} \arrow[ul, bend left=50] &
\end{tikzcd}
\end{center}

See the definition of \emph{orderly} (Definition \ref{def:orderly-system})
for a set of further constraints on inputs and outputs
that are necessary for proving security properties of ECS models.

For some security related applications of this model,
it is necessary to distinguish between ``high-side''
and ``low-side'' inputs and outputs.
High-side variables denote sensitive variables
that should not be leaked.

\begin{dfn}[High and low sides]
  Inputs $\mathcal{S}$ are partitioned into high $\mathcal{S}_H$
  and low $\mathcal{S}_L$ side variables.
  Similarly, outputs $\A$ are each partitioned
  high-side $\A_H$ and low-side $\A_L$ variables.
\end{dfn}
\begin{center}
\begin{tikzcd}[column sep=tiny]
  \mathcal{S_H} \arrow[r]
  & \Y \setminus (\mathcal{S} \cup \A)
  \arrow[r]
  \arrow[dr]
  & \A_H \arrow[ddl, bend left=100]\\
  \mathcal{S_L} \arrow[ur] &
  &
  \A_L \arrow[dl, bend left=50]\\
  &
  \mathcal{E}
  \arrow[uul, bend left=100]
  \arrow[ul, bend left=50]
  &
\end{tikzcd}
\end{center}

Note that in the above diagram and throughout this paper,
we will sometimes refer to a set of random variables
such as the set of all high-side inputs $\mc{S}_H$
as if it is a single random variable.
This is well-motivate, because for any set of random
variables $\mc{X} = \{X_0, X_1, X_2, ...\}$
one can define a new random variable
whose domain $Dom(\mc{X})$ is the cross product
$Dom{X_0} \times Dom{X_1} \times ...$ and whose
probability distribution is the joint probability
distribution $Pr(X_0, X_1, ...)$.

\begin{exm}
  A hospital uses a system to manage its medical records.
  It takes input from many health care professionals through
  many different forms of treatment.
  Most medical records are considered a low-side input
  because they can be accessed by other professionals
  treating the patient.
  Psychotherapy notes are a high-side input because they
  have special restrictions on their use.
\end{exm}

\begin{exm}
  An intelligence agency has many classified sensors,
  such as satellites and drone imagery, which contain information
  that is critical for national security.
  These are high-side inputs.
  They also use many data sets that are available publicly
  and commercially.
  These are low-side inputs.
\end{exm}

\subsection{System design}
\label{sec:design}

In many cases what we are interested in is the possibility
of a \emph{system designer} inventing a system subject
to certain constraints.

%An example of such a constraint is the system's
%meeting a \emph{specification}, which is a desired
%relationship between inputs and outputs.
%
%\begin{dfn}[Specification]
%  A \emph{specification} is a function
%  $f \from \mathcal{S}, \A \to [0,1]$
%  \mct{Do you mean $f \from \mathcal{S} \times \A \to [0,1]$?}
%  that is a joint probability distribution
%  over $\{\mathcal{S}, \A\}$.
%  The system $\Y$ fulfills the specification iff
%  $Pr(\mathcal{S},\A) = f(\mathcal{S},\A)$.
%  \mct{I'm not sure what $f(\mathcal{S},\A)$.  If $f$ is a function
%    from $\mathcal{S} \times \A$, then it has to take elements of them
%    as inputs, not the whole sets.}
%  \mct{If $\Y$ is fulfilling something, it must be found in the
%    equation somewhere.  I suspect it's implicitly determining $Pr$,
%    but maybe that should be made explicit as $Pr_{\Y}$.}
%\end{dfn}
%
% I dont' think I use this definition  of nontriviality anywhere in the draft...
%
%\begin{dfn}[Nontriviality]
%  A specification $f \from \mathcal{S}, \A \to [0,1]$
%  is \emph{nontrivial} iff given $Pr = f$,
%  $Pr(\A \vert \mathcal{S}) \neq Pr(\A)$.
%\end{dfn}
%
%\mct{I think you're just saying that $f$ depends upon the input
%  $\mc{S}$: there exists $s_1$, $s_2$, and $a$ such that
%  $f(s_1,a) \neq f(s_2,a)$.}
%
%\mct{Making $f$ be a joint distribution over inputs and outputs is
%  nonstandard.  Typically, we make it a function from inputs to
%  distributions over outputs.  This isn't just currying, but looks
%  similar.  We should discuss.}
%

%The general theory of Bayesian networks shows
%that the extent to which nomic associations
%between variables allow information to flow
%between them depends not only on the causal
%structure of the network, but also on which
%other variables have already been conditioned
%on or, in a Bayesian sense, are observed.
%

We are interested in the ways that an ECS
enables inferences, and how these inferences
depend on what is known or observable about
the system and its environment.
We define some terms here to denote properties
of the conditioning set that we will use in
later proofs.
Intuitively, conditioning sets can be interpreted
as observed states of the world that are available
to an adversary trying to learn high-side information.

The condition of being \emph{present} captures the
intuition that system designers cannot account for
the ways that downstream uses of system outputs
may be used to reveal sensitive information.

\begin{dfn}[Present]
  A system $\mathcal{Y}$ with conditioning set $\mc{C}$ is \emph{present}
   iff
  \begin{itemize}
  \item No descendants of $\mathcal{A}$ are in the
    conditioning set $\mc{C}$, and
  \item No descendant of $\mathcal{A}$ is in
    $\mathcal{S}$.
  \end{itemize}
\end{dfn}

The term ``present'' indicates that the attacker
is able to condition on variables prior to and
during the operating of the system, but not
variables in the ``future'' of the system.
Requiring that the system outputs are not ancestors of
the the system inputs guarantees that the system is
in fact positioned in a particular place in time,
so to speak.

The condition of being \emph{covered} captures the intuition
that in general we do not expect attackers to have
the ability to observe systems 
\emph{as they are functioning}, even if we allow them
to know exactly how a system works because
they know the causal relationships between the
system components.

\begin{dfn}[Covered]
  \label{def:covered-system}
  A system is \emph{covered} if no $Y \in \mathcal{Y}$
  is in the conditioning set $\mc{C}$.
\end{dfn}

%
%\mct{A somewhat subtle point here is that people don't see $\A$
%  directly, but rather see the effects of $\A$.  If the reader isn't
%  paying attention they might take your model as ruling out seeing
%  outputs altogether.}
%

We also specify a condition on the relationship
between sensors, actuators, the system, and the environment.
In some cases we will not allow an input to be caused
by a system variable.
We will also not allow an output to cause a system variable.
When both these conditions hold, we call a system \emph{orderly}.

\begin{dfn}[Orderly]
  \label{def:orderly-system}
  A system is \emph{orderly} iff:
  \begin{itemize}
  \item $\forall X \in \mathcal{W}, S \in \mathcal{S}, X \in Pa(S) \Longrightarrow X \in \mathcal{E}$
  \item $\forall X \in \mathcal{W}, A \in \mathcal{A}, A \in Pa(X) \Longrightarrow X \in \mathcal{E}$
  \end{itemize}
\end{dfn}

Given only the subgraph represented by $\mathcal{Y}$,
under some condition it will be the case that the
high-side inputs and the low-side outputs are conditionally
independent.
We will name this property \emph{safety}.

\begin{dfn}[Safe]
  A system $\mathcal{Y}$ is \emph{safe} given
  conditioning set $\mc{C}$ iff
  when considering it as a subgraph,
  there are no unblocked paths between 
  $\mathcal{A}_L$ and $\mathcal{S}_H$.
\end{dfn}

If there are no unblocked paths between $\A_L$ and ${S}_H$
in the system subgraph, then these variables are d-separated
and so the system is safe.
We assume for our purposes that a system designer can
guarantee its safety through sound engineering alone.

For example, in the system defined by the graph
\begin{center}
\begin{tikzcd}
  \mathcal{S}_H \arrow[r] & \mathcal{A}_H \\
  \mathcal{S}_L \arrow[r] \arrow[ur]& \mathcal{A}_L
\end{tikzcd}
\end{center}

when $\mathcal{A}_H$ is not in the conditioning set,
the path between $\mathcal{S}_H$ and $\mathcal{A}_L$
is blocked, so the high-side input and low-side
outputs are independent.

In general, system designers can guarantee safety
by removing any direct paths from $\mathcal{S}_H$ to
$\mathcal{A}_L$ and then ensuring that the system is
covered.

\begin{thm}
  \label{thm:safety}
  If a system subgraph $\Y$ has no direct paths from
  $\mathcal{S}_H$ to $\mathcal{A}_L$ and is covered
  and orderly, then it is safe.
\end{thm}
\begin{proof}
  Assume there are no direct paths from $\mathcal{S}_H$ to
  $\mathcal{A}_L$ in the system subgraph.
  So any unblocked path between $\mathcal{S}_H$ and
  $\mathcal{A}_L$ must be indirect.
  
  Suppose there is a path $p$ that begins
  with an incoming edge to $\mathcal{S}_H$,
  as in $\mc{S}_H \leftarrow \cdots \mc{A}_L$.
  Because the system is orderly, incoming edges
  to into input $\mathcal{S}_H$ must come
  from nodes that are not in the system $\Y$,
  as in $\mc{S}_H \leftarrow \mc{E} \cdots \mc{A}_L$.
  Because these
  nodes are not in the system subgraph, they cannot be
  in $p$.
  
  Therefore, the path $p$ must begin with an outgoing edge to
  $\mathcal{S}_H$. By a parallel argument, the path
  must end with an incoming edge to $\mathcal{A}_L$,
  as in $\mc{S}_H \rightarrow \cdots \rightarrow \mc{A}_L$.

  Because the path $p$ is indirect and it begins with an outgoing
  edge and ends with an incoming edge, there must be some
  node $X$ such that $X$ is on the path and $X$ is a common
  effect node, 
  as in 
  $\mc{S}_H \rightarrow \cdots \rightarrow X \leftarrow \cdots \rightarrow \mc{A}_L$.
  %\mct{Why?}
  
  Because the system is covered, $X$ must be unobserved.
  This implies that the path $p$ is blocked.
  Therefore, there are no unblocked paths between
  $\mathcal{S}_H$ and $\mathcal{A}_L$.
  Thus, by Theorem~\ref{thm:d-seperated}, these
  variables are independent and the system is safe.
\end{proof}

Information flow security literature often considers
systems or programs in isolation from their environment.
In practice, systems are always connected with an environment,
which is why we have developed ECS.
So Theorem \ref{thm:safety} is not enough to show the conditions of
security in an ECS model because its inputs and outputs
are not connected with variables in an environment.
For this, we turn to a well established formal security model,
noninterference.


\subsection{Noninterference}
\label{sec:noninterference}

\emph{Noninterference}, introduced by \cite{gm82security},
is a security policy model widely used in computer science.
Sabelfeld and Myers~\cite{sabelfeld03journal} define noninterference
informally as ``a variation of confidential (high)
input does not cause a variation of public (low) output.''

More formally, model a program $C$ as taking an
input state $s = (s_h,s_l)$,
as producing an output in a set $S \cup \{ \bot \}$,
where
$\bot$ stands for nontermination
and $\bot \notin S$.
We use $\lbb C\rbb  : S \rightarrow S \cup \{ \bot \}$
to denote such a semantics.
An equivalence relation $=_L$ holds when
two inputs they agree on the low values ($s=_Ls'$ iff
$s_l = s_l'$). The attacker's power is characterized
by a relation $\approx_L$ such that if two behaviors
are related by $\approx_L$ they are indistinguishable
to an attacker. 

Following \cite{tschantz15csf} and \cite{datta2017use},
we expand the definition of the operator $\lbb \cdot\rbb $
to be a function to and from probability distributions
over states, which affords a probabilistic definition
of noninterference.

\begin{dfn}[(Probabilistic) noninterference]
For a given semantic model, $C$ is
  exhibits \emph{noninterference} or \emph{is secure} iff for all $s_1$ and $s_2$ in $S$, $s_1 =_L s_2$ implies $\lbb C\rbb(s_1) \approx_L \lbb C\rbb(s_2)$.
\end{dfn}

This definition admits a wide range of possible semantics for
the attacker's equivalence relation $\approx_L$.

We will choose a particular semantics relevant
to the ECS model.
We impose a probability distribution over inputs
$\mathcal{S}$.
With it we can construct the variable 
$\A = \lbb C\rbb(\mathcal{S})$.
We use $\Y$ to denote the minimal Bayesian network relating
$\mathcal{S}$ to $\A$ and treat it as the system model.
As per the ECS model, we partition the inputs
and outputs into high and low sides,
$(\mathcal{S}_H, \mathcal{S}_L)$ and
$(\mathcal{A}_H, \mathcal{A}_L)$,
respectively.
%\mct{This should be done in a way respecting $=_L$, right?}
%sb: I don't know what this means.
Define attacker indistinguishability $\approx_L$
as probabilistic indistinguishability of the low-side outputs when
conditioned on inputs:

\begin{dfn}[ECS attacker indistinguishability]
  $$\A \approx_L \A' \text{ iff } Pr(\A_L) = Pr
  (\A_L')$$
\end{dfn}

Conceptually, we are modeling the execution of the
program $C$ as the realization of the random variables in
$\Y$.
$C$ implies a probability distribution $Pr(\mc{A} \vert \mc{S})$.
It also implies a probability distribution of $\mc{A}$
conditional on $\mc{S} = s$ realized.
$\lbb C\rbb(s) = Pr(\A \vert \mathcal{S} = s)$, where $s$ is an instantiation of $\mathcal{S}$.
For $s \in \mathcal{S}, a \in \A$, let
$(s_h,s_l,a_h,a_l) =_L (s_h',s_l',a_h',a_l')$
iff $s_l = s_l'$.

\begin{dfn}[ECS Noninterference]
  For a given ECS model, $\Y$
  exhibits \emph{noninterference} or \emph{is secure} iff
 $$\forall s_1,s_2  \in \mathcal{S}, s_1 =_L s_2 \Longrightarrow P(\A \vert \mathcal{S} = s_1) \approx_L P(\A \vert \mathcal{S} = s_2)$$
\end{dfn}

\begin{cor}\label{cor:noninterference-ind}
  $\Y$ is secure by noninterference iff
  $$\A_L \independent \mc{S}_H \given \mc{S}_L$$
\end{cor}
\begin{proof}
  $\Y$ is secure by noninterference iff
  $$\forall s_1,s_2  \in \mathcal{S}, s_1 =_L s_2 \Longrightarrow P(\A \vert \mathcal{S} = s_1) \approx_L P(\A \vert \mathcal{S} = s_2)$$
  iff
  \begin{equation}
    \begin{split}
      & \forall s_H,s_H',s_L,s_L'  \in \mathcal{S}, s_L = s_L' \\
      & \Longrightarrow P(\A \vert \mathcal{S}_H = s_H, \mathcal{S}_L = s_L) \approx_L P(\A \vert \mathcal{S}_H = s_H', \mathcal{S}_L = s_L')
    \end{split}
  \end{equation}
  iff
  \begin{equation}
    \begin{split}
      & \forall s_H,s_H',s_L \in \mathcal{S} \\
      & P(\A_L \vert \mathcal{S}_H = s_H, S_L = s_L) = P(\A_L \vert \mathcal{S}_H = s_H', S_L = s_L)
    \end{split}
  \end{equation}
  iff
  $$A_L \independent S_H \vert S_L$$
\end{proof}

Under what conditions is a system secure by noninterference?

\iffalse
\cite{sabelfeld03journal}'s informal definition of
noninterference is expressed in terms of causation:
``a variation of confidential (high)
input does not cause a variation of public (low) output.''
In Pearl's account of causation, an intervention or
manipulation of a variable in a causal model sets its
value and, for the purposes of inferring other variables,
severs the edges pointing to that variable from others
in the graph.
Consequently, even if we model a system within its environment,
if there is not path from its high-side inputs to its low-side
outputs then an \emph{intervention} (in Pearl's sense)
on its high-side inputs will not cause a variation in the
low-side outputs.
\fi

We can prove that if the system designer can guarantee
that the system is present and safe, then it is secure
by noninterference.

\begin{lem}
  \label{lem:present}
  If an ECS model is present,
  then there can be no unblocked path between
  $\mc{S}_H$ and $\A_L$ that includes an outgoing
  edge from $\A_L$.
\end{lem}
\begin{proof}
  Proof by contradiction.
  
  Suppose an unblocked path exists
  between $\mathcal{S}_H$ and $\A_L$
  such that the edge connecting to $\A_L$ was outgoing.
  That is, suppose the path was of the form:
  $$S_H \cdots \leftarrow \A_L$$
  for some $S_H$ in $\mc{S}_H$.

  Consider the sequence of nodes on the path
  $S_H, X_1, X_2, \ldots, X_n, \A_L$ and the direction of
  the arrows between them, with $X_n \rightarrow \A_L$,
  and labeling
  $S_H$ with $X_0$.

  Because the system is present, no descendant of $\A_L$,
  is in $\mathcal{S}$.
  Therefor the must be at least one edge on the path
  such that $X_{i-1} \rightarrow X_i$.

  Count down from $n$ to $1$ and identify the first
  $X_i$ such that $X_{i-1} \rightarrow X_i$, 
 
  $X_i$ will be a descendant of $\A_L$ because
  there is a direct path between $\A_L$ and it.
  $X_i$ cannot be $X_0 = S_H$, which is in $\mc{S}$.

  Therefore node $X_i$ will be the common cause of a
  head-to-head connection.  That is,
  \[S_H \cdots X_{i-1} \rightarrow X_i \leftarrow X_{i+1} \leftarrow \cdots \leftarrow \A_L \]
  Because the system is present, this node
  $X_i$ must not be in the conditioning set.
  The path must therefore have a head-to-head
  connecting node that is not in the conditioning
  set.
  So it is a blocked path, resulting in a contradiction.
\end{proof}

\begin{thm}
  \label{thm:ecs-noninterference}
  Given an ECS model,
  if the system is present, safe, and orderly
  then the system
  is secure by noninterference.
\end{thm}

\begin{proof}
  Consider possible paths between $\A_L$ and $\mc{S}_H$
  while $\mc{S}_L$ is in the conditioning set.

  By Lemma \ref{lem:present}, a present system
  can have no unblocked paths from $\mc{S}_H$ to $\A_L$
  that end with an outgoing edge from $\A_L$.
  So any unblocked path from $\mc{S}_H$ to $\A_L$ must
  end with an incoming edge into $\A_L$.

  Because the system is orderly,
  if $X \rightarrow A$ for any $A$ in $\A$, then $X \in \Y$.
  Therefore, unblocked paths must include nodes
  in the $\Y$ subgraph.
  
  Because the system is safe, there are no unblocked paths
  between $\mathcal{S}_H$ and $\A_L$ consisting of 
  only nodes in the system $\Y$ subgraph.

  So any unblocked path between $\mc{S}_H$ and $\A_L$ must include
  both nodes that are in $\Y$ and nodes the are in $\mathcal{E}$.
  We have already ruled out paths that include descendants
  of $\A$.
  So such a path must include ancestors of $\mathcal{S}$.
  The path must begin with $\mc{S}_H$, go into the environment,
  then re-enter the system via $\mc{S}_L$, then go to $\A_L$.
  Because the system is orderly, incoming edges into $\mc{S}_L$
  must be $E \in \mathcal{E}$ and outgoing edges must be
  $Y in \mathcal{Y}$.
  That is,
  $$S_H \leftarrow \cdots E \rightarrow S_L \rightarrow Y \cdots \leftarrow A_L$$
  for some $S_H$ in $\mc{S}_H$, $S_L$ in $\mc{S}_L$, and $A_L$ in $\A_L$.

  $S_L$ is in the conditioning set and part of a head-to-tail
  structure $E \rightarrow S_L \rightarrow Y$ on the path.
  Therefore the path is blocked.
  
  So with $\mathcal{S}_L$ in the conditioning set,
  all paths between
  $\A_L$ and $\mathcal{S_H}$ must be
  blocked.

  So $\A_L \independent \mc{S}_H \vert \mc{S}_L$.
  By Corollary~\ref{cor:noninterference-ind},
  the system is secure by noninterference.
\end{proof}

It is therefore possible to implement an ECS that is
secure in the sense of noninterference
as long as a few conditions of the system
(covered, safe, and orderly) are guaranteed.
Privacy policies that restrict information flow
(e.g. by guaranteeing confidentiality) of data
based on how it was inputted into the system can
be modeled in this framework.
In the next section, we will show that policies
that impose restrictions on information flow
based on the content of information cannot be
as easily restricted; to be effective there
must be independence relations in the environment
of the system.
Thus the viability of privacy policies depends
on the distinction noted in \ref{sec:ambiguity}
between causal flow and association.

\subsection{Preventing associations}
\label{sec:prevention}

We have proven that under certain conditions
an ECS is secure by noninterference
(Theorem \ref{thm:ecs-noninterference}).
Noninterference is a widely respected formal
security model in among computer security
researchers.
One reason for this is that it is a criterion
that depends only one the internals of the
system.
Computer scientists can guarantee that a system,
such as a program, is secure by noninterference
without considering how that program will be used
in practice.
What if we wanted to hold systems to a higher standard
that takes into consideration the processes that generate
a system's data?
For this we need a stronger security property.

We can be more specific and introduce a security policy model
that is strictly stronger than noninterference.

\begin{dfn}[ECS semantic security] %%%%
  For a given ECS model, $\Y$ is
  exhibits \emph{semantic security} iff

  $$ \mc{S}_H \independent \A_L $$
\end{dfn}

Semantic security is a well known property of
cryptographic systems that means, intuitively,
that an attacker intending to determine the
contents of a message might as well not look at
an encrypted signal.
The term has taken on a wider use in the differential
privacy literature as it has been introduced
as a desideratum for statistical
databases along the lines of that proposed
by Dalenius in 1977. \cite{dalenius77statistik, dwork06icalp}
Dwork and Naor \cite{dwork06icalp,dwork08jpc} show that it is impossible to
guarantee the semantic security of such a database
given arbitrary auxiliary knowledge.

We draw on the spirit of this literature in our
definition of ECS semantic security.
The principle difference between ECS semantic
security and noninterference is that the
latter is concerned with the Independence of
system outputs from sensitive inputs
\emph{conditioned} on the inputs,
whereas the former takes into consideration
how environmental correlations may allow
system outputs to reveal system inputs.

Noninterference does not imply semantic security.
Put another way, the same system can be secure by
noninterference but semantically insecure.
Consider the system:

\begin{center}
\begin{tikzcd}
  E \arrow[r] \arrow[dr] & S_H \arrow[r] & A_H\\
  & S_L \arrow[r] \arrow[ur] & A_L \\
\end{tikzcd}
\end{center}

This system is safe when $A_H$ is not in the conditioning
set.
It is secure by noninterference because when $S_L$
is in the conditioning set, the path from $S_H$
to $A_L$ that goes through $E$ is blocked.
But when $S_L$ is not in the conditioning set,
this path is open and therefore $A_L$ can
be conditionally dependent on $S_H$. 

%\iffalse
%
%PROBLEM: A possible path is opened by $S_L$ being
%a common effect. So the system has to be present?
%
%Semantic security implies noninterference security
%under rather general conditions.
%
%*** PROOF NEEDED -- WHAT CONDITIONS? ***
%
%\begin{thm}
%  If a system is both orderly and
%  ECS semantically secure,
%  then it is secure by noninterference.
%\end{thm}
%\begin{proof}
%  *** Check edge cases! ***
%  If a system is semantically secure, then
%  $$A_L \independent S_H$$
%
%  By the definition of d-separation,
%  adding nodes to the conditioning set will only
%  unblock paths if those nodes are common
%  effects on the path, or are descendants
%  of a common path.
%
%  *** actually need ANOTHER condition,
%  which is that sensors are not
%  descendants from actuators?!? ***
%
%  $$A_L \independent S_H \vert S_L$$
%\end{proof}
%\fi
%

Our conjecture is that semantic security
cannot be guaranteed by the system designer alone.
We are able to prove sufficient conditions for
semantic security by including a general
property of the world, including the
environment outside the system.

\begin{dfn}[Propitious]
  The world $\mathcal{W}$ is \emph{propitious}
  iff there is no unblocked path between
  $S_H$ and $S_L$.
\end{dfn}

\begin{thm}
  \label{thm:semantic}
  If a system is present, safe, and orderly,
  and the world is propitious, then the
  system is semantically secure.
\end{thm}
\begin{proof}
  By Theorem \ref{thm:ecs-noninterference},
  the system is secure by noninterference,
  implying that
  $$S_H \independent A_L \vert S_L$$

  So no unblocked paths run from $S_H$ to $A_L$
  when $S_L$ is in the conditioning set.

  Recall from the proof of
  Theorem \ref{thm:ecs-noninterference}
  that this was because we ruled out all possible
  paths from $S_H$ to $A_L$.

  Paths running from $S_H$ through $S_L$ to $A_L$
  were blocked because $S_L$ was in the conditioning
  set.

  Consider any such path, now unblocked as $S_L$
  is not in the conditioning set.

  If it is unblocked, then there is an unblocked
  path between $S_H$ and $S_L$, which contradicts
  the assumption that the world is propitious.

  Therefore there are no unblocked paths between
  $S_H$ and $A_L$ and so $A_L \independent S_H$.
  
\end{proof}

%\mct{It feels like the punchline of this section should be that noninterference can be designed for whereas semantic security cannot.  However, this isn't made clear for two reasons.  
%\\
%First, it's not all the clear how much harder the conditions of Thm.~\ref{thm:semantic} are to meet than the conditions of Thm.~\ref{thm:ecs-noninterference}.  The harder one just requires adding one extra one, propitious, to the set of three required for the easier one.  A 25\% increase isn't that bad, right?  Well, that's, of course, not the right way of looking at this.  We need to look at how hard each of these conditions are to ensure.  I don't see such a discussion anywhere.  Indeed, more can be done to make each of these conditions intuitive (starting with their names).
%\\
%Second, people know that semantic security is something promised in crypto papers, so it can't be that hard, right?  We need some explanation about why it's possible to achieve it in that special case, but not in general.}
%
%\mct{Much of this section reminds me of work on composition properties for noninterference.  We should compare what we're doing to that work.}
%

\section{Formalizing origin privacy}
\label{sec:robustness}

We have defined origin privacy as privacy policies
that place restrictions on information based on its
provenance.
This is in contrast to policies that restrict information
based on its content.
Another way to put this difference is that origin-based
privacy policies restrict information based on
the structure of causal flows, while information content
based policies restrict information based on its nomic
associations.

The problem with restricting information flows based on
information content is well illustrated by the problem
of guaranteeing semantic security in an ECS system.
Any sensitive information content can potentially
have nomic associations with otherwise inocuous
inputs due to causal paths in the environment.
Guaranteeing the absence of associations depends
on properties of the environment that may be outside
the system designer's control.
Noninterference, on the other hand, is an achievable
property for a system designer.
However, it is defined in such a way that it can be guaranteed
even when some kinds of harmful information leaks are
probable in practice.

In our legal analysis in Section \ref{sec:policy}
we identified some policies that restrict information
based on its provenance, or origin, rather than its information
content.
In Section \ref{sec:ontology}, we have identified the origin
of information as the chain of processes causing its
input into the system.
Taking the concept of ``high-side'' input as those inputs
to a system that are treated with special sensitivity,
we can model an example world that meets the most basic
requirement of an origin based policy roughly like so:

\begin{center}
\begin{tikzcd}
  O \arrow[r] & R_0 \arrow[r] & \cdots R_n \arrow[r] & S_H \arrow[r] & A_H\\
  & & & S_L \arrow[r] \arrow[ur] & A_L \\
\end{tikzcd}
\end{center}

In this model, the original value $O$ is connected only
to the high-side input $S_H$ by a direct
path of relays $R_0, \ldots, R_n$.
We can define the origin property as:

\begin{dfn}[Origin restricted]
  A system $\Y$ with inputs $\mc{S}$ 
  is \emph{origin-restricted} for a protected
  variable $O$ iff all direct paths from $O$ to $\mathcal{S}$
  end in $\mathcal{S}_H$, and there is at least one such path.
\end{dfn}

In what sense is an origin restricted system secure?
We would like the low-side output of an origin restricted
system to be independent of the sensitive variable.
As we have seen in Sections \ref{sec:noninterference}
and \ref{sec:prevention}, there are multiple security
models that make different assumptions about the
conditions of security.
We can use analogous security models for origin privacy.

\begin{dfn}[Origin noninterference]
  $\Y$ is secure by origin noninterference with respect
  to a sensitive variable $O \in \mathcal{E}$ iff
  $$A_L \independent O \vert S_L$$
\end{dfn}

\begin{thm}
  \label{thm:origin-noninterference}
  Given an ECS model,
  if the system is present, safe, orderly,
  and origin restricted
  then the system
  is secure by origin noninterference.
\end{thm}

\begin{proof}
  Because the system is origin restricted,
  there is at least one direct path from $O$
  to $S_H \in \mathcal{S}_H$.

  If there were an unblocked path from $A_L$
  to $O$ that included an outgoing edge from $A_L$,
  this would extend into an unblocked path to $S_H$,
  violating the condition imposed by Lemma \ref{lem:present}.
  Therefore there is no unblocked path from $A_L$
  to $O$ that includes an outgoing edge from $A_L$.

  Because the system is orderly, incoming edges
  to $A_L$ must go to nodes in the system.
  Therefore, any unblocked path from $O$
  to $A_L$ must go through $\mathcal{S}$
  

  Because the system is safe, there is no unblocked
  path from $\mathcal{S}_H$ to $A_L$.

  Because the system is orderly, any path from $E \in \mathcal{E}$
  to $Y \in \Y$ going through $\mathcal{S}_L$ will
  include a head-to-tail triplet centered on $S_L$.
  Conditioning on this node $S_L$ blocks the path.

  Therefore there is no unblocked path between $O$
  and $A_L$, and the system is secure by origin noninterference.
\end{proof}

\begin{dfn}[Origin semantic security]
  $\Y$ is secure by origin noninterference with respect
  to a sensitive variable $O \in \mathcal{E}$ iff
  $$A_L \independent O$$
\end{dfn}

\begin{thm}
  Given an ECS model,
  if the system is present, safe, orderly,
  and origin-restricted
  and the world is propitious,
  then the system
  is secure by origin semantic security.
\end{thm}

\begin{proof}
  By Theorem \ref{thm:origin-noninterference},
  the system is secure by origin noninterference.

  The system is origin restricted, implying that
  there is at least one direct path from $O$ to $S_H$.
  $S_L$ cannoth be on this path because the system
  is orderly.
  As no node on this path is in the conditioning
  set, it is not blocked.
  
  Mirroring the proof to \ref{thm:semantic},
  we consider any path $\phi$ between $O$ and $A_L$
  that was blocked by conditioning on $S_L$.
  Such path must have a node $S_L$ either within
  a head-to-tail triplet or as a common cause.

  Suppose $\phi$ includes $S_L$ in a head-to-tail
  triplet.
  Then there is a subpath of $\phi$ there is an
  unblocked path between $S_L$ and $O$.
  But there is also an unblocked path from $O$
  to $S_H$, implying that there is an unblocked
  path from $S_L$ to $S_H$.
  This contradicts the condition that the world is
  propitious.

  Suppose $\phi$ includes a $S_L$ as a common cause
  node.
  Because the system is orderly, both outgoing
  edges must go to nodes in $\Y$.
  The path $\phi$ must therefore enter the system
  through a node in $S_H$.
  That implies a subpath of $\phi$ within the
  system runs unblocked from $S_H$ to $S_L$.
  That contradicts the condition that the world
  is propitious.

  Because no unblocked path between $O$ and $A_L$
  is possible, $O \independent A_L$ and the
  system has origin semantic security.
  
\end{proof}

This demonstrates that origin restrictions
do prevent associations between
low-side outputs and the sensitive environmental
variable under the condition that the systems
are otherwise secure.

%
%\mct{It would be nice if the above theorems were if-and-only-ifs.  I
%  suspect showing such a thing would be complicated by the possibility
%  of not having d-separation but the distributions having independence
%  anyhow.}%
%
% sb: that would be nice. We would need to 'faithfulness' property,
% or use the 'there exists one network with this structure such that...'
% techniques to establish soundness.
%

%
%Note that while it is considered true in privacy literature
%that it is impossible it is impossible to prevent
%inferences from being drawn from data because
%of the possible presence of auxiliary knowledge
%\cite{dwork06icalp,dwork08jpc}, we can see an exception
%to this rule when considering a system embedded in
%its environment.
%If there is in fact independence between the system and
%and environmental variable, there is no auxiliary information
%that would allow one to infer that environmental variable
%from system state.
%
%If we consider an embedded causal system,
%the structure of the world determines what
%auxiliary knowledge is possible.
%More granularly, this has implications
%for what combinations of structural knowledge and
%observations of particular variables are necessary
%for an adversary to draw sensitive inferences from
%system outputs.
%This granularity has let us define a positive privacy
%property, origin privacy, that allows for guarantees
%about inferences about sensitive variables on the
%condition that system designers have reliable
%knowledge of the structure of the world.
%

\section{Use Case: IoT and biometric data}
\label{sec:usecase}

In this section we introduce a use case of Origin Privacy that
we have identified through legal analysis and conversations
with stakeholders.

\begin{exm}[Smart building biometric sensing]
In an ``Internet of Things'' instrumented building, many sensors
collect information about the contents of rooms, including
photograph and other imagery such as infrared scanning
to identify the number and size of people
present. This information is useful to control the environment
in the room (heating, ventilliation). However, this data can
also be potentially identified using auxiliary information,
such as a facial recognition database.
This processed data reveals the identities of persons in
the room.
In some cases this may be intentional, as when it is used
for building security.
In other cases these revelations may be unexpected and constitute
an invasion of privacy.
\end{exm}

We chose this example because it highlights the way smart building
technology interacts with privacy policies around photography
and biometric data.

\subsection{GDPR biometric data}

Here we focus particularly on the EU's General Data Protection
Regulation (GDPR) (Regulation (EU) 2016/679).
In general, the GDPR places a number of restrictions on
the processing of \emph{personal data}, which it defines
thus:

\begin{quote}
  `personal data' means any information relating to
  an identified or identifiable natural person (`data subject');
  an identifiable natural person is one who can be
  identified, directly or indirectly, in particular
  by reference to an identifier such as a name,
  an identification number, location data,
  an online identifier or to one or more factors specific
  to the physical, physiological, genetic, mental,
  economic, cultural or social identity of that natural person;
  (Article 4 \S 1, GDPR)
\end{quote}

We interpret this definition as referring to the
topic or content of information; personal data is
any information \emph{relating to} a natural person.
As we have argued, a system designer cannot guarantee
that a system does not process information relating
to a natural person since these relations may be
caused by nomic associations that are external to the
system itself.

Noting this difficulty with ensuring compliance, we
can nevertheless continue to work with the more
specific requirements relating to biometric data.
In particular, the  GDPR makes a distinction between
photographs and biometric data:

\begin{quote}
The processing of photographs should not systematically 
be considered to be processing of special categories of personal 
data as they are covered by the definition of biometric data 
only when processed through a specific technical means 
allowing the unique identification or authentication 
of a natural person. (Recital 51, GDPR)
\end{quote}

By definition under the GDPR, biometric data is a form of personal
data that results from particular kinds of processes:

\begin{quote}
  'biometric data' means personal data resulting from
  specific technical processing relating to the physical,
  physiological or behavioural characteristics of a
  natural person, which allow or confirm the unique
  identification of that natural person, such as facial
  images or dactyloscopic data; (Article 4 \S 14, GDPR)
\end{quote}

Unlike the definition of personal data, the definition
of biometric data is an origin requirement because it
refers to the causal flow of data from a class of
processes.
Using these legal requirements, we can now
use Origin Privacy to formalize their semantics
with respect to a system.

\subsection{Formalizing GDPR requirements}

Consider the smart building example as an ECS.
Let $S_P$ be the photographic input to the system.
Let $S_D$ be a database of identified photographs,
originating from an external process $E_F$.
Let $Y_B$ be a component of the system $\Y$ caused by
$S_P$ and $S_D$; it includes imagery from $S_P$
that has been identified using the database $S_D$.

\begin{center}
\begin{tikzcd}
  E_F \arrow[r] & S_D \arrow[r] & Y_B \arrow[r] & A_H\\
  & S_P \arrow[ur] \arrow[rr] & & A_L \\
  \\
\end{tikzcd}
\end{center}

We note that the photographic input $S_P$ may
indeed ``relate to'' natural persons in a systematic
way if, for example, certain persons frequent the
smart building on a regular schedule.
Since these regularities (nomic associations) are
unknown to the system designer, there is little she
can do to guarantee the treatment of this information
as personal data.
What the system designer does have access to is the
identified faces database, $S_D$.
The \emph{process of identification} that results
in the biometric building surveillance data $Y_B$ requires
data from an identified source such as $S_D$.

The system designer knows about the origin of $S_D$.
Specifically, she knows that this data is sourced from $E_F$,
a process that personally identifies the facial images
within it.
Knowing the environmental source is sensitive,
they can impose the conditions for noninterference between
$E_F$ and $A_L$: that no unblocked path exist within $Y$ between
inputs originating in $E_F$ and $A_L$.
This implies that both $S_D$ and $Y_B$ be d-separated
from $A_L$ within $\Y$.
Note that in the diagram above, $E_F$ is indeedd d-separated
from $A_L$ when the system is \emph{covered}, i.e.
when none of its components are in the conditioning set.
Intuitively, $Y_B$ is subject to restricted flow because
it originates from the sensitive process $E_F$; it
inherits this origin from one of its parent components,
$S_D$.

We build on this result to model more complicated aspects
of GDPR compliance.
For example, processing of personal information, including
biometric information, is generally legal given the consent
of the identified person.
We can introduce identified sources into the ECS model
by denoting the set of natural persons $\mathcal{I}$,
and denoting a process that generates data from
an identified person $X_i$ for $i \in I$.
We can then place conditions on any data that is
a result of causal flow from this source, as
in this example specification:

\begin{exm}[Disclosure specification]
In the system, all outputs $A \in \A_L$ such that
$A$ is a descendant of $X_i$ must also be a descendant
of $Z_i$, where $i \in I$ is the identifier of a natural person,
$X_i$ is personally identifiable information, and
$Z_i$ is a disclosure agreement identifiable with that person.
\end{exm}

To the extent the GDPR controls on biometric information are
use restriction or topic restrictions
as opposed to an origin restriction,
they cannot be automatically enforced based on origin alone.
However, considering GDPR through the rigor of Origin Privacy
clarifies some of its requirements as formal specification
shows what knowledge is needed by the system designer for
origin based policy enforcement.

\section{Relationship to Differential Privacy}
\label{sec:differential}

In this section we will show the connection between
origin privacy and differential privacy.
Thus far we have defined origin privacy strictly in
terms of conditional independence.
This has been inspired by the formal security model
of \emph{noninterference}.

When assessing computational systems that process
personal information, it is possible for privacy
requirements to be looser than this strict security
requirement.
The particular case of privacy preserving statistical
analysis of databases of personal information has
motivated differential privacy as a formal privacy
model. \cite{dwork06icalp,dwork08jpc}

Formally, an algorithm $\A$ is $\epsilon$-differentially private
if for all subsets $S \subset image(\A)$ and
datasets $D_1$ and $D_2$ that differ on a single element,
the probability of the output of the algorithm run
on the datasets being in $S$ differs by at most a
multiplicative factor of $e^\epsilon$ and an additive
factor $\delta$. \cite{dwork2014algorithmic}

\begin{dfn}[Differential Privacy (($\epsilon,\delta$)-DP)]
$$Pr[\A(D_1) \in S] \leq e^\epsilon Pr[\A(D_2) \in S] + \delta$$
\end{dfn}

When $\delta = 0$, then $\A$ is $\epsilon$-differentially
private ($\epsilon$-DP).

Consistent with our origin privacy approach, we will
investigate how differential privacy can be assessed
given a model of system and its environment as a
causal Bayes network.
We will draw on prior results relating differential
privacy and causality and variations on differential
privacy expressed in terms of mutual information.

\subsection{Mutual information differential privacy}

\cite{cuff2016differential} demonstrate that there is
an equivalence between differential privacy and what they define
as mutual information differential privacy:

\begin{dfn}[Mutual information differential privacy ($\epsilon$-MI-DP)]
  A randomized mechanism $P_{Y \vert X^n}$ satisfies $\epsilon$-mutual
  information differential privacy if,

  $$sup_{i, P_{X^n}} I(X_i, Y \vert X^{-i}) \leq \epsilon$$

  where $X^{-i}$ is the dataset $X$ excluding variable
  $X_i$.
\end{dfn}

They prove that $\epsilon$-MI-DP is equivalent to differential
privacy in the sense that $\epsilon$-MI-DP implies
($\epsilon,\delta$)-DP) for some $\delta$, though
$\epsilon$-MI-DP is weaker than $\epsilon$-DP.
\cite{mcsherry_2017} argues that that $\epsilon$-MI-DP
falls short of the desiderata of $\epsilon$-DP.
Nevertheless, we will proceed with the
$\epsilon$-MI-DP because it is suggestive of
probabilistic structure may be used to
infer a privacy relevant bound.

That the mutual information limit
is conditioned on every other member of the database is
an indication of a disappointing fact about differential
privacy, which is that its beneficial properties
with respect to preserving privacy are not robust
to cases when entries in the database are correlated with each
other.

The value of using MI-DP for our purposes is that
the properties of mutual information are well-understood,
and we can derive a number of useful theorems
about mutal information between variables in a Bayesian
network.

\subsection{Randomizing database inputs}

We can now combine the previous results to
show how a system designer can develop a
Bayesian network model that guarantees
differential privacy.
A common way of achieving differential privacy
is by randomizing inputs to the database
prior to aggregation; this is done in practice
in \cite{erlingsson2014rappor}.
We can model randomization explicitly as in
the following example.

Let $X_i$ be a set of variables representing the
personal information to be aggregated.
Let $Y_i$ be a random variable over the same domain
as $X_i$ that is almost but not quite independent
of $X_i$; we'll say that the mutual information between
$X_i$ and $Y_i$ is bounded by $\epsilon_i$.
Then aggregate all the $Y_i$ variables into a database,
$Z$, that is available for querying.

\begin{center}
\begin{tikzcd}
  X_0 \arrow[r,"\epsilon"] & Y_0 \arrow[ddr] & \\
  X_1 \arrow[r,"\epsilon"] & Y_1 \arrow[dr] & \\
  X_2 \arrow[r,"\epsilon"] & Y_2 \arrow[r] & Z \\
  X_3 \arrow[r,"\epsilon"] & Y_3 \arrow[ur] & \\
  X_4 \arrow[r,"\epsilon"] & Y_4 \arrow[uur] & \\
\end{tikzcd}
\end{center}

In the above diagram, we annotate an arrow between
variables $A$ and $B$ with the upper bound on the
mutual information $I(A,B)$ afforded by the conditional
probability distribution. In this case, we
have set all the $\epsilon_i$ equal to each other,
$\epsilon$.

We can now use this graphical structure to prove
that this system is $2\epsilon$-MI-DP).
We can prove this using the Data Processing Inequality,

\begin{prp}[Data Processing Inequality]
  If three variables are in a Markov chain
  $$X \rightarrow Y \rightarrow Z$$
  where $X \independent Z \vert Y$, then $I(X,Y) \geq I(X,Z)$
\end{prp}

A standard proof of this is in the appendix.
We will also use the Path Mutual Information Theorem (PMIT),
that we prove as Theorem \ref{thm:path-mutual-information} in
the appendix.

\begin{exm}
  For the structure described above,
  $Z$ is $2\epsilon$-MI-DP.
\end{exm}

\begin{proof}
  Note that because $X_i$ and $X^{-i}$
  are joined only by a path with a common
  effect node, $I(X_i,X^{-i}) = 0$.

  It follows that:
  \begin{equation}
    \begin{split}
      I(X_i, Z \vert X^{-i})\\
      = I(X_i;Z,X^{-i}) - I(X_i,X^{-i}) \\
      = I(X_i;Z,X^{-i}) \\
      = I(X_i;Z) + I(X_i;X^{-1} \vert Z)
    \end{split}
  \end{equation}

  By DPI and the graphical structure,
  we know that for all $i$
  $$I(X_i,Z) \leq I(X_i,Y_i) = \epsilon$$

  By PMIT, we know the mutual information of
  two variables connected by a path with all
  of its common effect nodes observed is bounded
  by the mutual information of steps along the path.
  In this case, it entails that:

  $$I(X_i;X^{-1} \vert Z) \leq I(X_i,Y_i) = \epsilon$$

  By substitution, we know that:

  $$I(X_i, Z \vert X^{-i}) = I(X_i;Z) + I(X_i;X^{-1} \vert Z) \leq 2\epsilon$$

  As this holds for all $i$, it follows that $Z$ is $2\epsilon$-MI-DP.
\end{proof}


\subsection{Generalizing $\epsilon$-security}

We can generalize the two formal security models
that we introduced in Sections X and Y to models
that allow for $\epsilon$ mutual information between
sensitive inputs and low-side outputs.

\begin{dfn}[$\epsilon$-noninterference]
  A system is secure by $\epsilon$-noninterference iff 
  $$I(A_L, S_H \vert S_L) \leq \epsilon$$.
\end{dfn}

\begin{dfn}[$\epsilon$-semantic security]
  A system is $\epsilon$-semantically secure iff 
  $$I(A_L, S_H] \leq \epsilon$$.
\end{dfn}

Recall that two variables are perfectly independent
if and only if their mutual information is zero,
implying that $0$-noninterference is equivalent to
noninterference, and $0$-semantic security is equivalent
to ECS semantic security.

We can now show that differential privacy follows
from an application of $\epsilon$-noninterference
with one important caveat.
We have defined noninterference in terms of a system's
high-side and low-side inputs.
Schematically, we have considered the ``high side'' to
be a part of the system in need of special information
flow restrictions, while the ``low side'' is what's
available to less restricted access.

This model is intended to be generalized to cases where
there are multiple categories of restricted information.
In particular, to prove that $\epsilon$-noninterference
implies differential privacy, we must consider
\emph{each entry individually} to be noninterferent
with respect to the other entries in the data set.

\begin{thm}
  For an ECS model, if for all $D_i$,
  ECS is secure by $\epsilon$-noninterference with
  respect to $D_i$ as a high-side input and $D^{-i}$
  as low-side input, then $A$ is differentially
  private with respect to data set $D$
\end{thm}

\begin{center}
\begin{tikzcd}
  D_i  \arrow[rd,"\epsilon"] & \\
  D^{-1} \arrow[r] & A \\
\end{tikzcd}
\end{center}
\begin{proof}
  If for all $D_i$,
  ECS is secure by $\epsilon$-noninterference with
  respect to $D_i$ as a high-side input and $D^{-i}$
  as low-side input,
  then
  
  $$\forall i, I(D_i, A_L \vert D^{-i}) \leq \epsilon$$

  which implies that

  $$sup_{i, P_{X^n}} I(D_i, A_L \vert D^{-i}) \leq \epsilon$$

  which is the condition for $\epsilon$-MI-DP.
\end{proof}

It is interesting to note that in this proof,
the fact that noninterference conditions on some
inputs is necessary for proving its use in
securing differential privacy.
It is not generally the case that if
each of a database's entries $D_i$
is $\epsilon$-semantically secure
from the output $A$ that the output
will be $\epsilon$-differentially private.
A bound on $I(D_i;A)$ does not imply a
bound on $I(D_i; A \vert D^{-i}$), as
is apparent from Equation \ref{eq:mutual-with-semantic}.

\begin{equation}
  \label{eq:mutual-with-semantic}
  I(D_i;A \vert D^{-i}) = I(D_i,D^{-1}; A) - I(D_i;A)
\end{equation}

\section{Incentives in ECS}
\label{sec:incentives}

We have motivated ECS modeling by showing how it
captures the implicit ontology of privacy policies
and enables reasoning about the security properties
of systems embedded in an environment.
We have used Bayesian networks as a modeling tool
because they clarifying the relationship between
two aspects of information flow.
Bayesian networks also provide a robust, qualitative
means of determining dependency relations between their
variables, which we have used in our proofs about the
relationship between various system and privacy properties.
In this section, we will show how the same ECS framework
can be extended to include strategic actors and their incentives.

To accomplish this, we will use the Multi-Agent Influence Diagram
(MAID) framework developed by \cite{koller2003multi}.
In brief, a MAID is a Bayesian network with two important
extensions.

\begin{itemize}
\item Some of the variables are reserved as \emph{decision variables}
  and assigned to one of several agents.
  An agent's assignment of CPDs to its decision variables is
  that agent's \emph{strategy}; replacing each decision variable
  with the CPD from a \emph{strategy profile} transforms the MAID
  into a Bayesian network.
\item Some of the variables are reserved as \emph{utility variables}.
  These are assigned to one agent each, and are summed when realized
  into the agent's total utility.
  Utility variables must not have children.
\end{itemize}

A formal definition of MAIDS, strategies, and other useful
properties is given in Section \ref{appendix:maid},
which includes an account of the graphical notation we
will use in Section \ref{sec:maid-notation}.

\subsection{Expert services model}

As an illustration of an information game
that can be represented as a MAID, consider the
following diagram, which is a generalized model of
an expert service.
The services, which include health services, legal services,
as well as some software based services like search engines,
involve a client, $c$, who presents knowledge of their situation
to an expert, $e$.
The expert has access to general knowledge relevant to the
client's condition, and recommends some action based on
this knowledge.
The client can choose to take this recommendation.
In the idealized case considered here, the client
and the expert have perfectly aligned incentives and
so the expert will use their knowledge to the best
of their ability.

\begin{center}
  \begin{tikzcd}
    & W \arrow[ddl, bend right = 40] \arrow[ddr, bend left = 40]& \\
    & S \arrow[dl] \arrow[dd] \arrow[dr, bend left = 20, dotted]& \\
    V \arrow[dd] \arrow[ddr] &  & \tilde{R}_e \arrow[dl] \\
    & \tilde{A}_c \arrow[dl] \arrow[d] &\\
    \breve{U_e} & \breve{U_c} &\\
  \end{tikzcd}
\end{center}

In this model, $W$ is the generalized knowledge of the world that
is available to $e$ at their decision variable representing
their recommendation, $\tilde{R}_e$.
The action taken by the client $\tilde{A}_c$ and the
action value function $V$ determine the utility of
both the expert and the client, $\breve{U_e}$ and $\breve{U_c}$.
The value function is influence both by general facts about
the world $W$ and the particular situation of the client, $S$.
The client knows their situation but not the general expert knowledge.

The client's action is informed by the general knowledge only
through the recommendation of the expert.
The expert may or may not know the client's specific situation;
this is represented by the dotted arrow between $S$ and $\tilde{R}_e$.

The value of such a model is that a qualitative analysis
can readily provide insights into the influence of an
information flow on outcomes.
Since we know the expert's utility depends on influencing
the client to make the best possible action at $\tilde{A}_c$,
and that the value of this action depends on $V$,
the expert's effectiveness will be limited by how well
they can predict $V$ given the knowledge availabel to them
at $\tilde{R}_e$.
Without information about their client's specific situation $S$,
their advice can at best be perfectly general.
But with access to information $S$, the expert can improve
their recommendation and outcomes for both players.

\subsection{Expert ECS Model}
\label{sec:expert-ecs-model}

We now combine the expert service model with
the ECS model.
We will embed the expert insider an ECS and given them
access to personal information of the client via the
high-side input.
They will also have access to general knowledge
through the low-side input.
An adversary will have access to the low-side output,
but not the high-side output.
Using this model, we will be able to test how
the security properties we have analyzed in
Section \ref{sec:security}
and Section \ref{sec:robustness} can
be motivated in terms of the way they
affect the incentives of interacting with the
system.

\begin{center}
  \begin{tikzcd}
    & & W \arrow[dddll, bend right = 40] \arrow[dl, dotted, "e_1"] \arrow[ddd] \arrow[dddr, bend left = 30] & \\
    & C \arrow[d] \arrow[ddrr] \arrow[ddl] & & \\
    & \tilde{E}_a \arrow[d] & & \\
    V \arrow[dddd] \arrow[ddddr, bend right = 20] & \mc{S}_H \arrow[d] \arrow[ddr, dotted, "e_2"] & \mc{S}_L \arrow[dl] \arrow[dd] & V' \arrow[dddd] \arrow[ddddl, bend left = 20] \\
    & \tilde{D}_b \arrow[d] & & \\
    & \mc{A}_H \arrow[d] & \mc{A}_L \arrow[d] & \\
    & \tilde{D}_a \arrow[d] \arrow[dl] & \tilde{D}_e \arrow[d] \arrow[dr]& \\
    \breve{U}^+_a & \breve{U}_b & \breve{U}_e & \breve{U}^-_a\\
\end{tikzcd}
\end{center}

This game has three players, Alice ($a$), Bob ($b$),
and Eve ($e$).

Alice and Bob have perfectly aligned incentives,
and Eve is an attacker who is adversarial to
Alice.
We specify that the following relations hold:

$$\breve{U}^+_a = \breve{U}_b$$
$$\breve{U}^-_a = -\breve{U}_e$$
$$\breve{U}_a = \breve{U}_b - \breve{U}_e$$

At the center of this model is an ECS,
with high- and low- side sensors ($\mc{S}_H, \mc{S}_L$)
and actuators ($\mc{A}_H, \mc{A}_L$).

In this model, Alice is aware of her
personal information $\tilde{C}_a$ and decides at
$\tilde{E}_a$ what if any of it to divulge
secretly (via $\mc{S}_H$)
into the system because she wants an expert recommendation
from Bob.
Bob has access to general expertise $W$ through
a low-side input ($\mc{S}_L$).
These inputs are both available to Bob at his
decision node $\tilde{D}_b$, at which he chooses a
recommended action for Alice, which he passes through
the high-side output $\mc{A}_H$.

Alice will use the information about the recommendation
taken from the high-side output $\mc{A}_H$ to choose
an action.
The utility of this action will depend on the
action values $V$, which are a function of two
variables: the personal characteristics $C$ of Alice
and other general information about the world, $W$.

%\mct{Dummy cite: \cite{gm82security}}

Eve will make a decision $\tilde{D}_e$ based on
the low-side output of the system, $A_L$.
Eve's utility $U_e$ depends on this decision and
an action value function $V'$ that is analogous
to Alice's action value function $V$, in that
it depends on $C$ and $W$.

The system in this diagram is
$\mc{Y} = \{\mc{S}_H, \mc{S}_L, \tilde{D}_b, \mc{A}_H, \mc{A}_L\}$.

The diagram has two dotted edges, $e_1$ and $e_2$
Each dotted edge may be either included in the graph
(open) or excluded from the graph (closed).
The diagram therefore describes four distinct models:
none open, $e_1$ open, $e_2$ open, and both open.
We will analyze each case
in Section \ref{sec:expert-ecs-analysis}.


\subsubsection{Analysis}
\label{sec:expert-ecs-analysis}

First, we can analyze the expert ECS model
presented in Section \ref{sec:expert-ecs-model} in
terms of the system design properties introduced in
Section \ref{sec:design}.

Note that no descendent of $\mc{A}$ is in $\mc{S}$.
Therefore, the expert ECS model is present if none of
$\tilde{D}_a, \tilde{D}_e, \breve{U}^+_a, \breve{U}_b, \breve{U}_e,\breve{U}^-_a$
are in the conditioning set $\mc{C}$.

The system is covered if none of
$\mc{S}_H, \mc{S}_L, \tilde{D}_b, \mc{A}_H, \mc{A}_L$
are in the conditioning set $\mc{C}$.

It is plain from observation that the system is orderly.
It is also clear that the system is origin-restricted
with respect to the personal characteristics $C$:
there is one direct path from $C$ to $\mc{S}$ and it goes
to $\mc{S}_H$.

We are left with several candidates for the conditioning set:
$W, C, V, V', \tilde{E}_a$.
Recall that the world is propitious if there are no
unblocked paths from $\mc{S}_H$ to $\mc{S}_L$.
There are two ways an unblocked path can happen
under the conditions discussed so far.
One is that either $V$ or $V'$ is in the conditioning
set.
Another is that the edge $e_1$ is open.

It is clear that the system is safe if edge $e_2$ is closed
and not safe if edge $e_2$ is open.

Suppose that there are no variables in the conditioning
set.
Then by the reasoning above, the following properties
of the expert ECS system hold:
\begin{itemize}
\item If $e_1$ and $e_2$ are closed, then the system is
  origin secure with respect to $C$ both semantically
  and by noninterference.
\item If $e_1$ is open, then the system will be origin
  secure with respect to $C$ by nonintefernce, but may not
  be semantically secure.
\item If $e_2$ is open, then the system may be origin secure
  with respect to $C$ neither by noninterference nor semantically.
\end{itemize}

Though we have been able to show that these security
properties hold on the expert ECS model, this model
also reveals how these security properties do not
provide all desireable guarantees a system might
provide in terms of the incentives of Alice and Bob.

What can be shown is that given that the expert ECS
system is semantically secure, it is also the case
that $\tilde{E}_a$ and $\tilde{D}_e$ are tactically
independent (see Definitoin \ref{dfn:tactical-independence}),
meaning that for any strategy specifying decision rules
to each decision variable, in the induced probability
distribution $\tilde{E}_a$ and $\tilde{D}_e$ are independent.
In other words, at the level of tactics, Alice's choice to
reveal her information to the ECS will not depend on Eve's
choice of how to use the system's low-side outputs
adversarially.

However, despite these security properties,
we can show with this model that $\tilde{E}_a$
may \emph{strategically rely} on $\tilde{D}_e$
(see Definition \ref{dfn:strategic-reliance}).
This means that Alice's choice of decision rule 
at $\tilde{E}_a$ can depend on Eve's choice of
decision rule at $\tilde{D}_e$.
This can be a problem for system designers if
their goal is to guarantee that the presence of
Eve has no deterring effect on Alice's choice
to reveal her data to the ECS.

Further exploration of the relationship between
system security properties and incentives of
players in this causal formalism is left to
future work.

\section{Discussion and future work}
\label{sec:future}

We have analyzed privacy policies and discovered
that they variously restrict information based on
origin and topic.
We developed an informational ontology that reflects
the assumptions underlying many of these policies.
We have shown that this informal ontology can
be formalized in terms of causal graphical models.
Under this formalism, we show that the two aspects
of information flow correspond to precisely defined
concepts of causal flow and nomic association.
Using Bayesian networks, which clarify the difference between
two senses of information flow, we developed a model
of system security, the ECS model.

This model can
demonstrate and extend known results in computer security,
such as those concerning noninterference, semantic security,
and differential privacy.
It also allows us to formally define a new privacy property,
origin privacy, which assumes that system designers have
some control over the paths through which information enters
their systems.
We demonstrate how the ECS model can be used to elucidate
a case of implementing GDPR compliance on biometric data.
We demonstrated preliminery results on how the ECS model
can be extended into game theoretic form to account for
how strategically acting agents interact with systems
with or without relevant security properties.

These contributions are suggestive of several lines
of future work.

\subsection{Specifiability Criterion}

One direction for future work is to question what
these results mean for the design and interpretation
of legal policies.
Are privacy policies based on information topic harder
for consumers to understand than policies based on
information origin?
How do end users interpret ambiguous language in privacy
policies?

One benefit of using causal models as opposed to
program analysis for considering information flow
security of technical systems is that it shows
that information flow security is not only
a technical problem.
Because information leaks based on nomic associations
may be properties of any mathematically defined causal
system, these results extend to human institutions
as well.
In general, mathematical results about the limits
of computational enforceability will generalize
to the possibility of enforceability through
non-computational means as well.
Privacy policies and social expectations
that restrict information
based on its contents may be unenforceable or ambiguous
in general.

Concretely, the brittleness of these kinds of
restrictions is exposed
by advances in large-scale data analysis, or ``big data'',
which routinely confound our expectations about what
information is about.
Data about a person's purchases of scent-free hand lotion,
which we might have assumed to be innocuous, has been
shown to be correlated with sensitive information about
early-stage pregnancy \cite{hill12forbes}.
Ad targeting systems that use all available
correlational information in the data they collect risk
violating people's privacy due to unexpected discoveries
from automated learning processes.

By demonstrating the limits of what security properties
can be enforced, and by whom, this framework can
shed light on how privacy policies should be written
and which parties should be held liable in the event
of a violation.
This may build on other work in identifying legal
liability in cases of perceived inappropriate behavior
of a sociotechnical system \cite{datta2018discrimination}.

\subsection{Observer capabilities}
\label{sec:observer-capabilities}

We have shown that nomic associations between
system outputs and sensitive environmental variables
can lead to violations of privacy policies.
In order for these threats to be material,
nomic associations must be known to attackers
as auxiliary information.
A natural next step in this line of inquiry
is a more systematic study of how
observer's capabilities for learning nomic associations
factor into information flow security considerations.

In our models in this article, we have used
Bayesian networks as models of the objective
frequencies of variable outcomes.
Possible inferences from observed variables
have been interpreted as those inferences
possible \emph{in fact} from the causal
structure of the world.
Information flow security guarantees were
possible when the system designer was
assumed to have some true knowledge about
the system's environment, such as the
origin of its inputs.

In practice, most systems will be embedded in
environments that are only partially known.
In the cases of fraud and spam detection,
and other adaptive machine learning systems,
a model of the origin of inputs is trained
continuously from collected data.
Probabilistic graphical models are indeed
one of the many machine learning paradigms
used in these applications \cite{bishop2006pattern}.

A direction for future work is developing a
theory of information flow security and privacy
under conditions where observer knowledge of
nomic associations is itself a function of system
inputs. In simple cases this may reduce
to single program analyses already established
in work on use privacy \cite{datta2017use}.
In cases where multiple systems interact,
there may be novel problems.

\subsection{Incentives based policy design}

Prior work has been done on game theoretic
models of information flow policies \cite{barth07csf},
mechanism design with differential privacy
\cite{mcsherry2007mechanism}, and 
semantic interpretations of differential privacy
framed in terms of the incentives of data subjects
\cite{kasivisiwanathan14jpc}.
We see potential to continue this line of work
using statistical models of systems, their
internal processes, and their environment,
including the process that generate their
input data.

We have shown how Multi-Agent Influence Diagrams
(MAIDS) \cite{koller2003multi} can be used in
game theoretic modeling of security problems.
\cite{feng2014security} have used Bayesian networks
to model security risks and perform a security
vulnerability propagation analysis.
They build their causal model from observed cases
and domain experts.
We anticipate
new frameworks for aligning the incentives
of system designers and data subjects which
are sensitive to risk of data misuse (see
\cite{brooks15nist}).
Our work in Section \ref{sec:incentives} is
a first step in developing a new way of assessing
the value of security properties based on
their impact on game outcomes.

An application of this work is automated policy
design for smart buildings and cities, where
interacting data subjects and control systems
must share information while minimizing data
misuse.

\hfill
\hfill

We have unpacked the assumptions of privacy
policies to develop a general ontology of systems,
processes, and information.
We have then formalized this ontology using the
assumption that these systems are subject to
the laws of probability and an interventionist
account of causation \cite{woodward2005making}.
Using these simple assumptions, we have
confirmed well-known results about information
flow security about programs and systems in
isolation.
We have also gone beyond these results by showing
explicitly what their consequences are when
systems are embedded in an environment, developing
a general security modeling framework for this purpose.

We prove the relative effectiveness of
origin based information flow restrictions over
association based information flow restrictions
using the causally embedded system framework.
We show how origin privacy would be applied in
a GDPR and Internet of Things use case.
We anticipate new lines of inquiry extending
from the intersection of causal modeling
and information flow security, including
evaluation of policy enforceability,
modeling the role of observer knowledge
in privacy and security, and automated
policy generation through incentives-based
design.


\end{document}
