 \documentclass[../thesis.tex]{subfiles}
 \begin{document}

 This dissertation introduces data games,
 a formal method for determining
 the value of information in strategic games involving information
 flow.
 This method depends on a core theoretical contribution:
 the definition of information flow as a channel in the context
 of a causal system.
 This definition is built on the theoretical contributions
 of Shannon, Dretske, Pearl, Nissenbaum, and Koller,
 and motivated by legal analysis.
 It's applicability has been demonstrated through several
 studies in computer security and data economics.
 Among other uses, this definition and method is able
 to answer a question opened by the systematic study
 of Contextual Integrity's projection in computer science:
 the problem of cross-context information flows.
 This theoretical contribution therefore stands to inform
 future efforts in Privacy-by-Design \cite{gurses2011engineering}.

 One result for the economics of information implied by
 this definition and method is that
 \textit{information flows are not goods}.
 A change in information flow is a change in the strategic landscape
 and the structure of the social field.
 In the data economy information is not a good;
 it is a strategic resource.
 Because a new information flow can change the nomic associations
 of many connected events and actions in the economic field,
 any transaction may have a wide variety of externalities.
 A consequence of this is that traditional market economics
 developed for tangible goods are a poor fit for the information
 economy.
 More work is needed to develop information economics in a way
 that takes the causal structure of information flows into
 account.
 
 Though the definition of information flow developed in
 this dissertation depends on mathematical results
 from information theory, statistics, and computer science,
 it has implications for the structure of the information
 economy and the design and regulation of sociotechnical systems.
 Among other purposes, it is intended to inform the pragmatic
 matter of information law.
 Hence, I conclude this dissertation by explicitly positioning it
 as a scientific contribution, in two senses.
 First, in the Bourdieusian sense 
 \citep{bourdieu2004science}: as a claim to the arbitration of
 the real, as opposed to as a way of participating in
 an academic discipline which do not make such claims.
 Second, simply by virtue of its dependence on
 mathematics that are common in computer science
 and statistics while also addressing social phenomena,
 I position it as a contribution
 to the emerging field of
 computational social science, which gets its legitimacy
 from its acceptance of the objectivity of mathematical
 proof and the shared use of computational instrumentation
 \citep{benthall2016philosophy}.

 I believe the formal constructs developed in this
 dissertation are a good start to a scientific understanding
 of social information flow.
 But it is only a start.
 I end with a coda in the same spirit of
 Chapter \ref{chapter:bridge}:
 an expression of where this work resonates on a different
 intellectual plane.

 The definition of information flow used in this dissertation depends on models of the world in terms of Bayesian networks and
 Pearlian causation.
 It may be objected that the world can only ever approximately be
 described in terms of probabilistic events.
 Familiar questions about the interpetation of probability
 as either subjectivist or frequentist arise to further complicate
 matters

 To address these questions, consider:
 Pearlian causation has been successful not only as a statistical
 theory of causation useful in machine learning and the social
 sciences, but also in philosophy \citep{woodward2005making} and
 cognitive psychology \citep{sloman2005causal}, including
 cognitive psychology of moral judgement \citep{sloman2009causal}.
 A reason to use Pearlian causation
 in modeling the world is that it formalizes
 how we experience and reason about the world implicitly.
 We cannot escape the constraints that are implicit in the
 structure of our experience.
 And our implicit models of how the world works
 are structured in ways that can be characterized as Pearlian models.

 Critically, our models of the world are often wrong.
 This detail marks the inadequacy of the models presented
 in this dissertation: they assume that all agents have
 a shared understanding of the true causal structure of
 the world.
 This is hardly ever the case, and so there are a number of
 open problems resulting from the need to model
 differences in observer capabilities. (This point
 has been made already in Section \ref{sec:observer-capabilities}).

 The key consideration is that whereas the true nomic associations
 of an event depend on the true causal structure of the world,
 the intepreted meaning of an event to an observer will depend
 on their model of that structure.
 In terms of Shannon information theory, the world encodes
 itself into an event that the observer decodes upon observation.
 When there is a mismatch between the encoding and the decoding,
 information is lost.

 Proper analytic treatment of this subject will require more
 formal work.
 Speculatively, we can hypothesize that the social need for
 shared and accurate models of how information is generated
 is one of the reasons why society differentiates into
 social spheres:
 contextual roles, purposes, and information norms maintain
 the matching between social processes generating information
 and expectations of meaning society depends on to function.
 Developing and testing this hypothesis may have the happy
 result of putting contextual integrity onto the robust
 theoretical and empirical footing of computational sociology.
 This too must rest for now and await future work.

\end{document}
