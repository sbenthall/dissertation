\documentclass[../thesis.tex]{subfiles}
 \begin{document}

 The core contribution of this dissertation is a proposed
 solution to an old theoretical problem: the problem of the
 nature of information.
 
 
 
 By this standard, this completed dissertation is an impossible
 one.
 Its subject is the nature of privacy in general and whether and how it may be preserved in an economy dominated by information technology.
 It accomplishes this breadth through conceptual and mathematical
 abstraction.
 It draws on economics, law, and philosophy, fields that have long acknowledged their shared project of capturing the rules guiding social reality, as well as on computer science, which is arguably the field best equipped to understand the powers of computational technology in the world.
 Scholars such as \citet{lessig1999code} have argued that the market, the government, social norms, and technology have complementary roles in regulating society as a whole.
 Historically, these institutions have all had their own dedicated intellectual fields.
 This dissertation culminates in a mathematical framework that attempts to encompass their interelatedness all at once.

 I intend this dissertation as scientific contribution
 in the specific sense of science furnished by
 \citet{bourdieu2004science}.
 In a Bourdieusian science, scientists compete through their work
 for recognition for contributions to an understanding of the
 objectively real.
 By publishing this work, I am claiming that its conclusions
 are objectively true, or closer to the truth than the partial
 views that it synthesizes.
 I hope that they will be judged according to this standard.

 The force of the truth claims of this work come from the
 objective correctness of mathematical truth even and especially
 with respect to the organization of social reality, which I've
 argued is the cornerstone of the emerging field of
 computational social science \citep{benthall2016philosophy}.
 Computational social science is, in its current manifestation,
 most often considered an empirical science, using computational
 statistics to study data generated by society.

 This work inverts this picture.
 In Chapter \ref{chapter:games-and-rules}, I present
 an economic analysis of the information economy.
 Implicit in this analysis is the claimt that
 the mathematics of probability and information theory
 govern the economic value of data around which much
 social activity is organized.

 This is not the only theoretical inversions performed by this
 work.
 
 

 
 
 \section{}

 Two of the major chapters of this book are republications of joint work
 done with collaborators.

 Chapter 2, ``Contextual Integrity through the Lens of Computer Science'',
 was co-authored by Dr. Seda G{\"u}rses and Dr. Helen Nissenbaum \cite{benthall2017contextual}.

 Chapter 3, ``Origin Privacy: Causality and Data Protection'' resulted from
 long collaborations with Michael Tschantz and Anupam Datta. 



 
 \begin{quote}
 For the rational study of the law the blackletter man may be the man of the present, but the man of the future is the man of statistics and the master of economics. It is revolting to have no better reason for a rule of law than that so it was laid down in the time of Henry IV. It is still more revolting if the grounds upon which it was laid down have vanished long since, and the rule simply persists from blind imitation of the past. \cite{holmes1897path}
 \end{quote}

 
 Other literature on regulation of the Internet sheds
 light on this conceptual dilemma.
 \citet{lessig2009code} argues that
 cyberspace is subject to four broad modalities of regulation,
 which are technical architecture, the market, the law, and
 social norms.
 Contextual integrity can be thought of as an articulation
 of the logic of social norms, which is one of those four modalities.
 Chapter \ref{chapter:CI-through-CS} attempts to discover intellectual
 tensions between the logic of social norms and the logic of technical
 architecture design.


 
\end{document}
