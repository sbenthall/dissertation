% \usepackage[backend=bibtex,maxcitenames=10,maxbibqnames=10]{biblatex}
%biblatex issue: http://tex.stackexchange.com/questions/311426/bibliography-error-use-of-blxbblverbaddi-doesnt-match-its-definition-ve
%\makeatletter
%\def\blx@maxline{77}
%\makeatother

% 10pt is allowed only with an ugly font.
%\usepackage[utf8]{inputenc}
%\usepackage[T1,T2A]{fontenc}
%\usepackage[latin1]{inputenc}


\usepackage{amsthm}
%\theoremstyle{plain}
\theoremstyle{definition}
\newtheorem{thm}     {Theorem}
\newtheorem{lem}[thm]{Lemma}
\newtheorem{prp}[thm]{Proposition}
\newtheorem{cor}[thm]{Corollary}
\newtheorem{cnj}[thm]{Conjecture}
%\theoremstyle{definition}
\newtheorem{dfn}[thm]{Definition}
\newtheorem{exm}[thm]{Example}

\usepackage{amsfonts}

\newtheoremstyle{taskstyle}% name of the style to be used
  {.32\baselineskip±.15\baselineskip}% measure of space to leave above the theorem. E.g.: 3pt
  {.32\baselineskip±.15\baselineskip}% measure of space to leave below the theorem. E.g.: 3pt
  {\it}% name of font to use in the body of the theorem
  {}% measure of space to indent
  {\bf}% name of head font
  {. }% punctuation between head and body
  { }% space after theorem head; " " = normal interword space
  {}% Manually specify head
\theoremstyle{taskstyle}
\newtheorem{task}{Task}
%
% \newcounter{tasknumber}
% \newcounter{tasklabel}
% \renewcommand{\thetasklabel}{\textbf{\thetasknumber}}
% \newenvironment{task}{
% \begin{list}{\textbf{Task
%     }\thetasklabel:}{\usecounter{tasklabel}\stepcounter{tasknumber}\item\em}}{\\[-7pt]\end{list}}

\newcounter{subtasknumber}
\newcounter{subtasklabel}
\renewcommand{\thesubtasklabel}{\textbf{\thetasknumber}}
\newenvironment{subtask}{
\begin{list}{\textbf{Sub-Task }\thesubtasklabel:}{\usecounter{subtasklabel}\stepcounter{subtasknumber}\item\em}}{\\[-7pt]\end{list}}

\usepackage{csquotes}
\usepackage[normalem]{ulem}
\usepackage{pifont} % for circled numbers using the ding command
\usepackage{epsfig,times}
\usepackage{xcolor}
\usepackage{fancyhdr}
\usepackage{xspace}
%\usepackage{pdfpages}
%\usepackage[hyphens]{url}
\usepackage[colorlinks]{hyperref}
\hypersetup{
  breaklinks=true,
  linktoc=all,
  linktocpage,
  citecolor=blue,
  filecolor=blue,
  linkcolor=blue,
  urlcolor=blue,
}
%\usepackage{hyperref}
\usepackage{afterpage}
\usepackage{amsmath}
\usepackage{wrapfig}
\usepackage{alphalph}
%\usepackage{enumerate}
\usepackage{enumitem}

\usepackage{standalone}

\usepackage{booktabs} % for various \hrules
%\newcommand{\midrule}{\\[-3.5ex]}
\usepackage{fancyvrb}
%\usepackage{subfigure}
%\usepackage[footnotesize]{subfig}

\usepackage{tabulary}

\usepackage{subcaption}
\renewcommand{\thesubfigure}{(\alph{subfigure})}
\captionsetup[subfigure]{labelformat=simple, labelsep=space}
% By default, subcaption labels subfigures with things like "(a)" and "(b)", but \ref{subfigure-label} produces things like "1a" and "1b".  The makes it produce "1(a)" and "1(b)" instead.

\usepackage{tikz}
\usetikzlibrary{
  shapes,calc,arrows,fit,positioning,decorations.pathmorphing,snakes,intersections,shapes.geometric,trees,cd
}

% % the below lets us define subsubsections without line breaks after
% % section title:
% \usepackage{titlesec}
% \titleformat{\subsubsection} %
%   [runin]             % shape
%   {\bfseries}         % format
%   {\thesubsubsection} % label
%   {0.5em}             % sep
%   {}                  % before-code
%   []                  % after-code
% 
%\captionsetup{compatibility=false}
%\newenvironment{tab}[1]
{%\let\oldhline=\hline
\let\oldarraystretch=\arraystretch
\renewcommand{\arraystretch}{1.2} % make whole row higher
%\setlength{\extrarowheight}{0.2mm} % put extra space above words to avoid hlines
% also try bigstrut for really tuning hlines


%\setlength{\tabcolsep}{1.25em}
%%%\begin{tabular}{@{}#1@{}}
%%%\toprule
%%%}
%%%{\bottomrule
%%%\end{tabular}


%\setlength{\extrarowheight}{0mm}
\renewcommand{\arraystretch}{\oldarraystretch}}

%\newcommand{\href}[1]{{#1}}
%\newcommand{\href}{\url}

\newcommand{\cons}{{:}}

\makeatletter
\newcommand{\gobblepars}{\@ifnextchar{\par}{\expandafter\gobblepars\@gobble}{}}
\makeatother
\newcommand{\minisec}[1]{\smallskip\par\noindent\textbf{{#1}.}\ \ \gobblepars}
\newcommand{\subminisec}[1]{\par\textsc{{#1}.}\ \gobblepars}
\newcommand{\itemizeminisec}[1]{\smallskip\par\noindent\bullet\textbf{{#1}.}\ \gobblepars}

\newcommand*{\filter}[2]{\lfloor{#1}{\downarrow}{#2}\rfloor}
\DeclareMathOperator{\st}{\,s.t.\,}
\newcommand{\inn}{\mathrel{\in\hspace{-1ex}\in}}
\newcommand{\doo}{\mathrm{do}}
\newcommand{\about}{\mathsf{about}}


%\newcommand{\todo}[1]{{\textcolor{red}{[\textbf{Todo:} \emph{#1}]}}}
%\newcommand{\work}[1]{{\textcolor{blue}{{#1}}}}
\newcommand{\work}[1]{\emph{#1}}
%\newcommand{\todo}[1]{}
%%\newcommand{\cut}[1]{} % considered harmful.  Use % instead please.


\newcommand*{\bagleft}{\mbox{$\{\hspace{-0.85ex}\{$}}
\newcommand*{\bagright}{\mbox{$\}\hspace{-0.85ex}\}$}}
\newcommand*{\bag}[2]{\bagleft {#1} \given {#2} \bagright}
%\newcommand*{\emptybag}{\mbox{$\{\!\!\{\}\!\!\}$}}
%\DeclareMathOperator*{\Cupop}{\Cup}
%\newcommand*{\bigCup}{\Cupop\limits}
%\newcommand*{\bigCup}{\Cup}
%\newcommand*{\bag}[1]{\mathsf{bag}({#1})}
%\newcommand*{\bagunion}{\uplus}
%\newcommand*{\bigbagunion}{\textstyle{\biguplus\nolimits}} % will want to over with limits sometimes.
%\newcommand*{\Bigbagunion}[2]{\mbox{$\biguplus_{#1}^{#2}$}\,}


\newcommand{\sref}[1]{\S\ref{#1}}

\newcommand{\mc}{\mathcal}
\newcommand{\msf}{\mathsf}
\newcommand{\given}{\mathrel{|}}

%\newcommand*{\set}[2]{\{{#1} \given {#2}\}}

\newcommand{\first}{\emph{(i)}~}
\newcommand{\second}{\emph{(ii)}~}
\newcommand{\third}{\emph{(iii)}~}
\newcommand{\fourth}{\emph{(iv)}~}
\newcommand{\fifth}{\emph{(v)}~}
\newcommand{\sixth}{\emph{(vi)}~}
\newcommand{\perc}{\,\%\xspace}

\newcommand{\bc}{\begin{center}}
\newcommand{\ec}{\end{center}}
\newcommand{\vs}{\vspace*}
\newcommand{\vsa}{\vspace*{2.0 ex}}
\newcommand{\vsb}{\vspace*{3.0 ex}}

\newenvironment{linemize}
{\renewcommand{\item}{\relax}
\newcommand{\ho}{ho}}
{}


\let\vect=\vec
\renewcommand{\vec}[1]{\mathbf{#1}}
\newcommand{\R}{\mathbb{R}}
\newcommand{\B}{\mathbb{B}}
\newcommand{\E}{\mathbb{E}}
\newcommand{\A}{\mathcal{A}}
\newcommand{\X}{\mathcal{X}}
\newcommand{\Y}{\mathcal{Y}}
\newcommand{\D}{\mathcal{D}}
\newcommand{\N}{\mathcal{N}}
\newcommand{\x}{\mathbf{x}}
\newcommand{\y}{\mathbf{y}}
\newcommand{\xbold}{\mathbf{x}}
\newcommand{\ubold}{\mathbf{u}}
\newcommand{\infl}{\iota}
\newcommand{\minfl}{\mu}
\newcommand{\piexp}{\tilde{\pi}}
\newcommand{\true}{\mathit{true}}
\newcommand{\false}{\mathit{false}}
\newcommand{\tup}[1]{\langle #1 \rangle}
\newcommand{\SV}{\varphi}
\newcommand{\banzhaf}{\beta}
\newcommand{\DP}{\delta}
\newcommand{\euler}{\mathrm{e}}
\newcommand{\eps}{\varepsilon}
\newcommand{\ind}{\text{ind}}
\newcommand{\indact}{\text{ind-act}}
\newcommand{\indav}{\text{ind-avg}}
\newcommand{\grp}{\text{grp}}
\newcommand{\disp}{\text{disp}}
\newcommand{\diffp}{\text{dp}}
\newcommand{\lr}{\text{lr}}
%\renewcommand{\Pr}{\text{Pr}}
\newcommand{\qoi}{Q}
\newcommand{\Expect}{{\rm I\kern-.3em E}}
\newcommand{\Indicator}{\mathbbm{1}}

% % *** Use these commands for commenting ***
% \long\def\authornote#1{%
%   \leavevmode\unskip\raisebox{-3.5pt}{\rlap{$\scriptstyle\diamond$}}%
%   \marginpar{\raggedright\hbadness=10000
%     \def\baselinestretch{0.8}\tiny
%     \it #1\par}}
% \newcommand{\yair}[1]{\authornote{Yair: #1}}
%\newcommand{\anupam}[1]{\authornote{Anupam: #1}}


% \newcounter{mynote}[section]
% \newcommand{\notecolor}{blue}
% \newcommand{\thenote}{\thesection.\arabic{mynote}}
% \newcommand*{\authornote}[2]{\refstepcounter{mynote}\textcolor{#1}{[\thenote {#2}]}}

% To remove notes, just uncomment the line below:
\def\final{}

\ifdefined\final
  \newcommand{\authornote}[2]{}
\else
  \newcommand*{\authornote}[2]{\textcolor{#1}{[#2]}}
\fi
\newcommand*{\anupam}[1]{\authornote{purple}  {AD:  {#1}}}
\newcommand*{\mct}   [1]{\authornote{magenta}{MCT: {#1}}}
\newcommand*{\shayak}[1]{\authornote{red}    {SS:  {#1}}}
\newcommand*{\amit}  [1]{\authornote{blue}   {Amit:{#1}}}
\newcommand*{\matt}  [1]{\authornote{red}    {MF:  {#1}}}
\newcommand*{\tnote} [1]{\authornote{cyan}   {TomR:{#1}}}
\newcommand*{\pxm}   [1]{\authornote{teal}   {PM:  {#1}}}
\newcommand*{\helen} [1]{\authornote{purple} {HN:  {#1}}}
\newcommand*{\seb} [1]{\authornote{blue} {SB:  {#1}}}
\newcommand*{\instructions}[1]{\authornote{gray}{#1}}
%\newcommand{\tnote}[1]{\refstepcounter{mynote}{\bf \textcolor{cyan}{$\ll$TomR~\thenote: {\sf #1}$\gg$}}}
%\newcommand{\pxm}[1]{\refstepcounter{mynote}{\textcolor{teal}{PM~\thenote: #1}}}


\newcommand{\dist}[1]{\mathbb{D}\paren{#1}}
\newcommand{\paren}[1]{\left( #1 \right)}
\newcommand{\sparen}[1]{\left[ #1 \right]}
\newcommand{\ra}{\rightarrow}
\newcommand{\Ra}{\Rightarrow}
\newcommand{\la}{\leftarrow}
\newcommand{\La}{\Leftarrow}
\newcommand{\stacklabel}[1]{\stackrel{\smash{\scriptscriptstyle \mathrm{#1}}}}
\newcommand{\defeq}{\stacklabel{def}=}
\newcommand{\defeqq}{\stacklabel{def?}=}

\newcounter{ctr}
\newcounter{savectr}
\newcounter{ectr}

\newenvironment{newitemize}{%
\begin{list}{$\bullet$}{\labelwidth=9pt%
\labelsep=7pt \leftmargin=16pt \topsep=1pt%
\setlength{\listparindent}{\saveparindent}%
\setlength{\parsep}{\saveparskip}%
\setlength{\itemsep}{1pt} }}{\end{list}}

\newenvironment{newenum}{%
\begin{list}{{\rm (\arabic{ctr})}\hfill}{\usecounter{ctr} \labelwidth=18pt%
\labelsep=7pt \leftmargin=25pt \topsep=3pt%
\setlength{\listparindent}{\saveparindent}%
\setlength{\parsep}{\saveparskip}%
\setlength{\itemsep}{2pt} }}{\end{list}}

% \makeatletter
% \renewcommand{\paragraph}{%
%   \@startsection{paragraph}{4}%
%   {\z@}{1.25ex \@plus 1ex \@minus .2ex}{-1em}%
%   {\normalfont\normalsize\bfseries}%
% }
% \makeatother

\newcommand*{\Fr}{\mathrm{Fr}}
\newcommand*{\Cr}{\mathrm{Cr}}
\newcommand*{\from}{\colon}
\newcommand*{\compose}{\circ}
\newcommand*{\set}[2]{\{{#1}~|~{#2}\}}
\newcommand{\powerset}[1]{2^{{#1}}}
\newcommand{\tvar}[1]{$\mathtt{#1}$}

% for fixed width table columns
\usepackage{array}
\newcolumntype{L}[1]{>{\raggedright\let\newline\\\arraybackslash\hspace{0pt}}m{#1}}
\newcolumntype{C}[1]{>{\centering\let\newline\\\arraybackslash\hspace{0pt}}m{#1}}
\newcolumntype{R}[1]{>{\raggedleft\let\newline\\\arraybackslash\hspace{0pt}}m{#1}}


\usepackage[T1]{fontenc}
\newcommand*{\lt}{{\textup{\guilsinglleft}}}
\newcommand*{\rt}{{\textup{\guilsinglright}}}

\usepackage{stmaryrd}
\newcommand*{\lbb}{\llbracket}
\newcommand*{\rbb}{\rrbracket}

